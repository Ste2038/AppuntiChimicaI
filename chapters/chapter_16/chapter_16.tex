\section{Entropia, energia libera e direzione delle reazioni}
\subsection{Trasformazioni spontanee}
Viene definita trasformazione spontanea un processo che si svolge da solo (al massimo neecessità di un input instantaeo per avviarsi, come una scintilla), assorbendo o cedendo calore.\\
Quando la trsformazione inizia, prosegue senza intervento esterno perchè il sistem arilascia energia sufficiente all'ambiente per far svolgere la trasformazione.\\
Di contro una trasformazione non spontanea è un processo che ha continuo bisogno di energia dall'ambiente.\\
Spontanea non significa istantanea, non ha nulla a che fare con la durata della trasformazione.
\subsection{Entropia}
Ogni molecola di un gas in ogni istante assume un particolare stato determinato dallo stato elettronico, traslazionale, rotazionale, vibrazionale. E' chiaro che per una singola molecola sono possibili molte combinazioni di questi stati e il numero di stati energetici quantizzati è enorme.\\
Una combinazione di questi stati prende il nome di microstato (W).\\
Ludwig Boltzmann correlò il numeor di microstati (W) con l'entropia (S) del sistema: $S = k_b * lnW$ da cui deriva che:
\tab- Un sistema che può distribuire la sua energia su W minore ha entropia minore.\\
\tab- Un sistema che può distribuire la sua energia su W maggiore ha entropia maggiore.\\\\
Se W aumenta durante una trasformazione, ci sono più modi in cui l'energia del sistema può essere distribuita tra essi. Ciò comporta un aumento di entropia, e viceversa.\\
%Fine slide 4
E' una funzione di stato: $\Delta S = S_{finale} - S_{iniziale}$.\\
Un metodo per calcolare $\Delta S$ è basato sugli scambi di calore che avvengono durante le trasformazioni: $\Delta S = \frac{q_{rev}}{T}$\\
In cui T è la temperatura a cui avviene lo scambio di calore; q è il calore assorbito, il pedice rev si riferisce a un processo reversibile.\\
In generale se l'entropia aumenta il sistema risulta meno ordinato, mentre se l'entropia dimnuisce il sistema risulta più ordinato.\\
Le reazioni spontanee sono quelle in direzione di aumento dell'entropia dell'universo. Questo è il secondo principio della termodinamica.\\
Ne consegue che per una reazione spontanea si ha $\Delta S_{universo} = \Delta S_{sistema} + \Delta S_{ambiente} > 0$.\\\\
L'entropia può essere misurata in valore assoluto, a differenza dell'entalpia che si misura solo relativamente.\\
\paragraph*{Terzo principio della termodinamica} :\\
Un cristallo perfetto ha entropia zero alla temperatura dello zero assoluto: $S_{sistema} = 0$ a 0 K.\\
Dove perfetto significa che tutte le particelle sono disposte ordinatamente nelal struttura cristallina senza alcun tipo di difetto. Ne consegue W = 1, quindi $S = k_b * ln1 = 0$\\
Come molte altre variabili termodinamiche, quando si confrontano fra loro nei stati standard alla temperatura di interesse si utilizza l'entropia molare standard($S^0$) misurata in $\frac{[J]}{[mol][K]}$.\\
\subsubsection{Effetti dell'entropia}
A livello molecolare si è in grado di prevedere alcuni effetti:\\
\tab- Variazioni di temperatura: per una data sostanza, $S^0$ aumenta all'aumentare della temperatura.\\
\tab- Stati fisici e transizioni di fase: per una data sostanza, $S^0$ aumenta quando la sostanza si trasforma da solido a liquido e da liquido a gas.\\
\tab- Dissoluzione di un solido o di un liquido: l'entropia di un solido disciolto o di un soluto liquido, è di solito maggiore dell'entropia del soluto puro, ma il tipo di soluto e di solvente e la natura del processo di dissoluzione influenzano la variazione di entropia complessiva. Per soluti di tipo molecolare o liquidi, la variazione etropica è molto contenuta (ma positiva), ed è principalmente causata dal mescolamento casuale dei due tipi di molecole.\\
\tab- Dissoluzione di un gas: comporta sempre una diminuzione di entropia, perchè le particelle gassose disciolte in una fase condensata non avranno mai libertà di movimento e dispersione di energia che hanno allo stato gassoso. Invece, se un gas si scioglie in un altro gas, l'entropia aumenta a causa del mescolamento dei due tipi molecole.\\
\tab- Raggio atomico o complessità molecolare: per gli elementi in un gruppo, il raggio atomico riflette la massa molare, e $S^0$ aumenta scendendo lungo il gruppo; lo stesso vale per gli elementi di quel gruppo quando vengono legati ad un altro elemento fisso. Nel caso di forme allotropiche, l'entropia è maggiore nelal forma con legami che permettono agli atomi un movimento maggiore. Nel caso dei composti, l'entropia aumenta con la complessità chimica, cioè col numero di atomi nel composto; questa tendenza si basa sul tipo di movimento disponibile agli atomi e vale soltanto per specie aventi lo stesso stato d'aggregazione.\\\\
Si può spesso prevedere il segno dell'entropia standard di reazione ($\Delta S^0_r$). Un fattore decisivo è la variazione del numero di moli di gas nel passaggio da reagenti a prodotti. \\
Il valore di $\Delta S_r^0$ è generalmente positivo se il numero di moli aumenta e viceversa.\\
$\Delta S_r^0 = \Sigma mS^0_{prodotti} - \Sigma nS^0_{reagenti}$, dove m e n sono i coefficienti stechiometrici dei singoli prodotti e reagenti della reazione bilanciata.\\
Le reazioni caratterizzate da $\Delta S_r^0 < 0$ sono comunque vincolate al secondo principio della termodinamica:\\
l'entropia del sistema può diminuire solo se l'entropia dell'ambiente aumenta in misure superiore. Il ruolo dell'ambiente è quello di fornire o sottrarre calore al sistema e può partecipare distintamente per:\\
\tab- Trasformazioni esotermiche: l'ambiente acquista calore dal sistema, e aumenta di entropia.\\
\tab- Trasformazioni endotermiche: il calore acquistato dal sistema è ceduto dall'ambiente, equest'ultimo vede diminuire la propria entropia.\\
In sostanza, la variazione entropica dell'ambiente è direttamente proporzionale alla quantità di calore trasferita al o dal sistema e, sopratutto, a una variazione opposta di calore del sistema.\\
Anche la temperatura dell'ambiente prima del trasferimento di calore influenza la sua variazione. Ad esempio una reazione che inizia a 20°C produrra più calore di una che iniziera a temperatura più alta, in quanto in questa seconda situazione l'energia è già distribuita in un ampio numero di microstati.\\
Ne consegue: $\Delta S_{ambiente} = -\frac{q_{sistema}}{T}$.\\
A pressione costante: $\Delta S_{ambiente} = -\frac{\Delta H_{sistema}}{T}$. Ciò significa che è possibile calcolare $\Delta S_{ambiente}$ misurando $\Delta H_{sistema}$ e la temperatura a cui avviene la trasformazione.\\\\
E' pertanto possibile giustificare perchè avvengono sia reazioni esotermiche che endoteriche spontanee. Indipendentemente da quale sia la variaione di entalpia, una reazione avviene perchè l'entropia totale (del sistema e dell'ambiente) aumenta. Ci sono due possibilità:\\
\tab- Reazione esotermica ($\Delta H_{sistema} < 0$): il sistema cede calore, aumentando l'entopia dell'ambiente ($\Delta S_{ambiente} > 0$):\\
\tab\tab- Se $\Delta S_{sistema} > 0$: La variazione di entropia totale sarà positiva.\\
\tab\tab- Se $\Delta S_{sistema} < 0$: $\Delta S_{ambiente}$ deve aumentare in misura maggiore per rendere positiva la variazione di entropia totale\\
\tab- Reazione endotermica ($\Delta H_{sistema} > 0)$: il calore ceduto dall'ambiente porta ad una variazione esempio la dissoluzione di molti composti ionici.
\subsection{Energia libera di Gibbs}
L'energia libera di Gibbs (G) è una funzione di stato che combina l'entalpia e l'entropia del sistema : $G = H- T* S$.\\
La variazione di energia libera ($\Delta G$) è una misura della spontaneità di una trasformazione e dell'energia utile che se ne può ottenere. Considerando la reazione $\Delta S_{ambiente} = -\frac{\Delta H_{sistema}}{T}$ e l'equazione $\Delta S_{universo} = \Delta S_{ambiente} + \Delta S_{sistema}$ si ottiene $\Delta S_{universo} = -\frac{\Delta H_{sistema}}{T} + \Delta S_{sistema}$.\\
Moltiplicando entrambi i temrini per -T, si ottiene $-T * \Delta S_{universo} = \Delta H_{sistema} T * \Delta S_{sistema}$\\
L'energia libera di Gibbs sostituisce a secondo membro: $\Delta G_{sistema} = \Delta H_{sistema} - T * \Delta S_{sistema}$.\\
Si ricava quindi che: $\Delta G_{sistema} = -T * \Delta S_{universo}$\\
Considerando il seocnod principio della termodinamica stabilisce che una trasformazione è spontanea se $\Delta S_{universo} > 0$ (e viceversa) e ricordando che T è misurata in [K] (quindi è sempre positiva), si può affermare che il segno di $\Delta G$ determina se:\\
\tab- $\Delta G < 0$: trasformazione spontanea\\
\tab- $\Delta G > 0$: trasformazione non spontanea\\
\tab- $\Delta G = 0$: trasformazione all'equilibrio\\
Ovviamente, se una reazione non è spontanea in una direzione, è spontanea nella direzione opposta.\\\\
$\Delta G$ è pertanto la funzione di stato che rappresenta il massimo lavoro utile non espansivo compiuto da un sistema durante una trasformazione spontanea a temperatura costante.\\\\
Come per altre variabili di stato, per essere confrontata con altre reazione c'è necessita di una variazione di energia libera standard ($\Delta G^0$), che si ha quando tutti i componenti del sistema sono allo stato standard: $\Delta G_{sistema}^0 = \Delta H_{sistema}^0 - T * \Delta S_{sistema}^0$.\\
Alternativamente è possibile utilizzare l'energia libera standard di formazione ($\Delta G_f^0$) dei componenti, cioè la variazione di energia libera che avviene quando viene formata 1 mol di composto allo stato standard a partire dai suoi elementi.\\
Essendo l'energia libera una funzione di stato si può scrivere la reazione: $\Delta G^0_r = \Sigma mG^0_{prodotti} - \Sigma nG_{reagenti}^0$, dove m e n sono i coefficienti stechiometrici dei singoli prodotti e reagenti della reazione bilanciata. Analogamente a quanto visto per $\Delta H_f^0$ si ha che $\Delta G_f^0 = 0$ per un elemento nel suo stato standard.\\\\
Nella maggior parte dei casi il contributo $\Delta H$ è molto maggiore del contributo di $T * \Delta S$ nel calcolo di $\Delta G$. Per questo motivo, la maggior parte delle reazioni esotermiche sono spontanee. Però la temperatura a cui avviene una reazione influenza il valore del termine $T * \Delta S$, pertanto si può affermare che in molti casi la spontaneità complessiva dipende dalla temperatura.\\
Un controllo di $\Delta H$ e $\Delta S$ permette di prevedere l'effeto della temperatura sul segno di $\Delta G$ e quindi sulla spontaneità di una trasformazione a qualsiasi temperatura.\\
I seguenti casi:\\
\tab- Reazione spontanea a qualsiasi temperatura: $\Delta H < 0 $ e $ \Delta S > 0$ porteranno sempre ad avere $\Delta G < 0$. (Combustione)\\
\tab- Reazione mai spontanea a qualsiasi temperatura: $\Delta H > 0$ e $\Delta S < 0$ porteranno sempre ad avere $\Delta G > 0$. Avvengono solo se l'ambiente fornisce sufficiente energia\\
\tab- Reazione spontanea ad alte temperature: $\Delta H > 0$ e $\Delta S > 0$ porteranno ad avere $\Delta G < 0$ solo se T è sufficientemente alta.\\
\tab- Reazione spontanea a basse temperature: $\Delta H < 0$ e $\Delta S < 0$ porteranno ad avere $\Delta G < 0$ solo se T è sufficientemente bassa.\\\\
\subsubsection{Energia libera e costante di equilibrio}
E' possibile dimostrare che $\Delta G$ di un sistem aè la differenza tra il suo valore in uno stato iniziale (Q) e il suo valore nello stato finale (K): per un sistema all'equilibrio Q = K, quindi $\Delta G = 0$