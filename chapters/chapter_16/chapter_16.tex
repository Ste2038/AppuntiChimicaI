\section{Entropia, energia libera e direzione delle reazioni}
\subsection{Trasformazioni spontanee}
Viene definita trasformazione spontanea un processo che si svolge da solo (al massimo neecessità di un input instantaeo per avviarsi, come una scintilla), assorbendo o cedendo calore.\\
Quando la trsformazione inizia, prosegue senza intervento esterno perchè il sistem arilascia energia sufficiente all'ambiente per far svolgere la trasformazione.\\
Di contro una trasformazione non spontanea è un processo che ha continuo bisogno di energia dall'ambiente.\\
Spontanea non significa istantanea, non ha nulla a che fare con la durata della trasformazione.
\subsection{Entropia}
Ogni molecola di un gas in ogni istante assume un particolare stato determinato dallo stato elettronico, traslazionale, rotazionale, vibrazionale. E' chiaro che per una singola molecola sono possibili molte combinazioni di questi stati e il numero di stati energetici quantizzati è enorme.\\
Una combinazione di questi stati prende il nome di microstato (W).\\
Ludwig Boltzmann correlò il numeor di microstati (W) con l'entropia (S) del sistema: $S = k_b * lnW$ da cui deriva che:
\tab- Un sistema che può distribuire la sua energia su W minore ha entropia minore.\\
\tab- Un sistema che può distribuire la sua energia su W maggiore ha entropia maggiore.\\\\
Se W aumenta durante una trasformazione, ci sono più modi in cui l'energia del sistema può essere distribuita tra essi. Ciò comporta un aumento di entropia, e viceversa.\\
%Fine slide 4