\section{Lo stato gassoso}
\subsection{Proprietà dei gas}
\tab- Il volume di un gas varia in base a temperatura e pressione\\
\tab- I gas hanno viscosità relativamente bassa e questo permette di viaggiare all'interno di un tubo o di uscire da un foro.\\
\tab- Un gas ha densità relativamente bassa (circa 1000 volte più piccola di un liquido)\\
\tab- I gas sono completamente miscibili fra loro in qualsiasi proporzione.
\subsection{Pressione}
Forza esercitata su una superficie diviso l'are della superficie.\\
$p = \frac{F}{A}$\\
\paragraph*{Barometro}:\\
Il barometro è lo strumento utilizzato per misurare la pressione.\\
La pressione atmosferica è uguale a 760mm di mercurio (questa era la pressione che si era spostata utilizzando mercurio contro la pressione atmosferica).\\\\
\paragraph*{Manometro}:\\
Un altro misuratore di pressione.\\
Ne esiste di due tipi:\\
\tab- A tubo chiuso: costituito da un tubo curvo pieno di mercurio chiuso ad un'estremità. Quando nel recipiente posto all'altra estremità è vuoto, i livelli di mercurio dei due rami del tubo sono uguali perchè nessun gas esercita pressione sull'altra superficie di mercurio. Quando il recipiente contiene un gas, esso fa scendere il libello di Mercurio nel ramo più vicino a lui. La differenza di altezze fra i due rami indica la pressione del gas.\\ 
\tab- A tubo aperto: costituito da un tubo curvo pieno di mercuirio con estremità aperta e l'altra collegata ad un recipiente. A differenza del precedente, questa volta in presenza di un gas c'è una pressione su entrambe le superfici, una data dal gas, l'altra dalla pressione atmosferica.\\\\
Unità di misura\\
Pa: pascal\\
atm: atmosfera 1 atm = 101325Pa\\
mmHg/tor: millimetri di mercurio = 1/760 atm\\
bar: bar  $1 bar = 10^5\ Pa$
\subsection{Gas Perfetti}
\tab- Costituito da un gran numero di particelle, il numero di particelle è molto piccolo rispetto al volume di particelle, quindi hanno massa ma non volume.\\
\tab- Le particelle di gas sono sempre in continuo movimento rettilineo casuale.\\
\tab- Gli urti sono elastici ovvero non c'è scambio di energia, quindi niente attrito.\\
\tab- Tra particelle di gas non ci sono forze.\\
Sono concentrazioni gassose si esprimono in \% (\%vol = \%moli)
\subsection{Legge di Boyle}
Isoterma (temp = const)\\
$PV = $const, pressione e volume sono inversamente proporzionali.
\subsection{Legge di Charles}
Isobara (P = const)\\
$\frac{V}{T} =$ const, volume e temperatura sono direttamente proporzionali.
\subsection{Legge di Gay-Lussac}
Isocora (V = const)\\
$\frac{P}{T} =$ const, pressione e temperatura sono direttamente proporzionali.
\subsection{Legge di Avogadro}
$\frac{V}{n} =$ const, volue e moli sono direttamente proporzionali.
\subsection{Equazione di stato dei gas perfetti [Avogadro]}
$P\ V\ = n\ R\ T$\\
R = costante di un gas = $0,08206\ atm\ L\ mol^{-1}\ K^{-1}$\\
in condizione normale(NTP) 0C(273K) e 1atm\\
in queste condizioni 1 mol di gas perfetto occupa 22,4L.
\subsection{Pressioni Parziali}
Sapendo che i gas si possono mischiare perfettamente, sappiamo che ogni gas del composto esercita la sua pressione parziale ($p_i$) e sappiamo che la somma di tutte le $p_i$ è uguale a P.\\
$p_i = \frac{n_i\ R\ T}{V}$\\
\paragraph*{Frazione molare}:\\
$x_i = \frac{n_i}{n_{tot}} * 100$
\subsection{Energia Cinetica}
$E_k = \frac{1}{2}m\ V^2$\\
Energia cinetica media $E_k = \frac{1}{2}\ MM\ V^2$\\
Energia cinetica media dei gas $E_k = \frac{3}{2}\ k_B\ T$\\
Velocità quadratica media $V_{qm} = \sqrt{V^2} = \sqrt{\frac{3RT}{MM}}$\\ 
\subsection{Effusione} 
Teoria cinetica con cui il gas riesce tramite un foro a riempire un recipiente adiacente con pressione più bassa.\\
Nel casodue gas volevesser passare dal foro.\\
$\frac{V_a}{V_b}=\frac{\sqrt{MM_b}}{\sqrt{MM_a}}$\\
In questo caso tentano di eseguire una reazione ma invece di capitare al centro, capitava un lato poichè uno dei due composti era più veloce.
\subsection{Atmosfera}
\tab- più sali, meno pressione\\
\tab- la temperatura ha un andamento molto irregolare\\
\tab- composizione:\\
\tab\tab- omosfera(<80000 metri)\\
\tab\tab- eterosfera(>80000 metri)\\
\tab\tab- nell'omosfera i gas si mischiano per via della miscelazione convettiva\\
\tab\tab- nell'eterosfera i composti non si mischiano e stratificano dal più leggero al più pesante.\\
\subsection{Equazione di Van der Waals}
$\left(p+\frac{n^2a}{V^2}\right)\left(V-nb\right) = nRT$\\
Dove a, b sono costanti di Van der Waals: numeri positivi determinati sperimentalmente per ogni gas. In particolare:\\
\tab- a: termine correttivo della pressione ed esprime l'intensità delle interazioni tra molecole gassose.\\
\tab- b: è il covolume, cioè il volume occupato dalle molecole di una mole di gas. A basse pressioni tale volume è trascurabile rispetto a quello in cui si muovono le molecole, mentre non lo è a pressioni molto elevate.
