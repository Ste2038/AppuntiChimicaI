\section{Equilibri acido-base e di solubilità in soluzioni acide}
\subsection{Teoria di Arrhenius}
Arrhenius fu il primo a dare una definizione di acidi e basi in acqua nel 1884:\\
\tab- Acido: una sostanza che contiene H nella sua formula e si dissocia in acqua per produrre $H_3O^+_{(aq)}$. (Es. HCl)\\
\tab- Base: Una sostanza che contiene OH nella sua formula e si dissocia in acqua per produrre $OH^-_{(aq)}$. (Es. NaOH)\\
Si noti che le basi di Arrhenius sono composti ionici e si "dissociano", mentre gli acidi sono composti covalenti che "ionizzano". Tuttavia si utilizza il termine "dissociazione" per entrambi.\\
Quando un acido e una base reagiscono tra loro subiscono una "neutralizzazione". Lo ione $H_3O^+_{(aq)}$ proveniente dall'acido di Arrhenius sta nella sua incapacità di giustificare il comportamento basico di alcune sostanze prive di OH nella formula (Es: $NH_3$).\\\\
Sia acidi che basi si differenziano poi in forti e deboli in acqua. Questo fa riferimento alla loro capacità di dissociarsi in acqua:\\
\tab- Acidi o basi forti: Si dissociano completamente in acqua.\\
\tab- Acidi o basi deboli: Si dissociano parzialmente in acqua.\\
\subsubsection{Costante di dissociazione acida}
Una costante introdotta per confrontare la "forza" di un acido. Più $K_a$ è grande, più l'acido sarà forte.\\
$K_a = \frac{[H_3O^+]*[A^-]}{[HA]}$\\
Le parentesi quadre indicano la molarità della specie in condizioni di equilibrio.\\
\subsubsection{Alcuni acidi e basi}
L'impiego di tabelle riportanti le costanti di dissociazione è il modo più sicuro per quantificare la forza di un acido e basi. Riporto alcuni esempi:\\
\tab- Acidi forti: HCl, HBr, HI, $HNO_3$, $H_2SO_4$, $H_2SeO_4$ e $HClO_4$. Si noti che gli ossiacidi forti devono avere un numero di atomi di O superiore di due o più unità rispetto al numero di atomi di H.\\
\tab- Acidi deboli: tutti gli altri, tra i quali si citano i più ricorrenti: HF, HCN, $H_2S$, $HNO_2$, $H_2SO_3$, $H_3PO_4$, $HClO_3$, $HClO_2$, $HClO$, $RCOOH$ (dove R è detto gruppo alchilico, di formula generale $-C_nH_{2n+1}$, e contraddistingue la famiglia degli acidi organici).\\
\tab- Basi forti: idrossidi dei gruppi 1 e 2. Indirettamente, sono quindi basi forte anche gli ossidi basici dei gruppi 1 e 2 in quanto la loro reazione con acqua produce un idrossido.\\
\tab- Basi deboli: $NH_3$ e ammine, cioè composti organici in cui uno o più atomi di H in $NH_3$ sono sostituiti da gruppi R (Es.$CH_3CH_2NH_2$). Si noti che nessuna di queste è una base di Arrhenius, pertanto la definizione di costante di dissociazione basica verrà discussa da una teoria successiva.\
\subsection{Autoionizzazione (o Autoprotolisi)}
L'acqua è un elettrolita debole, infatti si dissocia lievemente in ioni mediante un procesos di autoionizzazione:\\
$H_2O_{(l)} + H_2O_{(l)} \Leftrightarrow H_3O*+_{(aq)}+ OH^-_{(aq)}.$\\
Di questa reazione si può scrivere l'espressione della $K_c$ e quello che si ottiene è una nuova costante detta "prodotto ionico dell'acqua $K_w = \left[H_3O^+\right] * [OH^-] = 10^{-14} (a 25°C)$"\\
Abbiamo quindi trovato che l'acqua pura contiene $\left[H_3O^+\right] = \left[OH^-\right] = 10^{-17} M$\\
Molarità: 55.5 M\\
Soltanto una molecola su 555 si dissocia in ioni.\\
Conseguenze:\\
\tab- Una variazione di $\left[H_3O^+\right]$ determina una variazione inversa di $\left[OH^-\right]$, e viceversa.\\
\tab- Entrambi gli ioni sono presenti in tutti i sistemi acquosi, ed è possibile parlare di:\\
\tab\tab- Soluzione acida: $\left(\left[H_3O^+\right] > \left[OH^-\right]\right)$\\
\tab\tab- Soluzione basica: $\left(\left[H_3O^+\right] < \left[OH^-\right]\right)$\\
\tab\tab- Soluzione neutra: $\left(\left[H_3O^+\right] = \left[OH^-\right]\right)$
\subsection{Scala p e pH}
Nelle soluzioni acquosa $\left[H_3O^+\right]$ ha un range di $\left[10^{-15}, 10\right]$. Per semplificare si introduce la scala p.\\
Applicando la scala p a $\left[H_3O^+\right]$ otteniamo il pH, tramite la relazione: $pH = -log\left[H_3O^+\right]$.\\
Se ad esempio in soluzione acquosa abbiamo $\left[H_3O^+\right]= 10^{-3}$ avremo pH = 3.\\
Analogamente possiamo ottentere $\left[H_3O^+\right] = 10^{-pH}$.
\subsubsection{Classificazione}
In soluzioni acquose si parla di:\\
\tab- Soluzione acida: se $pH < 7$\\
\tab- Soluzione basica: se $pH > 7$\\
\tab- Soluzione neutra: se pH = 7
\subsubsection{Altri utilizzi di scala p}
La scala p può anche essere utilizzata per esprimere:\\
\tab- Concentrazione di ioni ossidrile: $pOH = -log[OH^-]$\\
\tab- Costante di equilibrio: $pK = -log\ K$
\subsubsection{Misurare il pH}
A livello sperimentale i valori del pH possono essere ottenuti con:\\
\tab- pHmetro: misura $\left[H_3O^+\right]$ mediante due elettrodi immersi nella soluzione acquosa da studiare. 
\tab- Indicatore acido-base: è una molecola organica il cui colore dipende dall'acidità o dalla basicità della soluzione in cui è disciolta. Ne è un esempio la cartina indicatrice di pH, cioè una striscia di carta impregnata di un indicatore o di una miscela di indicatori. Si determina il pH in base al colore che assume controntato con una tabella di riferimento.
\subsection{Teoria di Bronsted-Lowry}
Uno dei limiti della teoria di Arrhenius è che molte sostanze che producono ioni $OH^-$ in acqua non contengono OH nella loro formula. Inoltre, la teoria vincola la relazione acido-base alla presenza di acqua. Successivamente, Nicolaus Bronsted e Thomas Lowry ridefinironno acidi e basi:\\
\tab- Acido: è qualsiasi specie in grado di donare protoni, deve quindi contenere H nella formula. Tutti gli acidi di Arrhenius sono acidi di Bronsted-Lowry.\\
\tab- Base: è una qualsiasi specie in grado di accettare protoni, deve contenere una coppia solitaria di elettroni per legare un protone. Le basi di Bronsted-Lowry non sono basi di Arrhenius, ma tutte le basi di Arrhenius contengono basi di Bronsted-Lowry.\\
Una reazione acido-base ha pertanto come requisito la copresenza di una specie che doni protone e una che lo accetti. Queste reazioni possono avvenire tra gas, in soluzioni acquose, e non acquose e in miscele eterogenee.\\
Resta però imprescindibile la simultaneità di una specie che si comporti da acido e una che si comporti da base.\\
L'acqua è detta specie anfotera in quanti si comporta sia da acido che da base a seconda dei casi.\\
Le reazioni hanno questa formula: $Acido_{(1)} + Base_{(2)} \Leftrightarrow Base_{(1)} + Acido_{(2)}$. Ogni acido ha una base coniugata e viceversa, secondo le seguenti regole:\\
\tab- La base coniugata della coppia ha un H in meno e una carica negativa in più rispetto all'acido. Ad esempio $Cl^-$ è la base di HCl.\\
\tab- L'acido coniugato della coppia ha un H in più e una carica negativa in meno rispetto alla base. Ad esempio $NH_4^+$ è l'acido coniugato di $NH_3$\\
Si ricordi che acidi e basi possono essere specie neutre, ma anche cationi e anioni. Inoltre la stessa specie può essere un acido o una base, a seconda dell'altra specie che partecipa alla reazione.\\
La direzione della reazione acido-base dipende dalle forze relative degli acidi e delle basi che partecipano alla reazione.\\
La reazione procede nel verso in cui un acido più forte e una base più forte formano un acido più debole e una base più debole.\\
Se è presente un acido più debole, esso avrà una base coniugata più forte, e viceversa.\\
E' quindi possibile costruire una scala delle forze delle coppie acido-base coniugate. La reazione procederà verso destra se l'acido reagisce con una base che è più bassa nella scala, perchè questa combinazione produce una base coniugata più debole (rispetto alla base nei reagenti) e un acido più deboli (rispetto all'acido dei reagenti).\\
Al diminuire della concentrazione dell'acido il grado di dissociazione ($\alpha$) dell'acido aumenta.\\
Es. un acido HA con $K_{\alpha} = 1.3*10^{-5}$ presenta questi casi:\\
\tab- se $[HA]_i = 0.10\ M, \alpha = 1.1\%$\\
\tab- se $[HA]_i = 0.010\ M, \alpha = 3.6\%$\\\\
Questo comportamento è in accordo con Le Chatelier: infatti si vede la soluzione di acido più diluita come una soluzione in cui sia sta aggiunta più acqua, e per risposta il sistema descritto dalla reazione.\\
$HA_{(aq)} + H_2O_{(l)} \Leftrightarrow A^-_{(aq)} + H_3O^+_{(aq)}$ tende a destra.\\
\subsubsection{Acidi poliprotici}
Gli acidi con più di un protone ionizzabile sono detti acidi poliprotici.\\
In soluzioni questi acidi dissociano un protone alla volta e ciascuno stadio di dissociazione ha un valore $K_{\alpha}$.\\
In generale notiamo che il primo si dissocia maggiormente del secondo e così via.\\
Negli esercizi occorre ricordarsi che acidi e basi politropici forti liberano una molarità doppia o tripla di ioni idronio e ossidrile.
\subsubsection{Forza di un acido}
La forza di un acido dipendedalla sua capacità di donare un protone, la quale in termini molecolari dipende a sua volta dalla forza del legame con il protone acido. \\
Con riferimento agli idracidi, la facilità con cui un protone viene rilasciato è determinata dall'elettronegatività del metallo (E) e dalla forza del legame H-E. Le tendenze periodiche indicano:\\
\tab- La forza degli idracidi aumenta lungo un periodo.\\
\tab- La forza degli idracidi aumenta lungo un gruppo. \\\\
Con riferimento agli ossiacidi, la facilità con cui un protone viene rilasciato è determinata dall'elettronegatività del non metallo (E) e dal numero di atomi O. Le tendenze periodiche indicano:\\
\tab- La forza degli ossiacidi aumenta al crescere dell'elettronegatività di E, a parita di atomi O attorno ad E.\\
\tab- La forza degli ossiacidi aumenta al crescere del numero di atomi O attorno ad E.\\\\
\subsubsection{I problemi sugli equilibri acido-base}
Seguono un approccio simile a quelli degli equilibri in fasse gassosa. Ad esempio, un problema di calcolo del pH di un acido ($K_\alpha$ nota) prevede qusti punti:\\
\tab- Scrivere la reazoine di dissociazione acida e l'espressione algebrica di $K_\alpha$\\
\tab- Costruire lo schema IVE, indicando con x la variazione della concentrazione tra stato iniziale e stato di equilibrio.\\
\tab- Sostituire i valori ottenuti al fondo dello schema IVE all'interno dell'espressione della $K_\alpha$.\\
\tab- Risolvere l'equazione algebrica in funzione di x, tenendo conto che x si può semplicare nelle somme algebriche senza commettere un errore rilevante se $K_\alpha \le 10^{-4} e HA \ge 10^{-2}\ M$.
\subsubsection{Costante di dissociazione basica}
Una soluzione acquosa di una base debole è descritta da questa equazione: $B_{(aq)} + H_2O_{(l)} \Leftrightarrow BH^+_{(aq)} + OH^-_{(aq)}$.\\
In analogia per quanto visto dagli acidi deboli è possibile scrivere la costante di dissociazione basica: $K_b = \frac{[BH^+] * [OH^-]}{[B]}$, valida ad una certa temperatura.\\
La nomenclatura della costante è ingannevole, perchè in realtà (come è evidente dall'equazione chimica) non c'è nulla che si dissocia, ma secondo la teoria di Bronsted-Lowry, la base è specie che accetta protone e lo lega per mezzo di una coppia di elettroni non condivisi.\\
Es. L'ammoniaca è la base debole più nota e dà la seguente reazione in acqua $NH_3 + H_2O \Leftrightarrow NH_4^+ + OH^-$.\\
Si tenga presente che nella maggior parte delle molecole di $NH_3$ è intatta in acqua, essendo $K_b = 1.76 * 10^{-5}$.\\
\subsubsection{Idrolisi}
Sperimentalmente si osserva che sciogliere un sale in acqua può portare ad una variazione di pH. La soluzione risultante può essere acida, neutra o basica. Ciò è dovuto al fatto che i sali sono elettroliti: dissociano in acqua (quasi al 100\%) e può succedere che gli ioni si comportino da acidi o basi di Bronsted-Lowry.\\
Con il termine idrolisi si indica la reazione che avviene tra gli ioni liberati da un sale e l'acqua. Se questi ioni reagiscono con l'acqua, verrà generato in parte l'elettrolita debole da cui derivano ioni, e ciò provoca una variazione di pH.\\
Si distinguono 4 tipi di sali:\\
\tab- Sali derivati da acido forte e base forte: si dissocia completamente ma nessuno dei componenti ha interesse a reagire con l'acqua poichè derivano da acido o base forti. Non si ha idrolisi e il pH resta neutro.\\
\tab- Sali derivati da acido debole e base forte: quando si dissocia completamente, lo ione base forte non da reazione, mentre l'altro ione, essendo base coniugata di un acido debole, reagirà con l'acqua per dare equilibrio. Questo equilibrio viene descritto con uno schema IVE, e le concentrazioni di equilibrio vengono introdotte nella forma algebrica della costante di idrolisi definita come $K_i = \frac{K_w}{K_a}$. Questo processo prende il nome di idrolisi basica, e il pH risulterà sempre basico.\\
\tab- Sali derivati da acido forte e base debole: si dissocia completamente. Lo ione acido forte non da reazioni con l'acqua, mentre l'altro ione, essendo acido coniugato di una base debole, reagirà con l'acqua per dare l'equilirbio. Questo equilibrio viene descritto con lo schema IVE e le concentrazioni di equilibrio vengono introdotte nella forma algebrica della costante di idrolisi, definita da $K_i = \frac{K_w}{K_b}$. Questo processo prende il nome di idrolisi acida e il pH risulta sempre acido.\\
\tab- Sali derivati da acido debole e base debole: si dissocia completamente. Uno sarà lo ione base coniugata di un acido debole, reagirà con l'acqua per dare equilibrio. L'altro ione sarà acido coniugato di una base debole, e reagirà per portare l'equilibrio. In base ai valori di $K_a$ e $K_b$ la soluzione sarà acida, basica o neutra.
\subsection{Teoria di Lewis}
Gilbert Newton Lewis propose un terzo modello per descrivere i comportamenti di acidi e basi, dando importanza alle coppie di elettroni di valenza nel legame covalente. La teoria di Lewis definisce:\
\tab- Acido: Una specie che accetta una coppia di elettroni.\\
\tab- Base: Una specie che dona una coppia di elettroni.\\
Rispetto alla teoria di Bronsted-Lowry, la classe delle basi non viene estesa, mentre quella degli acidi viene notevolmente ampliata.\\
Il prodotto di una reazione acido-base di Lewis viene detto Addotto, cioè una singola specie che contiene un nuovo legame covalente fra l'acido e la base. Il requisito necessario è che la base abbia una coppia di elettroni solitaria da donare e l'acido abbia un orbitale vuoto.\\
\subsubsection{Acidi di Lewis}
Possono essere:\\
\tab- Con atomi poveri di elettroni: hanno un elemento centrale elettrondeficiente. Principalmente B e Al.\\
\tab- Con legami multipli polari: molecole con legame doppio polare.\\
\tab- Cationi metallici: formano i ocmplessi, prodotti di formazione di un legame tra atomo o ione centrale e altri circostanti. (tipico nei sistemi biologici.)
\subsection{Equilibrio dei sistemi tampone}
Si definisce tampone acido-base una soluzione che non subisce variazioni di pH significative a seguito dell'aggiunta di un acido o di una base.\\
Il loro funzionamento è basato sull'effetto dello ione in comune, avviene quando un reagente conenente un dato ione è aggiunto a una soluzione in equilibrio che contiene già quello ione, e la posizione di equilibrio si sposta nella direzione che comporta il suo consumo.\\
Essendo già innumerevoli ioni, la nuova aggiunta non comporta uno squilibrio notevole.\\
Il potere tamponante dipende dalle concentrazioni assolute delle componenti: più queste sono concentrate, maggiore è il potere tamponante.\\ 
Es. Un sistema tampone composto da volumi uguali di sostanza1 1M e sostanza2 1M ha lo stesso pH del tampone in cui le due molarità valgono 0.1M, ma la prima soluzione ha potere tamponante maggiore.\\\\
Il potere tamponante è anche influenzato dal rapporto fra le concentrazioni all'interno del sistema. Un tampone ha il massimo potere tamponante se le concentrazioni delle sue componenti sono uguali.\\
Per una data concentrazione, un tampone ha il massimo potere tamponante quando il suo pH è vicino o uguale al valore di $pK_a$ della sua componente acida. Si definisce campo di tamponamento il campo di pH in cui il sistema tampone agisce efficacemente. Solitamente esso è $\pm 1$ unità di pH dal valore $pK_a$ dell'acido.\
Lo stesso può essere detto per i sistemi tampone in ambiente basico, costituiti da una base debole e dal suo sale formato con un acido forte. \\
E' sempre requisito che i suoi due componenti non reagiscano tra loro.
\subsection{Titolazione acido-base}
Processo con cui una soluzione a concentrazione nota ci aiuta a determinare la concentrazione incognita di un'altra soluzione mediante una reazione monitorata.\\
Es. Titolazione di un acido forte monoprotico:\\
\tab- Si versa in una beuta un volume noto della soluzione dell'acido e si aggiungono alcune gocce di soluzione indicatore (Una sostanza il cui colore è diverso al di sopra o al di sotto di un valore pH caratteristico)\\
\tab- Si aggiunge alla beuta una soluzione a concentrazione nota di base, lasciandola scendere lentamente da una buretta.\\
Quando la titolazione si avvicina alla fine, le molecole di indicatore cambiano colore in prossimità di una goccia di base aggiunta, per effetto di un eccesso temporaneo di ioni ossidrile. Chiameremo questo punto di equivalenza della soluzione; tutte le moli di ioni idronio derivate dall'acido presente nel volume iniziale hanno reagito con un numero equivalente di moli di ioni ossidrile aggiunte dalla buretta.\\
Parliamo invece di punto finale quando si ha un piccolo eccesso di ioni ossidrile che fa cambiare momentaneamente colore. Nei calcoli supponiamo questo eccesso trascurabile, quindi potremmo dire che i due punti si equivalgono.\\
Molarità incognita = $M_{acido} = \frac{M_{base}*V_{base}}{V_{acido}}$
\subsubsection{Indicatore acido-base}
Un indicatore è un acido debole organico che ha un colore diverso da quello della sua base coniugata e per quale la variazione di colore avviene in un campo specifico e ristretto di pH. Tipicamente, una o entrambe le forme sono intensamente colorate.\\
La selezione richiede che si sappia il valore approssimativo del pH del punto finale della titolazione, che a sua volta richiede la conoscenza delle specie ioniche presenti.\\
\subsubsection{Curva di titolazione}
E' il grafico che riporta il pH della soluzione in funzione del volume aggiunto di soluzione titolante.\\
Per gli acidi forti si distinguono tre regioni:\\
\tab- Il pH iniziale è basso, riflettendo l'elevata $[H_3O^+]$ dell'acido forte, e aumenta gradualmente man mano che l'acido viene neutralizzato dalla base aggiunta.\\
\tab- Il pH aumenta bruscamente: l'aumento inizia quando il numero di moli di ioni ossidrile che sono state aggiunte è quasi uguale al numero di moli di ione idronio inizialmente presenti nell'acido. L'ulteriore aggiunta di una o due gocce di base neutralizza il piccolo eccesso di acido e introduce una piccola quantità di base, percio il pH passa da 6 a 8.\
\tab- Dopo questa parte ripida la curva continua a crescere lentamente man mano che viene aggiunta altra base.\\\\
Operativamente si cerca di scegliere un indicatore che garantisca un punto finale vicino al punnto di equivalenza.\\\\
Per la titolazione di acido debole e base forte ci sono delle differenze:\\
\tab- Il Ph iniziale è più alto, l'acido debole si dissocia solo parzialmente rispetto ad una soluzione di acido forte con la stessa concentrazione.\\
\tab- Prima del brusco salto si osserva una porzione di curva crescente in modo graduale, detta regione tampone. Si osserva inoltre che il punto centrale della regione tampone è uguale al $pK_a$. E questo è il modo utilizzato per determinare $K_a$ incognita.\\
\tab- Al punto di equivalenza, $pH >7$ come ci si aspetta per una soluzione con base forte e acido debole.\\
La scelta dell'indicatore è più limitata rispetto al caso dell'acido forte, poichè il salto di pH avviene in un intervallo più ristretto.\\\\
Per la titolazione di un acido poliprotico, è noto che il primo protone è perso più facilmente di quello successivo. Pertanto a causa della grande differenza dei valori di $K_a$, si noteranno due regioni tampone e due punti di equivalenza distinti. E' necessario stesso volume di base aggiunta per rimuovere i protoni in ognuno dei due processi.\\
\subsection{Equilibri di Solubilità}
La maggior parte dei soluti ha solublità limitata in particolare solvente. In una soluzione satura a una particolare temperatura, esiste un equilibrio di solubilità. Questo si può descrivere attraverso una costante $K_c$.\\
Definiamo inoltre il prodotto di solubilità $K_{ps} = \left[A^{b+}\right]^a*\left[B^{a-}\right]^b$ valido ad una determinata temperatura. Questi valori sono tabellari.\\
La presenza di ioni in comune diminuisce la solubilità di un composot ionico poco solubile.\
Nel caso di una soluzione satura si ha $K_{ps} = Q_{ps}$. Se il sistema viene perturbato si presenta una delle seguenti situazioni:\\
\tab- $Q_{ps} > K_{ps}$: Si forma precipitato fino a quando la soluzione ritorna satura.\\
\tab- $Q_{ps} < K_{ps}$: La soluzione è insatura e non si forma precipitato.
\subsubsection{Effetto del pH}
Viene spesso utilizzato per vedere se un minerale contiene carbonati.
