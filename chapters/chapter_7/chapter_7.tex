\section{Teoria quantistica e struttura atomica}
Riportando questa teoria nell'esperimento fotoelettrico abbiamo che un fotone di energia $h * v$ colpisce la superficie di un metallo, che ne assorbe l'energia. Se l'energia del fotone è maggiore di un valore specifico (dipende dal metallo), allora l'elettrone avrà assorbito energia sufficiente a svincolarsi dal metallo e si allontanerà con un proprio valore di energia cinetica.\\
Con queste due teorie si assegnava alle onde energetiche le caratteristiche tipiche della materia: quantità fissa e particelle discrete.\\
In realtà ora sono dimostrate e giuste entrambe le due teorie, particellare e ondulatoria.\\
Si nota poi che se creiamo una differenza di potenziale passante per un gas come ad esempio un compione di idrogeno, esso si illumina.
\subsection{Analisi dell'idrogeno}
Questo non dovrebbe avvenire ma riesce ad avvenire poichè sotto campo elettrico il composto $H_2$ si divide creando del plasma (insieme di elettroni e ioni globalmente neutro).\\
Con $H_2$ si crea quindi del plasma con due ioni $H^+$. Quasi immediatamente gli ioni ed elettroni si legano formando H. Per poi liberarsi dell'energia necessaria e poi rilegarsi fra loro ricreando $H_2$.\\
Se si prendeva la luce creata da $H_2$ e la si passava per il prisma (strumento usato per separare i colori) si notava che non si aveva tutta la banda luminosa ma solamente da un certo numero di componenti, detti anche Righe spettrali. La riga che più brillava era quella rossa (656 nm), seguite da altre onde nel campo uv e ir. L'insieme di queste onde costruisce lo spettro di emissione di $H_2$.\\
Balmer studiò lo spettro dell'idrogeno e scoprì che aveva un andamento regolare.\\
Rydberg riuscì a determinare l'equazione che era in grado di capire la frequenza delle righe spettrali:\\
$v = R\left(\frac{1}{n_1^2} - \frac{1}{n_2^2}\right)$ con $n_1 \epsilon {1, 2, ...}$ e $n_2 > n_1$ con R come costante di Rydberg.\\
Lo spettro H si notano: \\
\tab- Serie di Lyman: l'insieme di righe nella regione dell'uv. $n_1 = 1$\\
\tab- Serie di Paschen: l'insieme di righe nella regione dell'ir.\\
\tab- Serie di Balmer: nel campo visibili. Alcuni colori visibili. $n_1 = 2$\\
\subsubsection{Analisi successive}
Si prova poi a cambiare elemento e si nota che 3 componenti (Hg, Sr, Ne). Si notano 3 spettri diversi.
\subsubsection{Conclusioni}
Si può fare l'effetto contrario. Si può creare uno spettro di assorbimento, si ottiene passando una luce bianca attraverso un vapore di atomi di un elemento, osserveremo uno spettro con alcune componenti mancanti (rappresentate come righe verticali nere). Queste lunghezze mancanti sono state quindi assorbite dall'elemento vaporizzato.\\
Il fatto che lo spettro è prevedibile significa che l'energia all'interno può avere solo valori discreti (ovvero valori precisi) e sono detti livelli energetici. La riga viene osservata nel momento in cui l'elettrone effettua una transizione fra due livelli energetici.\\
La differenza di energia tra due livelli è pari a quella della radiazione elettromagnetica emessa dall'atomo. Il fatto che si osservano righe spettrali distinte suggerisce che nell'atomo un elettrone può avere solo certi valori di energia.\\
Siamo ancora contro i modelli della fisica classica perhè secondo la fisica lelettrone doveva perdere continuamente energia fino a collassare. Essi si aspettavano spettri di emissione continui e non discreti.
\subsection{Modello atomico di Bohr}
Per risolvere questi problemi Bohr crea il suo modello di atomo valido solo per l'idrogeno.\\
I suoi postulati indicano che:\\
\tab- L'atomo di idrogeno ha solo certi livelli energetici permessi, detti stati stazionari. Ciascuno di questi stati è associato ad un'orbita circolare fissa dell'elettrone attorno al nucleo.\\
\tab- L'atomo non irraggia energia mentre è in uno dei suoi stati stazionari. Anche se questo va contro la fisica del tempo. L'elettrone non cade sul nucleo perchè l'elettrone ha una forza centrifuga che lo respinge dal nucleo bilanciata con la forza elettrostatica.\\\\
Una riga spettrale si forma quando si emette un fotone di energia specifica. Questo avviene perchè l'elettrone effettua una transazione da un livello più alto ad uno più basso (se un elettrone scende di livello emana energia pari alla frequenza del colore che noi non vediamo).\\
Il valore dell'energia emessa è data da $E = n\ h\ v$, dove n è un numero intero positivo detto numero quantico principale, associato al raggio dell'orbita dell'elettrone. L'orbita più piccola è la 1 quindi $n = 1$.\\
Atomo allo stato fondamentale (atomo con lo stato energeticamente più basso, con elettrone in livello 1).\\
Questa orbita 1 ha un raggio di 53 pm (picometri) e si chiama Raggio di Bohr.\\
Quando, invece, si trova in un'orbita n> 1 un atomo viene definito eccitato.\\\\
Capiamo quindi che è stato fatto lo spettro di H si sono generate tre serie:\\
infrarossa, visibile, ultravioletta.\\
Quella ultravioletta aveva valori di energia più bassi poichè erano gli elettroni che passavano dalla loro orbita iniziale verso l'orbita numero 1.\\
Successivamente abbiamo la serie visibile, composta dalle energie emesse dagli atomi che passavano dalle loro orbite nell'orbita numero 2.\\
Infine c'è la serie di raggi infrarossi creati dagli elettroni che passavano all'orbita numero 3.\\\\
Le orbite vengono anche chiamate "stati stazionari" di un elettrone.\\
L'orbita 1 si chiama stato fondamentale\\
L'orbita 2 si chiama primo stato eccitato.\\
L'orbita 3 si chiama secondo stato eccitato e così via...\\\\
Bohr non riucì a dimostrare che si potessero prevedere spettri di altri elementi. Ma si scopri poi che funziona per le specie mono elettroniche (es. H, $He^+$, $Li^{++}$) non teneva in considerazione le addizionali attrazioni tra nucleo ed elettroni causate dai legami.\\
Il secondo limite del modello di Bohr era riguardo il moto degli elettroni che veniva considerato su orbite fisse. Questo venne smentito qualche anno più tardi.\\
Si è però ancora oggi conservato lo stato fonamentale, gli stati eccitati e i livelli discreti in cui sono distribuiti gli stati degli elettroni.\\
Altri limiti sono che lui non diede intensità alle righe dello spettro, non c'è stato un criterio per ripartire gli elettroni nelle orbite.\\
\subsection{Modello atomico di Sommerfeld}
Sommerfeld riprese il modello di Bohr studiando però l'elio. Scopri i multipletti, serie vicinissime di righe nello spettro.\\
Secondo lui le orbite erano ellittiche di diversa eccentricità legate dal momento angolare dell’elettrone e indicate con un numero quantico del momento angolare (l).
\subsection{Modello atomico di Zeeman}
Qualche anno dopo Zeeman segui gli stessi esperimenti sotto l’azione di un campo magnetico ed osserva che alcune di quelle righe presenti nel vecchio spettro di emissione si vanno a separere (creando multipletti).\\
Zeeman modifica il modello di Sommerfeld che modificò quello di Bohr.\\
Questo modello rimane ancora con molti limiti, come per esempio che non erano spiegate la natura di certi comportamenti. Quando si verificava un trasferimento? per quanto tempo un elettrone può stare su un orbita?\\
Einstein pubblica intanto la formula $E = m * c^2$ che metteva in relazione l’energia alla massa, insieme alla teoria della relatività.
\subsection{Modello atomico di De Broglie}
Ipotesi de Broglie sul modello di Bohr: se questa energia è di natura particellare, allora la materia è di natura ondulatoria. Adesso doveva verificare attraverso degli esperimenti.\\
Prese una chitarra, con corde fatte di materia. Le corde si muovono in  ondulatorio. Quindi sa che un corpo con materia può avere un modo ondulatorio. \\
Si prova a proiettare il modo ondulatorio intorno ad un orbita. scopriamo che se il numero non è intero si crea una frequenza distruttiva e crea un’onda proibita. quindi sappiamo che l’elettrone avendo massa può effettuare l’orbita ondulatoria solo a certe frequenze, confermando la teoria.\\
Lo stesso esperimento venne poi fatto con un fascio di raggi X su un campione di grafite e si osservò che la lunghezza d’onda della radiazione in uscita era maggiore di quella in entrata. I raggi X hanno perso una quantità di moto che era stata trasferita al campione di grafite. \\
Ciò significa che l’onda si sta comportando come materia dualismo onda particella. ovvero il fatto che entrambe le teorie erano intrinsecamente contenute all’interno della materia e dei raggi di luce.
\subsection{Teoria di Heisenberg}
Dato questo dualismo è impossibile conoscere la posizione precisa dell’elettrone. Questo significa che più conosco la posizione, più è l’errore del momento e viceversa. \\
Heisenberg formulò il principio di indeterminazione.\\
In sostanza definire la traiettoria di un elettrone intorno ad un nucleo non ha senso.\\
Si crea ora la meccanica quantistica. E’ la parte della fisica che studia il moto ondulatorio dei corpi nel mondo microscopico, in particolare su scala atomica. in cui sostituisce il concetto di localizzazione con probabilità di presenza.
\subsection*{Teoria di Schrodinger}
Schrodinger crea la funzione d’onda, una funzione matematica che descrive il moto di un elettrone lungo le tre coordinate rispetto al nucleo.\\
La formula di Dchrodinger si può semplificare come Energia cinetica + energia potenziale = energia totale. \\
Ora non esiste più le orbite di bohr ma esistono gli orbitali atomici definiti da Dchrodinger. 
\subsection{Motello atomico attuale}
Se eleviamo $\psi$ alla seconda individuiamo la densità di probabilità elettronica.\\
Si definisce nodo il punto in cui $psi = 0$, quindi abbiamo lo 0\% di trovare elettroni.\\
La nuvola elettronica è la rappresentazione grafica che ci fa capire una nuvola di posizioni, in cui il colore più intenso, cioè vicino al centro, indica la maggior probabilità di trovare elettroni.\\\\
Un altro modo di pensare gli orbitali sono delle sfere concentriche.\\
In prossimità del nucleo, il volume di ciascuno strato aumenta più rapidamente di quanto diminuisca la sua densità elettronica.\\
questo comporta che la massima probabilità radiale non la abbiamo nel nucleo, ma nel secondo orbitale.\\
Distribuzione di probabilità radiale: $4\pi r^2 * \psi^2$\\\\
Per l’atomo di H, il massimo della distribuzione di probabilità radiale è a 0.529, stesso valore di Bohr, però egli parlava di “totalità del tempo di permanenza dell’elettrone”, qui invece si parla di “maggior parte del tuo tempo”.\\
I grafici di probabilità mostrano che non si raggiunge mai il valore 0, pertanto non è possibile assegnare un volume definito ad un atomo. si tende perciò a visualizzare gli atomi con una superficie di contorno a probabilità costante del 90\%.\\\\
Risolvendo l’equazione di Schrodinger per un atomo 3D, si trova che ogni funzione d’onda è caratterizzata da tre numeri quantici:\\
\tab- Numero quantico principale (n): intero positivo che indica la dimensione relativa dell’orbitale. indica la distanza relativa dal nucleo. Specifica il livello energetico dell’atomo: maggiore è n, maggiore è il liv energetico\\
\tab- Numero quantico del momento angolare (l): num intero compreso tra 0 e n-1. \\
\tab- Numero quantico magnetico(ml): numero compreso da -l a +l passando per 0. Impone l’orientamento dell’orbitale nello spazio intorno al nucleo.\\\\
Termini:\\
\tab- Livello (Guscio): è dato dal valore di n. Minore è n, minore sarà il livello energetico dell’elettrone, maggiore sarà la probabilità che esso si trovi vicino al nucleo.Tutti gli orbitali di un livello hanno la stessa energia, indipendentemente da l. (orbitale degeneri)\\
\tab- Sottolivello (Sottoguscio): i livelli atomici contengono dei sottolivelli in base a l:\\
\tab\tab- l = 0: sottolivello s.\\
\tab\tab- l = 1: sottolivello p.\\
\tab\tab- l = 2: sottolivello d.\\
\tab\tab- l = 3: sottolivello f.\\
\tab- Orbitale: ciascuna combinazione permessa di n, l, m specifica uno degli orbitali atomici
\subsubsection{Orbitale S}
Il primo sottolivello. Ha una forma sferica.\\
In base a n, si hanno n' regioni di densità elettroniche. Più n è grande, più sarà grande, ma meno probabile che ci sia un elettrone.
\subsubsection{Orbitale P}
Ha due regioni di alta probabilità, simmetrico rispetto al nucleo. Non sono sferici, quindi il numero quantico magnetico è importante.\\
Sono sempre in coppia, simmetrici al nucleo.\\
Ne troveremo quindi 3, $P_x$, $P_y$, $P_z$ in base all'asse in cui si trovano. Hanno tutti la stessa probabilità e si chiamano degeneri.
\subsubsection{Orbitale D}
Cinque possibili valori di numero quantico magnetico ($\pm 2, \pm 1, 0$).\\
Ogni orbitale ha 4 zone di probabilità, ortogonali rispetto al centro e li chiamiamo dx, dy, dz.\\
C'è un quarto orbitale ($dx^2-y^2$) che ha nodi nelle direzioni dell'asse x e y (mentre prima erano a 45 gradi).\\
Infine il quinto orbitale $dz^2$ presenta due lobi lungo l'asse z e una regione di densità elettronica a forma di ciambella (toroide) al centro.
\subsubsection{Orbitale F}
Esistono 7 possibili orbitali con forme multilobate molto complesse.
\subsubsection{Aggiunte}
Per descrivere gli orbitali di altri atomi al di fuori dell'idrogeno servono aggiungere alcune caratteristiche:\\
\tab- Un quarto numero quantico.\\
\tab- Un limito di elettroni permessi in unn dato orbitali.\\
\tab- Un più complesso insieme di livelli energetici degli orbitali.