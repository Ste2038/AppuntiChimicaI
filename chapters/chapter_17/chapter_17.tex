\section{Cinetica chimica}
E' lo studio delle velocità di reazione, ovvero la variazione delle concentrazioni dei reagenti in funzione del tempo.
\subsection{Fattori che influenzano la velocità di reazione}
Possiamo trovare:\\
\tab- Concentrazione: dato che le molecole devono collidere per reagire, la velocità di reazione è direttamente proporzionale alla concentrazione dei reagenti.\\
\tab- Stato fisico: dato che devono mescolarsi per potersi urtare, più finemente è diviso un reagente, maggiore sarà l'area della sua superficie riferita all'unità di volume. Di conseguenza sarà maggiore il contatto che esso avrà con l'altro reagente e quindi più veloce.\\
\tab- Temperatura: dato che le molecole devono urtarsi con energia succifiente, l'aumento di temperatura porta un incremendo di urti, con un aumento dell'energia degli urti.\\
\tab- Presenza di un catalizzatore.\\\\
\subsection{Velocità di reazione}
Data una reazione $ A \rightarrow B$, la sua velocità può essere espressa tramite l'equazione $v = -\frac{\Delta [A]}{\Delta t}$ cioè la variazione della molarità del reagente nel tempo.\\
Possiamo definire:\\
\tab- Velocità media di reazione: è calcolata lungo un intervallo di tempo finito. In un grafico che esprime la molarità del reagente in funzione del tempo. La velocità media è data dal coefficiente angolare della retta che congiunge due punti sulla retta.\\
\tab- Velocità istantanea di reazione: è calcolata in un particolare istante della reazione, ed è data dalla pendenza della retta tangente alla curva in un particolare punto.\\
\tab- Velocità iniziale di reazione: è la velocità istantanea nel momento in cui reagenti vengono mescolati. In queste condizioni, i prodotti sono trascurabili e anche la velocità della reazione inversa.\\\\
\subsection{Legge cinetica di reazione (o equazione cinetica)}
Ha forma: $v = k [A]^m [B]^n$ dove compaiono:\\
\tab- (k) Costante di velocità: è specific di una reazione, a una data temperatura e non cambia nella reazione.\\
\tab- (m, n) Ordini di reazione: indicano come la velocità di reazione è influenzata dalla concentrazione dei reagenti.\\
\tab- Eseguire una serie di esperimenti in cui la concentrazione di un reagente viene mantenuta costante mentre l'altra varia. Dopodiche si misura l'effetto di tali variazioni sulla velocità iniziale e si determinano gli ordini di reazione.\\
\tab- Usare i suddetti valori (velocità iniziale e ordini di reazione) per calcolare la costante di velocità.\\\\
\subsection{Ordini di reazione}
Occorre distinguere fra "ordine parziale" e "ordine complessivo".\\
Supponendo una reazione con singolo reagente [A].\\
\tab- Primo ordine: La velocità di reazione è direttamente proporzionale a [A] quindi: $v = k[A]$.\\
\tab- Secondo ordine: la velocità di reazione è proporzionale al quadrato di [A] quindi $v = k[A]^2$\\
\tab- Ordine zero: la velocità non dipende da [A] quindi $v = k$.\\\\
L'ordine parziale è dato dal ordine maggiore fra i reagenti.\\
L'ordine complessivo è dato dalla somma degli ordini dei reagenti.\\
L'unità di misura di k cambia a seconda dell'ordine di reazione.\\
%Fine slide 8