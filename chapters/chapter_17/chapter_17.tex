\section{Cinetica chimica}
E' lo studio delle velocità di reazione, ovvero la variazione delle concentrazioni dei reagenti in funzione del tempo.
\subsection{Fattori che influenzano la velocità di reazione}
Possiamo trovare:\\
\tab- Concentrazione: dato che le molecole devono collidere per reagire, la velocità di reazione è direttamente proporzionale alla concentrazione dei reagenti.\\
\tab- Stato fisico: dato che devono mescolarsi per potersi urtare, più finemente è diviso un reagente, maggiore sarà l'area della sua superficie riferita all'unità di volume. Di conseguenza sarà maggiore il contatto che esso avrà con l'altro reagente e quindi più veloce.\\
\tab- Temperatura: dato che le molecole devono urtarsi con energia succifiente, l'aumento di temperatura porta un incremendo di urti, con un aumento dell'energia degli urti.\\
\tab- Presenza di un catalizzatore.\\\\
\subsection{Velocità di reazione}
Data una reazione $ A \rightarrow B$, la sua velocità può essere espressa tramite l'equazione $v = -\frac{\Delta [A]}{\Delta t}$ cioè la variazione della molarità del reagente nel tempo.\\
Possiamo definire:\\
\tab- Velocità media di reazione: è calcolata lungo un intervallo di tempo finito. In un grafico che esprime la molarità del reagente in funzione del tempo. La velocità media è data dal coefficiente angolare della retta che congiunge due punti sulla retta.\\
\tab- Velocità istantanea di reazione: è calcolata in un particolare istante della reazione, ed è data dalla pendenza della retta tangente alla curva in un particolare punto.\\
\tab- Velocità iniziale di reazione: è la velocità istantanea nel momento in cui reagenti vengono mescolati. In queste condizioni, i prodotti sono trascurabili e anche la velocità della reazione inversa.\\\\
\subsection{Legge cinetica di reazione (o equazione cinetica)}
Ha forma: $v = k [A]^m [B]^n$ dove compaiono:\\
\tab- (k) Costante di velocità: è specific di una reazione, a una data temperatura e non cambia nella reazione.\\
\tab- (m, n) Ordini di reazione: indicano come la velocità di reazione è influenzata dalla concentrazione dei reagenti.\\
\tab- Eseguire una serie di esperimenti in cui la concentrazione di un reagente viene mantenuta costante mentre l'altra varia. Dopodiche si misura l'effetto di tali variazioni sulla velocità iniziale e si determinano gli ordini di reazione.\\
\tab- Usare i suddetti valori (velocità iniziale e ordini di reazione) per calcolare la costante di velocità.\\\\
\subsection{Ordini di reazione}
Occorre distinguere fra "ordine parziale" e "ordine complessivo".\\
Supponendo una reazione con singolo reagente [A].\\
\tab- Primo ordine: La velocità di reazione è direttamente proporzionale a [A] quindi: $v = k[A]$.\\
\tab- Secondo ordine: la velocità di reazione è proporzionale al quadrato di [A] quindi $v = k[A]^2$\\
\tab- Ordine zero: la velocità non dipende da [A] quindi $v = k$.\\\\
L'ordine parziale è dato dal ordine maggiore fra i reagenti.\\
L'ordine complessivo è dato dalla somma degli ordini dei reagenti.\\
L'unità di misura di k cambia a seconda dell'ordine di reazione.\\
%Fine slide 8
\subsubsection{Leggi cinetiche integrate}
\tab- Legge cinetica integrata del primo ordine: $ln\frac{[A]_0}{[A]_t} = k* t$\\
\tab- Legge cinetica integrata del secondo ordine: $\frac{1}{[A]_t}-\frac{1}{[A]_0} = k* t$\\
\tab- Legge cinetica integrata di ordine zero: $[A]_t - [A]_0 = -k*t$\\\\
\subsubsection{Tempo di dimezzamento ($t_\frac{1}{2}$)}
Intervallo di tempo impiegato dalla concentrazione del reagente per raggiungere la metà del suo valore iniziale.    
\tab- Reazione del primo ordine: $t_\frac{1}{2} = \frac{ln2}{k}$. t è costante.\\
\tab- Reazione del secondo ordine: $t_\frac{1}{2} = \frac{1}{k[A]_0}$. t è inversamente proporzionale alla concentrazione iniziale del reagente.\\
\tab- Reazione del ordine zero:$t_\frac{1}{2} = \frac{[A]_0}{2k}$. t è direttamente proporzionale alla concentrazione iniziale del reagente.\\
\subsubsection{Effetto della temperatura}
Troviamo una proporzionalità diretta fra k e tempertaura. \\
Operativamente si trova che il diagramma di k in funzione della temperatura è una curva esponenziale descritta da Arrhenius: $k = A* e^{-\frac{E_a}{RT}}$ in cui:\\
\tab- A: Fattore di frequenza pre-esponenziale\\
\tab- $E_a$: Energia di attivazione: l'energia minima che le molecole deovno avere per reagire.\\\\
Anche l'equazione di Arrhenius può essere linearizzata e si ottiene $lnk = -\frac{E_a}{RT} * \frac{1}{T} + lnA$.
\subsection{Teoria delle collisioni}
L'equazione di Arrhenius fu formulata empiricamente e in seguito vennero proposti due modelli in grado di spiegare gli effetti della concentrazione e della temperatura:\\
\tab- Teoria delle collisioni: considera la velocità di reazione come il risultato di collisioni tra particelle che avvengono con una certa frequenza e un'energia minima.\\
\tab- Teoria dello stato di transizione: offre un modello che evidenzia come l'energia di un urto converta il reagente in prodotto.\\
\subsubsection{Teoria delle collisioni}
Parte dalla considerazione del fatto che le particelle di reagente devono urtarsi per reagire. Perciò, il numero delle collisioni nell'unità di tempo impone un limite superiore alla velocità a cui può avvenire la reazione.\\
Spiega:\\
\tab- Presenza del prodotto delle concentrazioni nella legge cinetica: Se le particelle devono urtarsi per poter reagire, le leggi della probabilità impongono che la velocità di reazione dipenda dal prodotto delle concentrazioni dei reagenti, e non della loro somma.\\
\tab- Effetto della temperatura sulla velocità di reazione: All'aumentare della temperatura cresce la velocità e quindi la frequenza delle collisioni. Nella maggior parte dei casi le collisioni non sono reazioni ma semplici urti. Secondo Arrhenius ogni reazione ha bisogno di un'energia di attivazione ($E_a$), l'energia necessaria per attivare le molecole a trasferirsi ad uno stato in cui i legami possano convertirsi nei legami dei prodotti.\\
%Fine slide 15
\tab- Effetto della struttura molecolare sulla velocità di reazione: 