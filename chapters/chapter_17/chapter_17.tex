\section{Cinetica chimica}
E' lo studio delle velocità di reazione, ovvero la variazione delle concentrazioni dei reagenti in funzione del tempo.
\subsection{Fattori che influenzano la velocità di reazione}
Possiamo trovare:\\
\tab- Concentrazione: dato che le molecole devono collidere per reagire, la velocità di reazione è direttamente proporzionale alla concentrazione dei reagenti.\\
\tab- Stato fisico: dato che devono mescolarsi per potersi urtare, più finemente è diviso un reagente, maggiore sarà l'area della sua superficie riferita all'unità di volume. Di conseguenza sarà maggiore il contatto che esso avrà con l'altro reagente e quindi più veloce.\\
\tab- Temperatura: dato che le molecole devono urtarsi con energia succifiente, l'aumento di temperatura porta un incremendo di urti, con un aumento dell'energia degli urti.\\
\tab- Presenza di un catalizzatore.\\\\
\subsection{Velocità di reazione}
Data una reazione $ A \rightarrow B$, la sua velocità può essere espressa tramite l'equazione $v = -\frac{\Delta [A]}{\Delta t}$ cioè la variazione della molarità del reagente nel tempo.\\
Possiamo definire:\\
\tab- Velocità media di reazione: è calcolata lungo un intervallo di tempo finito. In un grafico che esprime la molarità del reagente in funzione del tempo. La velocità media è data dal coefficiente angolare della retta che congiunge due punti sulla retta.\\
\tab- Velocità istantanea di reazione: è calcolata in un particolare istante della reazione, ed è data dalla pendenza della retta tangente alla curva in un particolare punto.\\
\tab- Velocità iniziale di reazione: è la velocità istantanea nel momento in cui reagenti vengono mescolati. In queste condizioni, i prodotti sono trascurabili e anche la velocità della reazione inversa.\\\\
\subsection{Legge cinetica di reazione (o equazione cinetica)}
Ha forma: $v = k [A]^m [B]^n$ dove compaiono:\\
\tab- (k) Costante di velocità: è specific di una reazione, a una data temperatura e non cambia nella reazione.\\
\tab- (m, n) Ordini di reazione: indicano come la velocità di reazione è influenzata dalla concentrazione dei reagenti.\\
\tab- Eseguire una serie di esperimenti in cui la concentrazione di un reagente viene mantenuta costante mentre l'altra varia. Dopodiche si misura l'effetto di tali variazioni sulla velocità iniziale e si determinano gli ordini di reazione.\\
\tab- Usare i suddetti valori (velocità iniziale e ordini di reazione) per calcolare la costante di velocità.\\\\
\subsection{Ordini di reazione}
Occorre distinguere fra "ordine parziale" e "ordine complessivo".\\
Supponendo una reazione con singolo reagente [A].\\
\tab- Primo ordine: La velocità di reazione è direttamente proporzionale a [A] quindi: $v = k[A]$.\\
\tab- Secondo ordine: la velocità di reazione è proporzionale al quadrato di [A] quindi $v = k[A]^2$\\
\tab- Ordine zero: la velocità non dipende da [A] quindi $v = k$.\\\\
L'ordine parziale è dato dal ordine maggiore fra i reagenti.\\
L'ordine complessivo è dato dalla somma degli ordini dei reagenti.\\
L'unità di misura di k cambia a seconda dell'ordine di reazione.\\
%Fine slide 8
\subsubsection{Leggi cinetiche integrate}
\tab- Legge cinetica integrata del primo ordine: $ln\frac{[A]_0}{[A]_t} = k* t$\\
\tab- Legge cinetica integrata del secondo ordine: $\frac{1}{[A]_t}-\frac{1}{[A]_0} = k* t$\\
\tab- Legge cinetica integrata di ordine zero: $[A]_t - [A]_0 = -k*t$\\\\
\subsubsection{Tempo di dimezzamento ($t_\frac{1}{2}$)}
Intervallo di tempo impiegato dalla concentrazione del reagente per raggiungere la metà del suo valore iniziale.    
\tab- Reazione del primo ordine: $t_\frac{1}{2} = \frac{ln2}{k}$. t è costante.\\
\tab- Reazione del secondo ordine: $t_\frac{1}{2} = \frac{1}{k[A]_0}$. t è inversamente proporzionale alla concentrazione iniziale del reagente.\\
\tab- Reazione del ordine zero:$t_\frac{1}{2} = \frac{[A]_0}{2k}$. t è direttamente proporzionale alla concentrazione iniziale del reagente.\\
\subsubsection{Effetto della temperatura}
Troviamo una proporzionalità diretta fra k e tempertaura. \\
Operativamente si trova che il diagramma di k in funzione della temperatura è una curva esponenziale descritta da Arrhenius: $k = A* e^{-\frac{E_a}{RT}}$ in cui:\\
\tab- A: Fattore di frequenza pre-esponenziale\\
\tab- $E_a$: Energia di attivazione: l'energia minima che le molecole deovno avere per reagire.\\\\
Anche l'equazione di Arrhenius può essere linearizzata e si ottiene $lnk = -\frac{E_a}{RT} * \frac{1}{T} + lnA$.
\subsection{Teoria delle collisioni}
L'equazione di Arrhenius fu formulata empiricamente e in seguito vennero proposti due modelli in grado di spiegare gli effetti della concentrazione e della temperatura:\\
\tab- Teoria delle collisioni: considera la velocità di reazione come il risultato di collisioni tra particelle che avvengono con una certa frequenza e un'energia minima.\\
\tab- Teoria dello stato di transizione: offre un modello che evidenzia come l'energia di un urto converta il reagente in prodotto.\\
\subsubsection{Teoria delle collisioni}
Parte dalla considerazione del fatto che le particelle di reagente devono urtarsi per reagire. Perciò, il numero delle collisioni nell'unità di tempo impone un limite superiore alla velocità a cui può avvenire la reazione.\\
Spiega:\\
\tab- Presenza del prodotto delle concentrazioni nella legge cinetica: Se le particelle devono urtarsi per poter reagire, le leggi della probabilità impongono che la velocità di reazione dipenda dal prodotto delle concentrazioni dei reagenti, e non della loro somma.\\
\tab- Effetto della temperatura sulla velocità di reazione: All'aumentare della temperatura cresce la velocità e quindi la frequenza delle collisioni. Nella maggior parte dei casi le collisioni non sono reazioni ma semplici urti. Secondo Arrhenius ogni reazione ha bisogno di un'energia di attivazione ($E_a$), l'energia necessaria per attivare le molecole a trasferirsi ad uno stato in cui i legami possano convertirsi nei legami dei prodotti.\\
%Fine slide 15
\tab- Effetto della struttura molecolare sulla velocità di reazione: E' necessario che le particelle che si urtano lo facciano con un certo orientamento altrimenti l'urto risulta inefficace. Questo effetto è contenuto nel termine fattore di frequenza pre-esponenziale (A).
\subsection{Teoria dello stato di transizione}
E' lo stato di transizione che si interpone fra lo stato con i reagenti e lo stato con i prodotti. Tutte le molecole sono formate da legami parziali temporanei. Richiede che ci sia un orientamento efficace delle molecole e l'energia dell'urto sia maggiore o uguale a $E_a$.\\\\
\tab- Diagramma di energia per reazione endotermica: I reagenti partono con energia minore dei prodotti, ma con un picco centrale che li supera entrambi.\\
\tab- Diagramma di energia per reazione esotermica: I reagenti partono con energia maggiore dei prodotti, ma con un picco centrale che li supera entrambi.\\\\
Al momento di inizio crescita, in salita verso il picco, iniziano a formarsi dei legami parziali.\\
Nel momento in cui siamo al massimo del picco, i legami parziali sono creati e da ora iniziano a formarsi i prodotti.\\
\subsection{Meccanismi di reazione}
Ogni reazione avviene attraverso un meccanismo, ovvero una sequenza di stati la cui somma da la reazione complessiva. Solitamente il completamento di uno stadio porta a dei prodotti che lo stadio successivo consumerà come reagenti.\\
In laboratorio possono essere isolati per studiare il meccanismo a livello teorico di una reazione chimica.\\
Ogni stadio viene detto elementare se descrive un singolo evento molecolare.\\
Es.\\
Reazione: $2O_3 \rightarrow 3O_2$\\
\tab- Stadio 1: $O_3 \rightarrow O_2 + O$, detto stadio elementare unimolecolare\\
\tab- Stadio 2: $O_3 + O \rightarrow 2O_2$ detto stadio elementare bimolecolare\\\\
Molarità maggiori di due sono molto rare perchè solimtamente si riducono a più stadi elementari\\
Nel caso si eseguano più stadi contemporaneamente, cosa che succede molto spesso, tutti gli stadi si devono adeguare al più lento di loro, ovvero lo stadio detto rate-determing step.
\subsection{Catalisi}
Molto spesso c'è necessità di velocizzare una reazione affinche possa essere sfruttata. A volte può essere fatto modificando la temperatura. Ma spesso andrebbe aumentata di troppo (economicamente)\\
La principale alternativa è l'utilizzo di un catalizzatore, una sostanza che aumenta la velocità di reazione senza essere consumata dalla stessa. Infatti spesso ne sono necessarie quantità piccolissime, non stechiometriche.\\
Occorre sottolineare che:\\
\tab- Un catalizzatore accellera la reazione diretta e la reazione inversa: pertanto la reazione non cambia i valori dei prodotti ma solo la velocità con cui lo si ottiene. Non modifica l'equilibrio.\\
\tab- Un catalizzatore abbassa l'energia di attivazione cornendo un differente meccanismo per la reazione: partecipa alla reazione favorendo stadi elementare di transizioni più economici in termini di $E_a$ ma si rigenera alla fine della reazione.\\\\
Si distinguono:\\
\tab- Catalizzatore omogeneo: esiste in soluzione con miscela di reazione. Deve essere un solido solubile, un liquido o un gas.\\
\tab- Catalizzatore eterogeneo: accellera la reazione che si svolge in una fase separata. Molto spesso  è solido che reagisce con gas o liquido. Quindi risulta importante l'area superfice di contatto.\\
\tab- Enzima: un catalizzatore proteico che accellera le reazioni che avvengono nelle cellule in condizioni blande. Si tratta di proteine globulari e una loro piccola parte costituisce il sito attivo, cioè la regione in cui la reazione va catalizzata.\\
\subsubsection{Processo di Haber-Bosch per la sintesi di $NH_3$}
A causa dei suoi molti usi $NH_3$ è uno dei composti chimici più prodotti al mondo. Più del 80\% viene utilizzato per la produzione di fertilizzanti.\\
Il maggior problema è rappresentato dalla difficoltà nello scindere il legame triplo che tiene uniti i due atomi di N e $N_2$. Questo pone la necessità di considerare due fattori:\\
\tab- Fattore termodinamico: La reazione $N_{2(g)} + 3H_{2(g)} \leftrightarrow 2NH_{3(g)}$ ha $\Delta H_r^0 = -91.8 kJ\ mol^{-1}$ pertanto occorre operare a basse temperature e alta pressione.\\
\tab- Fattore cinetico: La reazione è molto lenta, per aumentare la sua velcoità occorre aumentare la temperatura, tuttavia va contro il punto precedente e per il principio di Le Chatelier, un auemnto di temperatura potrebbe spostare l'equilibrio verso la formazione dei reagenti.\\\\
Per tali motivi il processo di Haber-Bosh opera secondo i criteri:\\
\tab- Temperatura: circa 400°C\\
\tab- Pressione: elevata circa 230 atm\\
\tab- Concentrazione: per aumentare la resa di reazione, si aumenta la concentrazione dei reagenti($N_2$ in particolare) e diminuendo la concentrazione dei prodotti.\\
\tab- Catalizzatore: eterogeneo a base di Fe.
\subsubsection{Marmitta catalitica}
Data la "sporca" reazione prodotta all'interno del motore di un auto, essa produce idrocarburi e NO da dover abbattere.\\
La prima soluzione utilizzata era una marmitta ossidante in grado di risolvere il problema degli idrocarburi ossidandoli, ma non di abbattere NO.\\
Le moderne marmitte catalitiche sono dette trivalenti perchè riescono ad eliminare tutti e tre gli inquinanti dai fumi di scarico. Sono realizzate unendo nella struttura due catalizzatori, uno riducente e l'altro ossidante.\\
Il riducente a base di Rh favorisce la decomposizione di NO in $N_2$ e $O_2$. A valle si pon quello ossidante che utilizza Ossigeno ancora presente nei gas per completare la combustone dei composti non completatamente ossidati.
Per un efficenza massima il rapporto ideale fra aria e benzina è di 14/1, ovvero il rapporto stechiometrico. A fine di controllare al meglio questo sono state introdotte le centraline elettroniche insieme agli iniettori elettronici.
\subsection{Radioattività e processi nucleari}
La composizione del nucleo atomico rimane solitamente costante nel tempo, in alcuni casi pul varaire spontanemanete o a seguito di bombardamento a opera di particelle elementari o ioni. Questo porta ad una variazione del numero atomico e quindi delle proprietà chimiche.\\
Con il termine nuclide si indica una singola specie nucleare, caratterizzata da un numero atomico Z, un numero di massa A e un particolare stato energentico.\\
Attualmente ci sono circa 2000 nuclidi noti(naturali e artificiali) solo 270 sono stabili, mentre i restanti sono radioattivi.\\
Tale instabilità induce spontanea trasformazione in altri isotopi che si accompagna con l'emissione di particelle e talvolta di radiazione elettromagnetica.\\
Il decadimento radioattivo è la trsformazione di un atomo radioattivo che, decadendo in un altro atomo, può divenire stabile.\\
\subsubsection{Fenomeni di decadimento radioattivo}
Globalmente elenchiamo i seguenti fenomeni:\\
\tab- Decadimento $\alpha$: una reazione caratteristica dei nuclidi con Z compreso fra 82 e 200 e costituita da He. Le particelle $\alpha$ a causa della loro massa elevata hanno scarso potere di penetrazione.\\
\tab- Decadimento $\beta^-$: una radiazione costituida da elettroni espulsi dal lucleo e che si suppone formatisi in esso in seguito alla trasformazione di un neutrone in un protone. Il loro potere di penetrazione è molto elevato.\\
\tab- Decadimento $\beta^+$: è una radiazione costituida da particelle di massa pari a quella degli elettroni, di pari carica elettrica in valore assoluto, ma di segno positivo. Si tratto di positroni e si originano dal nucleo per la trasformazione di un protone in un neutrone.\\
\tab- Cattura K: il rapporto tra neutroni e protoni può essere variato anche se uno delgi elettroni che ruotano più vicini al nucleo viene catturato da esso e, interagento con un protone, dà origine ad un neutrone. Questo decadimento è accompagnato da emissione di radioazione elettromagnetica ad alta frequenza (raggi X) causato dal riarrangiamento delgi elettroni atomici che segue la cattura K.\\
\tab- Radiazione $\gamma$: è una radiazione elettromagnetica ad altissima frequenza, che accompagna in alcuni casi l'emissione di particelle $\alpha o \beta$ da parte di rdionuclidi. L'origine dei raggi $\gamma$ è nucleare e si spiega come dovuta al riassestamento del nucleo che ha emesso una particella e che dopo tale emissione, si trova in uno stato eccitato, passa ad uno stato meno eccitato emettendo raggi $\gamma$, caratterizzati da un elevato potere penetrante.\\\\
Particelle alfa vengono fermate da un foglio di carta.\\
Particelle beta passano un foglio di carta ma vengono fermate da un foglio d'alluminio.\\
Particelle gamma passano carta e alluminio, vengono fermate solamente da uno spesso strato di piombo.\\\\
Il decadimento di un radionuclide è un fenomeno statistico: si considera un certo numero di nuclei tutti uguali di un determinato radionuclide, non è possibile stabilire con precisione quando un particolare nucleo si disintegrerà ma solo parlare di probabilità che ciò avvenga in un determinato intervallo di tmepo.\\
\subsection{Difetto di massa}
E' identificato come la differenza del reale peso di un nucleo con quello teorico. Il teorico peserà sempre di più per colpa di questo difetto dovuto dall'energia di legame tra nucleoni, ovvero l'energia necessaria per separare i nucleoni, quella che si libererebbe se si formasse un nucleo aggregando i nucleoni inizialmente isolati.\\
E' possibile calolare l'energia media di legame per nucleone come $\frac{\Delta E}{A}$.\\
Questo è alla base dei processi di:\\
\tab- Fissione nucleare: scissione spontanea (o causata con collisione) di un nucleo a elevato A in due o più frammenti a basso peso con emissione di un certo numero di neutroni. Accompagnato da un forte rilascio di energia dovuto dalla differenza tra l'energia di legame dei nucleoni nel nucleo che si scinde a quella nei nuclei che si originano\\
\tab- Fusione nucleare: è la reazione tra nuclei di elementi leggeri che vengono uniti per dar luogo a nuclei più pesanti. Affinchè ciò avvenga i nuclei di partenza devono urtarsi con un'energia cinetica elevata per vincere la forte repulsione elettrostatica che is manifersta quando essi vengono in contatto. Può accadere solo a temperature elevatissime per tempi lunghi.\\\\
I nuclidi radioattivi trovano ampie applicazioni in:\\
\tab- Campo medico: Utilizzati come sorgenti i radiazione per la cura di malattia o traccianti nella diagnostica.\\
\tab- Datazione radiometrica: Usato per determinare l'età di reperti organici