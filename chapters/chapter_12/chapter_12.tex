\section{Forze intermolecolari, liquidi, solidi e transizioni di fase}
\paragraph*{Fase}:\\
E' una porzione di sistema termodinamico che presenta stato fisico e composizione chimica uniforme, mentre altre grandezze(ad esempio temperatura e pressione) possono essere non uniformi.\
\paragraph*{Sistema omogeneo}: sistema con 1 sola fase.\\
La fase non corrisponde allo stato di aggregazione, dunque quando si parla di fase solida, liquida o fase gassosa si sta specificando lo stato di aggregazione che caratterizza una particolare fase del sistema, ma all’interno dello stesso sistema possono essere presenti ad esempio più fasi liquide.\\
Esempio: acqua e olio, hanno lo stesso stato di aggregazione (liquido) ma due fasi differenti. \\
Quando si parla di fasi condensate, si intende lo stato solido e liquido, no gas.\\
L’energia potenziale tra particelle (atomi, molecole o ioni) in un campione di materia è dovuta a forze attrattive e repulsive, che corrispondono alle forze intermolecolari. L’interazione tra quelle forze e l’energia cinetica delle particelle dà origine alle proprietà di ciascuno stato di aggregazione e alle transizioni di fase.\\\\
Stati: \\
\tab- Gas: il contributo di energia pot è basso rispetto all’energia cinetica delle particelle. E questo di nota dal moto casuale, comprimibili e con il fenomeno di diffusione prima studiato\\
\tab- Liquidi: maggiore contributo di energia potenziale ma il contributo cinetico è ancora prevalente, quindi le particelle sono libere di muoversi in modo casuale (assume forma del recipiente se c’è forza di gravità)\\
\tab- Solido:\\\\
Passaggi di stato:\\
\tab- Consensazione (o liquefazione) gas → liquido\\
\tab- Evaporazione (o vaporizzazione) liquido → gas\\
\tab- Solidificazione(o congelamento) liquido → solido\\
\tab- Fusione solido → liquido\\
\tab- Sublimazione solido → gas\\
\tab- Brinamento gas → solido\\\\
I passaggi di stato hanno variazione di entalpia. si indica con DeltaH con 0 ad apice se indica una condizione standard. \\
Distinguiamo due punti fondamentale nel cambio di stati:\\
\tab- In una fase: c’è una variazione di temperatura che è associata alla variazione di energia cinetica media.\\
\tab- Durante una transizione di fase: il processo viene definito isotermico, detto quindi calore latente perchè non cambia\\\\
Analizzando un grafico di temperature notiamo che per un passaggio di stato da 1 a 2, abbiamo sempre una fase di mezzo, nella prima e nell’ultima fase abbiamo una temperatura che aumenta o diminuisce costantemente, mentre la fase di mezzo è isotermica perchè il corpo sta assorbendo quella energia.
\subsection{Equilibrio dinamico}
\subsubsection{Gas-Liquido}
Quando avendo due stati, ad esempio uno liquido e uno gassoso, troviamo lo stesso numero di particelle che condensano ed evaporano.\
Questo sistema è apparentemente fermo perchè le due trasformazioni avvengono contemporaneamente.\\\\
La pressione che si misura in questo equilibrio si chiama pressione di vapore o tensione vapore.\\
Le dimensioni non contano\\
Osserviamo invece che se la temperatura aumenta avremo un maggiore numero di particelle in forma di gas, quindi aumenta la pressione, e viceversa.\\
\paragraph*{Equazione di Clausius-Clapeyron}:\\
$\ln\frac{p_2}{p_1}=\frac{-\Delta H_{vap}}{R}\left(\frac{1}{T_2}-\frac{1}{T_1}\right)$\\
$R = 8.31\ frac{J}{mol\ K}$\\
Eboliizione vs evaporazione\\
Evaporazione: quando il liquido in superficie passa allo stato gassoso.\\
Ebollizione: quando il bulk (ovvero il corpo della sostanza, tutta tranne la superficie) e la superficie passano allo stato gassoso. (Bolle di aria che salgono da sotto l'acqua)\
In montagna l'acqua bolle prima perchè c'è più pressione atm più bassa.
\subsubsection{Liquido-Solido}
Se diamo energia alle particelle di un solido esse vibrano fino a muoversi completamente, se si muovono significa cheil solido è divenuto un liquido.\
Se togliamo energia a queste particelle esse si fermano e torno un solido.\\
Qui vediamo la temperatura di fusione rimane costante nelle due fasi, quella di solidificazione e quella di fusione.
\subsubsection{Solido-Gas}
I composti in grado di sublimare grazie alla loroalta pressione di vapore come ad esempio $I_2$, $CO_2$, canfora.\\
Questi elementi sono caratterizzati da attrazione intermolecolari molto deboli, ed inoltre la pressione atmosferica non deve essere troppo alto.\\
\subsection{Diagramma di fase}
Diagramma che indica gli stati di aggregazione di un composto in base alla temperatura (asse ascisse) e pressione (ordinate)\\
Nel diagramma compariranno 3 regioni, una per il solido, una per il liquido e una per il gas.\\
Ogni punto lungo queste rette corrisponde a un punto in cui coesistono due stati di aggregazione e la materia sta cambiando il suo stato.\\
Notiamo la pendenza della retta fra solido e liquido, essa ha coeff. angolare negativo quindi il liquido è più denso del solido.\\
Se il coeff angolare fosse stato positivo il solido era più denso del liquido. \\
Esiste poi un punto, chiamato punto triplo in cui la materia presenta tutti e tre gli stati di aggregazione. \\
Esiste infine un punto critico in cui la linea finisce e da li in poi liquido e vapore non sono più distinguibili e quello che si genera il fluido supercritico, avviene a temperatura critica e pressione critica.  \\
OSS. nel diagramma della $CO_2$ notiamo che la $CO_2$ non può esistere in forma liquida a pressione atm.\\
Nel grafico dell $H_2O$ troviamo una retta solido liquida ha coeff angolare negativo perchè il ghiaccio ha densità minore dell’acqua, quindi la natura preferisce lo stato che occupa meno spazio.\\\\
Punto triplo $H_2O$: 0,01 C TEMP e 0,006 ATM\\
Punto critico $H_2O$: 374 C TEMP e 218 ATM
\subsection{Distanze e lunghezze}
Abbiamo visto precedentemente la lunghezza di legame(nel legame covalente) che equivale alla distanza fra due nuclei di Cl legati covalentemente fra loro. Da questo potevamo ricavare il raggio di legame che equivale alla metà della lunghezza di legame.\\
Aggiungiamo invece la distanza di Van der Waals che è la distanza fra due atomi di Cl non legati fra loro, ma situati adiacenti. Da questa distanza si ricava il raggio di Van der Waals che equivale alla metà della distanza di Van der Waals.
\subsection{Forze di Van Der Waals}
\subsubsection{Forza Ione-Dipolo}
E' quella forza che permette l'unione di uno ione (catione o anione) con una molecola polare (chiamato dipolo). Questo legame è alla base della solubilizzazione in acqua degli ioni.
\subsubsection{Forza Dipolo-Dipolo}
E' un legame che unisce insieme due ioni uguali insieme ponendo le cariche opposte dello ione vicine fra loro.\\
Più forti sono queste interazione più sarà difficile fargli cambiare stato di aggregazione.
\subsubsection{Legame a idrogeno}
Def: si origina tra molecole che hanno un atomo di H legame ad un atomo piccolo, altamente elettronegativo e con coppie di elettroni solitarie.\\\\
Questi atomi con cui H si può legare sono F, O, N.\\
Le sostanze che formano legame ad idrogeno passano molte facilmente allo stato gassoso.
\subsubsection{Forze carica-dipolo indotto}
Ne esistono di due tipi: ione-dipolo indotto e dipolo-dipolo indotto\\
Per avvenire servono 2 sostanze, una ionica o polare, l’altra apolare.\\ 
Quello che succede è che la sostanza carica, cioè quella ionica o polare va ad indurre (ecco perchè si chiama indotto) l’altro sostanza ad un polarità in modo da permettere un legame. \\
La facilità con cui la sostanza che era apolare è stata modificata con una carica parziale delta meno e quindi attratta, si chiama polarizzabilità. (è più bassa nelle molecole piccole, più è grande, più è facile da fregare).
\subsubsection{Forze di dispersione (o di london)(o di dipolo\_istantaneo-dipolo\_indotto)}
E’ quello che succede per $Cl_2$, questa molecola è perfettamente bilanciata e nessun elettrone rimane solo e da dover sistemare.\\
In realtà in alcuni istanti la nostra carica elettronica non è uniformemente distribuita e in questi momenti succede che si crea un dipolo istantaneo su un atomo. questo andrà a indurre tutte le molecole circostanti in modo analogo.\\
Questo legame è più facile da formare in molecole con un numero più elevato di elettroni. successivamente un altro fattore importante è la geometria molecolare. poichè più superfici di contatto ho a disposizione, più sarà facile andare a creare un legame temporaneo.
\subsection{Proprietà dello stato liquido}
\subsubsection{Tensione superficiale}
In un campione liquido, es acqua, ci sono le molecole interne che hanno attrazioni con le altre molecole in tutte le direzioni, mentre quelle in superficie hanno attrazioni solo con quelle sottostanti, notiamo quindi che esse hanno un vettore risultante che punta verso il basso. L’acqua cerca sempre di occupare il minimo spazio possibile (se vediamo una goccia cadere essa sarà sferica.\\
Per aumentare l’area della superficie le molecole dovrebbero muoversi verso l’alto.\\
La tensione superficiale è l’energia richiesta per aumentare di una quantità unitaria l’area della superficie.\\ 
Per farlo si può ricorrere a dei tensioattivi (saponi, detergenti) che si aggregano alla superficie e rompono i legami ad idrogeno. si utilizzano per aumentare la bagnabilità e la miscibilità tra due liquidi.
\subsubsection{Capillarità}
La capacità di salire o scendere in uno spazio stretto (capillare) è chiamata capillarità. \\
E’ dovuta dalla competizione di forze coesive all’interno del liquido e quelle fra liquido e pareti del tubo, forze adesive. \\
\paragraph*{Acqua}:\\
Se mettiamo acqua in un capillare esso sale volentieri perchè si forma un legame ad idrogeno fra la parete in vetro e le molecole d’acqua. Qui le forze adesive prevalgono quindi vediamo una curva con concavità verso l’alto (menisco concavo). \\
\paragraph*{Mercurio}:\\
Se invece osserviamo il mercurio osserviamo che scende sotto il livello del mercurio esterno per avere meno interazioni possibili con il vetro. Osserviamo quindi una curva con concavità verso il basso poiché il liquido tenta di salire ma senza toccare il vetro (menisco convesso).
\subsubsection{Viscosità}
Quando un liquido fluisce le sue molecole scorrono l’una accanto all’altra. \\
La viscosità di un liquido è la sua resistenza allo scorrimento ed è dovuta dalle attrazione intermolecolari. \\\\
La viscosità diminuisce all’aumentare la temperatura (olio in padella)\\
La viscosità è anche influenzata dalla geometria molecolare, molecole lunghe fanno più contatto e quindi hanno viscosità maggiori.
\subsubsection{Acqua}
originata da atomi di H e O\\
due coppie di legame e due di non legame. \\
Forte differenza di elettronegatività tra H e O\\
geometria angolare\\
elevata polarità di legame\\
Grande potere solvente\\
Punto triplo $H_2O$: 0,01 C TEMP e 0,006 ATM\\
Punto critico $H_2O$: 374 C TEMP e 218 ATM\\
calore specifico eccezionalmente alto.\\ 
entalpia di vaporizzazione eccezionalmente alta. \\
elevata tensione superficiale\\
elevata capillarità\\\\
La formazione di 4 legami a idrogeno attorno a ciascuna molecola d’acqua conferisce al ghiaccio una struttura aperta esagonale, ecco da dove deriva la forma dei fiocchi di neve. \\
L’acqua cambia anche forma geometrica nel passaggio alla fase solida passando dai 104,5 gradi che aveva da molecola singola, ai classici 109.5 che formano un tetraedro.\\
\subsection{Solidi}
Classificazione sul grado d’ordine delle loro forme:\\
\tab- Solido cristallino: forma ben definita perchè le loro particelle sono disposte in una struttura ordinata e periodica nelle 3 dimensioni\\
\tab- Solidi amorfi: hanno forme mal definite perchè sono privi di un vasto ordine delle loro particelle.\\\\
Solido cristallino (o cristallo) è una struttura solida costituita da particelle aventi una disposizione geometrica regolare, che si ripete nelle tre dimensioni spaziali, detta reticolo cristallino. Si definisce cella elementare la più piccola parte del reticolo che, ripetuta nello spazio, forma il cristallo. Il numero di particelle prime vicine che circondano una particella è detto numero di coordinazione(N.C). \\
Un sistema cristallino è il raggruppamento di più strutture cristalline che presentano una cella elementare della stessa forma (cubica, tetragonale, …) esistono 7 sistemi cristallini (ognuno dei quali raggruppa più di un reticolo) per un totale di 14 reticoli differenti. \\
La maggior parte dei composti covalenti e ionici si presentano come reticoli cubici.\\\\
I 3 principali sono:\\
\tab- Cella cubica semplice: 8 particelle sui vertici condivise con i cubi circostanti. Ogni particella ha contatto con le 3 adiacenti nella stessa struttura. Se ampliamo la struttura vediamo che il numero di coordinazione di ogni particella è 6, 3 della stessa particella, e 3 delle particelle adiacenti.\\
\tab- Cella cubica a corpo centrato: 8 vertici più 1 centrale. per i vertici come prima ma tutti i vertici toccano la particella centrale. Se ampliamo la struttura vediamo che il numero di coordinazione  di ogni particella è 8, poiché le più vicine particelle adiacenti sono le 8 centrali delle strutture circostanti\\
\tab- Cella cubica a facce centrate: 8 vertici più 6, una al centro di ogni faccia. Il numero di coordinazione  diventerà quindi 12. \\\\
Per calcolare il numero di atomi in una cella basta sommare con alcuni accorgimenti il numero di particelle.\\
Ogni particella al vertice conta come $\frac{1}{8}$. In totale abbiamo sempre 1 poichè i vertici sono 8.\\
La particella al centro vale 1 poiché è interamente all’interno della figura.\\
Le particelle al centro della faccia sono all’interno solo per metà quindi varranno $\frac{1}{2}$ ciascuna. essendo 6 otteniamo 3.\\\\
Quindi facendi i calcoli:\\
\tab- Cella cubica semplice: 1\\
\tab- Cella cubica a corpo centrato: 2\\
\tab- Cella cubica a facce centrate: 4\\\\
Questo numero è molto importante perché ci fa capire che nella cella cubica a facce centrate, abbiamo più volume occupato rispetto a quello a cella cubica semplice. \\
Se proviamo ad eseguire un modello di impaccamento degli atomi (cioè creare una struttura) e analizziamo lo spazio osserviamo che solo il 52\% del volume è occupato. Questo metodo è inefficiente.\\ 
La cella cubica a corpo centrato ha un'efficienza del 68\%, è una struttura tipica dei metalli come ad esempio Fe, Cr, 1 gruppo, e altri.\\
La cella esagonale ha una base con forma esagonale appunto e spazi fra particelle molto più piccoli. \\
Ne distinguiamo due tipo:\\
\tab- esagonale compatto (HCP Hexagonal closest packing): in cui abbiamo due strati distinti posti in modo ABAB. In questa configurazione troviamo due tipi di vuoti, quello tetraedrico e quello ottaedrico.\\
\tab- Cubico compatto (CCP, cubic closest packing): in cui abbiamo tre strati, disposti come ABCABC Questo metodo corrisponde all struttura cubica a facce centrate.\\\\
Entrambi questi due metodi hanno efficienza di impaccamento pari a 74\%, con N.C. pari a 12. E’ impossibile impaccare celle con lo stesso raggio più efficientemente di questo metodo. \\
Per studiare queste strutture nella realtà utilizziamo un metodo chiamato analisi per diffrazione di raggi X (XRD= X-Ray diffraction)
\subsection{Equazione di Bragg}
Equazione: $n\lambda = 2\ d\ sen\theta$\\
Dove $\theta$ è l’angolo di inclinazione delle onde\\
d è la distanza fra gli stati del cristallo \\
$\lambda$ è la lunghezza d’onda.\\
Classificazione in base a forze intramolecolari tra particelle:\\
\tab- solidi atomici: atomi uniti da forze di dispersione di london: i gas nobili. Le loro proprietà sono con legami molto deboli. Abbiamo quindi temperature di evaporazione e fusione molto basse. L’impaccamento è cubico compatto.\\
\tab- Solidi molecolari: solidi in cui ogni punto di impaccamento è una molecola. fra queste molecole avremo legami dipolo istantaneo -dipolo indotto, oppure dipolo-dipolo, oppure legami ad idrogeno. Per questo motivo le caratteristiche fisiche variano in base al legame che andiamo a mettere\\
\tab- Solidi ionici: contiene particelle con cariche intere e di conseguenza le interazioni sono molto forti. In questi composti facendo impaccamento troveremo lo ione più grande che si impacca normalmente(spesso cubico compatto, NaCl) e lo ione più piccolo occupa gli spazi vuoti. Ciascun cation sarà circondato da 6 anioni e viceversa. I legami di questi solidi sono molto forti e da questo ne conseguono ottime proprietà meccaniche, alta temperatura di fusione, durezza.\\
\tab\tab- Ecc. importante, Cesio in CsCl ha raggio atomico grande quindi ha bisogno di reticolo cubico semplice.\\
\tab- Solidi metallici: sono tenuti insieme dal legame metallico. La maggior parte di questi solidi impacca in modo compatto. Le proprietà sono un elevata conducibilità elettrico termica, lucentezza, duttilità e malleabilità. \\
\tab- Solidi covalenti reticolari: es. Quarzo. Sono presenti forti legami covalenti tra gli atomi. Queste sostanze hanno temperature di fusione ed ebollizione estremamente alte , ma la loro conducibilità elettrica e termica e durezza dipende dal materiale. \\
\tab\tab- Allotropi. Sono quei composti che presentano un solo elemento ma con numeri differenti. (attenzione devono essere nello stesso stato di aggregazione) es. $O_2$, $O_3$. Due allotropi famosi del carbonio sono la grafite, il Diamante, Fullereni, Grafene e nanotubi di carbonio\\
\tab\tab- Polimorfismo: sono invece i composti  con la stessa formula ma impaccamento e aspetto fisico differente. \\
\tab\tab- Isomorfismo: sono isomorfi due composti che cristallizzano nello stesso modo
\subparagraph{Allotropi del carbonio}
\tab- Grafite:\\
\tab\tab- i diversi atomi di carbonio sono legati fra loro mediante orbitali ibridati sp2, sono planari e a 120 gradi l’uno dall’altro. L’ultimo orbitale p non coinvolto si pone perpendicolarmente al piano dei agli legami così da formare una struttura esagonale. Fra un piano e l’altro della nostra struttura non avremo legami chimici ma solamente forze di van der waals\\
\tab\tab- fonde a 3500 gradi C\\
\tab\tab- Conduce bene lungo un piano ma non fra due piani. \\
\tab- Diamante:\\
\tab\tab- cristallo trasparente composto da atomi di C a struttura tetraedrica. \\
\tab\tab- impacca in una cella cubica a facce centrate\\
\tab\tab- termodinamicamente è una forma instabile del C.\\
\tab\tab- cineticamente stabile\\
\tab\tab- estrema durezza\\
\tab\tab- elevato indice di dispersione ottica\\
\tab\tab- altissima conducibilità termica\\
\tab\tab- grande resistenza ad agenti chimica\\
\tab\tab- bassissimo coefficiente di dilatazione termica.\\
\tab\tab- idrorepellente\\
\tab\tab- sono isolanti elettrici. \\
\tab- Fullereni\\
\tab\tab- una sfera cava di atomi di C formata da una serie di esagoni e pentagoni collegati fra loro.\\
\tab\tab- Il più importante è il C60, chiamato buckminsterfullerene costituito da 20 esagoni e 12 pentagoni con 60 vertici (atomi) e 90 bordi (legami)\\
\tab\tab- resistenza termica\\
\tab\tab- superconduttore\\
\tab\tab- utilizzato in elettronica\\
\tab- Grafene\\
\tab\tab- è costituito da un unico piano di grafite, tutti esagoni collegati fra loro. \\
\tab\tab- resistenze meccaniche del diamante\\
\tab\tab- flessibile come la plastica \\
\tab\tab- miglior conduttore termico noto\\
\tab- Nanotubi di carbonio\\
\tab\tab- come i fogli di grafene ma posti sotto forma di pillola, le due estremità sono formate da esagoni e pentagoni che ne permettono la chiusura.\\
\tab\tab- Utilizzato in campo medico per veicolare medicinali e studiare il cancro, ma anche in campo elettromeccanico e della sensoristica
\subsection{Solidi amorfi}
Classico esempio il vetro, il quale aveva una struttura di impaccamento ma non gli è stato lasciato il tempo di raggiungerla durante il passaggio dalla fase solida alla fase liquida. Il vetro si forma infatti dal diossido di silicio liquido che viene raffreddato molto velocemente per permettergli di mantenere quella forma.
\subsection{Teoria delle bande}
Si nota che più atomi dello stesso tipo di combinano, più i loro orbitali sono energeticamente vicini fre loro. \\
Ad un certo punto questi orbitali non sono più distinguibili fra loro e prenderanno il nome di bande. Ne troveremo quindi due la banda di valenza e la banda di conduzione. \\
queste due bande si definiscono contigue fra loro e questo comporta che gli elettroni possono saltare dalla banda di valenza piena alla banda di conduzione vuota con un’energia infinitesima. In altre parole gli elettroni sono liberi di muoversi in tutto il campione di metallo. Questo comporta la conduttività di un metallo. \\
\tab- conduttori: le due bande non hanno nessun gap fra di loro quindi sono ottimi conduttori. Ma se la temperatura aumenta il nostro elemento perderà di conducibilità.\\
\tab- semiconduttori: fra le due bande è presente di un band gap, quindi serve un po di energia per permettere agli elettroni di passare. Andiamo quindi andiamo ad esempio a riscaldare un po il nostro conduttore. \\
\tab- isolanti: le due bande hanno un notevole gap fra loro e quindi non c’è nessun modo di permettergli di condurre energia.\\\\
Esistono poi i superconduttori, sono quei conduttori che riescono a condurre senza però rilasciare energia, per esempio sotto forma di calore. Esistono ad oggi alcuni materiali in grado di farlo ma con la peculiarità di doversi trovare a temperature molto molto basse
\subsection{Materiale}
Si definisce materia un oggetto destinato ad un particolare uso con un riferimento alla natura chimico-fisica di un corpo associata al suo utilizzo. Può essere costituito da una o più sostanze e viene definito materiale composito.  Ogni materiale ha una sua specifica morfologia, ovvero il modo in cui è organizzata la materia.\\
I materiale sintetici sono quelli realizzati dall’uomo
\subsection{Leghe}
Materiale formato da due o più metalli. di cui almeno uno deve essere un metallo della tavola periodica. \\
La legha ha proprietà metalliche diverse da quelli originali. \\
Leghe importanti:\\
\tab- Acciaio (Fe-C) prodotto per la sua durezza meccanica\\
\tab- Bronzo (Cu- Sn): prodotto per incrementare le caratteristiche del rame\\
\tab- Ottone (Cu-Zn): prodotto perchè più duro del rame e più lucente dello Zn\\\\
Da notare che nelle leghe non abbiamo un solo punto di fusione, data la presenza di due o più elementi avremo un range di fusione in cui coesiste la fase liquida e quella solida
\subsection{Leghe eutettiche}
Sono quelle leghe che grazie all’avanzamento tecnologico riescono a fondere ad un solo punto di fusione, sempre più basso dei due elementi da cui derivano\\
Es. Sn-Pb (63\%) e (37\%)\\\\
La lega è formata da un solvente e un soluto, relativamente quello presente maggiormente e minormente.\\
Il solvente va a dettare la struttura geometriche che assumerà il nostro prodotto finito mentre il soluto va ad occupare gli spazi lasciati dal solvente. \\\\
Abbiamo quindi:\\
\tab- Soluzione solida disordinata: gli atomi sono disposti secondo una precisa geometria ma posti casualmente fra loro\\
\tab- Soluzione solida ordinata: in questo caso sia la geometria che i legami sono ordinati. questo è dovuto dal fatto che in questo caso sono preferiti i legami fra atomi differenti rispetto a quelli fra lo stesso elemento\\
\tab- Composto intermetallico: sono atomi diversi che hanno una importante differenza di elettronegatività. La stechiometria non viene rispettata. 
\subsection{Difetti cristallini}
I difetti sono quelle imperfezioni che andiamo a riscontrare quando i cristalli si formano troppo velocemente.\\
Vengono catalogati come:\\
\tab- difetti di punto\\
\tab\tab- vacanza: mancanza di atomi dove doveva esserci\\
\tab\tab- atomo sostituzionale: atomo che non si doveva trovare nella lega che prende però posto di un altro\\
\tab\tab- atomo interstiziale: un atomo che si colloca in una posizione che doveva essere vuota\\
\tab- difetti di linea\\
\tab- difetti di superficie\\
\tab- difetti di volume\\\\
I difetti possono essere volontari per andare a modificare le proprietà\\
Spesso si va ad effettuare il drogaggio per modificare le bande di valenza e di conduzione per migliorare la conducibilità\\
\subsection{Materiali Ceramici}
Composti inorganici solidi prodotti da cottura\\
Sono classificati in:\\
\tab- ceramici tradizionali: Argilla, silice(SiO2), Quarzo e feldspati. \\
\tab\tab- materiali ceramici refrattari: materiali resistenti alle alte temperature e abrasivi\\
\tab- ceramici avanzati: ottenuti per sintesi chimica, presentano meno impurità.\\
\tab\tab- hanno elevata resistenza alle alte temperature e alla corrosione. \\
\tab\tab- trovano grande utilizzo nel campo automobilistico, aerospaziale ed aeronautico
\subsection{Legno}
Tessuto vegetale che costituisce il fusto delle piante
Tobusto e resistente
Combustibile. 
\subsection{Materiali tessili}
Materiali realizzati grazie alla tessitura (operazione che consiste nella tessitura dell'ordito (fili tesi sul telaio) con la trama(il filo che percorre a destra e sinistra fra i vari orditi))\\
\tab- Fibre artificiali: quelle che derivano da sostanze naturali, vegetali o animali. \\
\tab- Fibre sinstetiche: partono da sostanze ottenute dal petrolio chiamati polimeri ottenuti grazie alla polimerizzazione. \\
\subsection{Nanotecnologie}
Un ramo della scienza applicata alla tecnologia su scala nanometrica. per ottenere proprietà chimiche e fisiche differenti. Esempio i polimeri più leggeri ma rinforzati per aumentare la robustezza del materiale ma con peso minore. 
