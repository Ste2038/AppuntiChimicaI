\section{Classificazione delle reazioni chimiche}
\subsection{Reazioni chimiche}
Molte reazioni si svolgono in ambiente acquoso.\\
Solvente + soluto = soluzione\\
Solvente: componente maggioritario\\
Soluto: componente minoritario\\\\
Alcuni sono passivi, disperdono le sostaze disciolte in essi in singole molecole, altri sono definiti attivi, cioè con una forte interazione.\\
\paragraph*{Conduttività elettrica}
capacità di una soluzione di condurre elettricità.\\\\
Una sostanza sciolta in acqua che conduce energia si dice elettrolitica. (Sale binario e ternari sono elettroliti forti)\\
In questo caso l'acqua non compare nella reazione, è usata solo come un mezzo per farla avvenire.\\
L'acqua è in grado di far avvenire queste reazioni grazie alla sua struttura geometrica.\\
Ossigeno ha polo leggermente negativo mentre i due idrogeni hanno poli positivi.\\
Inoltre l'acqua non è composta in forma lineare ma a forma di retta.\\
Ci sono anche composti covalenti che si sciolgono in acqua grazie a legami HO analoghi a quelli dell'acqua.\\
Ci sono però sostanze che non si sciolgono ma rimangono intatte (es. Benzene $C_6H_6$)\\
Si sciolgono in acqua solo composti ionici o composti covalenti con OH.
\subsubsection{Regole di solubilità}
Sono solubili:\\
\tab- I composti con elemento del gruppo 1 o con $NH_4^+$\\
\tab- I nitrati ($NO_3^-$), gli acetati ($CH_3COO, C_2H_2O_2^-$) e la maggior parte dei perclorati($ClO_4^-$).\\
\tab- Tutti i cloruri($Cl^-$), i bromuri($Br^-$) e gli idruri eccetto quelli con $Pb_2$ e del gruppo 2 \\
\tab- Tutti i solfati($SO_4^{--}$) eccetto quelli con $Ca^{++}, Sr^{++}, Ba^{++}, Pb^{++}$\\\\
Sono insolubili:\\
\tab- Idrossidi metallici eccetto M del gruppo 1 e 2 (sotto $Ca^+$)\\
\tab- I carbonati ($CO_3^-$), i solfati ($PO_4^{--}$) eccetto quelli del gruppo 1 e $NH_4^{+}$.\\
\tab- Tutti i solfuri eccetto quelli del gruppo 1, 2 e $NH_4^+$
\subsection{Primo metodo di classificazioni delle reazioni}
\subsubsection{Reazioni di precipitazione}
Due composti ionici solubili reagiscono e creano un precipitato.\\
Si formano per la poca solubilità.\\
Reazione di doppio scambio o metastasi (Quando due composti si scambiano gli elementi).
\subsubsection{Reazione Acido-Base (Neutralizzazione)}
Acido: sostanza che in $H_2O$ produce ioni $H^+$\\
Base: sostanza chei in $H_2O$ produce ioni $OH^-$\\
Elettrolita forte: composta che in acqua si scioglie del tutto\\
Elettrolita debole: composto che in acqua si scioglie solo in parte\\
Se come prodotto troviamo un sale e acqua, la reazione è anche detta di salinizzazione.
\subsubsection{Reazione di Ossido-Riduzione (Redox)}
Si crea grazie al movimento netto degli elettroni da un reagente all'altro in una direzione, da quello che meno attrae elettroni a quello che più li attrae.\\
Se il legame che si crea è ionico, l'elemento più elettronegativo "strappa" gli elettroni all'altro.\\
Questo si chiama trasferimento elettronico.\\
Se il legame che si crea è covalente, gli elettroni vengono solamente condivisi e questo si chiama spostamento elettronico.\\
Nelle reazioni redox:\\
I metalli sono tipicamente riducenti e tenodno a cedere elettroni nelle reazioni.\\
I non metalli sono tipicamente ossidanti e tendono ad acquistare elettroni nelle reazioni.\\
\begin{table}
\end{table}
\paragraph*{Svolgimento}:\\
\tab- si attribuisce N.O. a tutti\\
\tab- calcolare la variazione $\Delta$N.O. per chi cambia\\
\tab- Si moltiplicano $\Delta$N.O. per il numero dei suoi atomi nella specie di partenza.\\
\tab- Il $\Delta$N.O. di un elemento diventa il coefficiente stechiometrico della specie che conviene l'altro el. e viceversa.\\
\tab- Bilanciare tutti gli altri H e O.\\
\tab- Bilanciare H e O.\\
\tab- Verifichiamo che i coefficiente non siano semplificabili.
\subsection{Secondo metodo di classificazioni delle reazioni chimiche}
\subsubsection{Reazione di combinazione (Sintesi)}
Due elementi che si combinano per formare un solo composto più complesso.\\
Es. reazione di combustione $S_8 + 8SO_2$ (con 1 solo prodotto, altrimenti non è sintesi)
\subsubsection{Reazione di decomposizione}
Un reagente, rottura in due elementi\\
si rompono i legami chimici del reagente, per farlo però serve energia.\\
Si può fare in due modi:\\
\tab- termica: rottura creata dal calore.\\
\tab- elettrolitica: inserendo energia elettrica. Esempio: elettrolisi, processo per separare l'acqua e ottenere idrogeno.
\subsubsection{Reazione di scambio}
xz + y = x + yz\\
quando alla reazione partecipa un metallo, l'atomo riduce lo ione (Da elettroni)\\
quando partecipa un non metallo, l'atomo ossida lo ione (prende elettroni)\\\\
i metalli dei gruppi 1 e 2 reagiscono molto bene, ci sono metalli meno reattivi come (Au, Ag) che reagiscono solo con alcuni acidi e altri che non sono reattivi con niente.\\
I metalli vengono classificati per la loro capacità di spostare H da $H_2O$ o da acidi.\\
Li, K, Ba, Ca, Na spostano $H_2$ da $H_2O$\\
Mg, Al, Mn, Zn, Cr, Fe, Cd spostano $H_2$ dal vapore acqueo\\
Co, Ni, Sn, Pb spostano $H_2$ da alcuni acidi\\
Cu, Mg, Au, Ag non spostano $H_2$\\\\
Si può spostare lo ione di un metallo con un altro metallo. Es: se metti in acqua rame e zinco dopo un po lo trovi lo zinco attaccato al rame.\\
Gli scambi possono avvenire negli alogeni secondo reattività decrescente. Quindi:\\
$F_2 > Cl_2 > Br_2 > I_2$ Più è in alto, più è ossidante.