\section{La stechiometria}
\subsection{Mole}
Quantità di sostanza di un sistema che contiene tante entità elementari quanti sono gli atomi di 12g di Carbonio(12).
\subsection{Numero di Avogadro}
$6,022*10^{23}$ senza unità di misura. Corrisponde al numero di atomi prensenti in 12g di Carbonio(12).
\subsection{Costante di Avogadro}
$6,022*10^{23}\ mol^{-1}$: numero di particelle in 1 mol.
\subsection{MA: massa atomica}
Se si parla di MA di un elemento si misura in Da(Dalton). Se invece si parla di 1 mol di atomi si esprimono in g(grammi)
\subsection{MM: massa molare}
Si tratta della somma dei valori di MA dei componenti di un composto. Se il composto è formato solo da un elemento allora MM = MA:\\
$n = \frac{m}{MM} \rightarrow [mol] = \frac{[g]}{[g]*[mol^-1]}$
\subsection{Percentuale in massa}
$\%E = \frac{a * (MA_e)}{MM}$
\subsection{Analisi per combustione}
Si porta un campione di materia in un forno, fornendo ossigeno si spinge il compoto in forma gassosa all'itnenro di alcuni assorbitori.\\
Assorbitore 1 raccoglie $H_2O$ dal composto\\
Assorbitore 2 raccoglie $CO_2$\\
Quello che rimane nel composto gassoso lo si lascia depositare in una terza camera per poi raccoglierlo ed effettuare i pesi.
\subsection{Problemi di Stechiometrica}
- Bilanciare equazione\\
- Convertire masse in moli\\
- Data 1 sola mole si trovano tutte le altre\\
- Si possono riottenere le masse
\subsection{Reagente limitante}
Se abbiamo dei reagenti, molto probabilmente uno dei due limita l'altro.\\
Il limitante va trovato capendo quale dei due utilizza meno moli(attenzione che le moli vanno divise per il coefficiente stechiometrico).\\
Quello che utilizza meno moli è l'agente limitante.
\subsection{Resa teorica}
La resa teorica è la resa che non comprende i problemi di laboratorio come residui etc.\\
Potrebbero accadere:\\
\tab- reazioni collaterali\\
\tab- equilibrio chimico\\
\tab- perdite di materia nel laboratorio.
\subsubsection{Percentuale di resa}
$\eta = \frac{resa\ reale}{resa\ teorica}$