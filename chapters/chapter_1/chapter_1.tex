\section{Introduzione}
\subsection{Stati della materia}
- Solido: le particelle costituenti sono strettamente impacchettate e ne risulta un'aggregazione rigida perchè le particelle non riescono a muoversi dalla loro posizione\\
- Liquido: forma fluida della materia, la cui superficie è ben definita, mentre la forma geometrica assunta è quella della porzione di recipiente che occupa. Le molecole sono impacchettate come nel solido ma hanno abbastanza energia per scorrere fra loro.\\
- Gas: Forma fluida della materia, che riempie interamente qualsiasi recipiente lo contenga. Le particelle godono di una libertà quasi totale: possono muoversi nello spazio, urtandosi. (Vapore è il termine che viene usato per indicare una sostanza che normalmente si presenta come solida o liquida)
\subsection{Proprietà della materia}
- Proprietà fisiche: caratteristiche che possono essere osservate e misurate senza mutare l'identità della sostanza in esame (Es. massa, temperatura, colore, densità, stato di aggregazione)\\
- Proprietà chimiche: caratteristiche che si riferiscono all'attitudine di una sostanza a mutarsi in o interagire (Es. idrogeno tende a reagire con ossigeno per creare acqua)
\subsection{Tipologia di proprietà}
- Proprietà estensive: dipendono dalla diensione del campione (massa, volume, altro)\\
- Proprietà intensive: non dipendono dalla dimensione del campione (temperatura, colore, densità, stato d'aggregazione, altro)\\
Energia cinetica: $E_k = \frac{mv^2}{2}$\\
Potenziale: $E_p = mgh = \frac{Q_1 Q_2}{4\pi\epsilon_0 r}$