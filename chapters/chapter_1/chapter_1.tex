\section{Introduzione}
\subsection{Stati della materia}
- Solido: le particelle costituenti sono strettamente impacchettate e ne risulta un'aggregazione rigida perchè le particelle non riescono a muoversi dalla loro posizione\\
- Liquido: forma fluida della materia, la cui superficie è ben definita, mentre la forma geometrica assunta è quella della porzione di recipiente che occupa. Le molecole sono impacchettate come nel solido ma hanno abbastanza energia per scorrere fra loro.\\
- Gas: Forma fluida della materia, che riempie interamente qualsiasi recipiente lo contenga. Le particelle godono di una libertà quasi totale: possono muoversi nello spazio, urtandosi. (Vapore è il termine che viene usato per indicare una sostanza che normalmente si presenta come solida o liquida)
\subsection{Proprietà della materia}
- Proprietà fisiche: caratteristiche che possono essere osservate e misurate senza mutare l'identità della sostanza in esame (Es. massa, temperatura, colore, densità, stato di aggregazione)\\
- Proprietà chimiche: caratteristiche che si riferiscono all'attitudine di una sostanza a mutarsi in o interagire (Es. idrogeno tende a reagire con ossigeno per creare acqua)
\subsection{Tipologia di proprietà}
- Proprietà estensive: dipendono dalla diensione del campione (massa, volume, altro)\\
- Proprietà intensive: non dipendono dalla dimensione del campione (temperatura, colore, densità, stato d'aggregazione, altro)\\
Energia cinetica: $E_k = \frac{mv^2}{2}$\\
Potenziale: $E_p = mgh = \frac{Q_1 Q_2}{4\pi\epsilon_0 r}$
\subsection{Componenti della materia}
\subsubsection{Elementi}
\subparagraph{Atomo}
Singola unità di materia, la più semplice, non divisibile chiaramente.\\
Un atomo di un certo elemento non può diventare atomo di un altro elemento.\\
Due atomi dello stesso elemento sono identici, con proprietà costanti.\\
(es. atomo di oro)
\subparagraph{Molecole}
2 o più atomi dello stesso elemento chimico (es. $O_2$)
\subsubsection{Composto}
2 o più atomi di elementi differenti legati chimicamente.
\subsubsection{Miscela}
Due sostanze mescolate uniformemente fra loro in modo fisico ma non chimico (es. Acqua e sale).\\
Le priorità finali di una miscela sono simili a quelle dei due componenti, a differenza dei composti. I suoi componenti sono inoltre separabii con metodi fisici.\\
Osserviamo che le proprietà chimiche dei composti sono differenti dalle proprietà degi elementi primitivi che lo compongono.\\
Nelle miscele eterogenee si devono poter osservare le sue differenti fasi, a volte sempre il microscopio (esempio il latte) oppure ad occhio nudo (esempio acqua e olio).\\
Nelle miscele omogenee non si osservano più fasi, i due o più composti si mescolano profondamente fino ad una certa quantità.
\subsection{Storia}
\subsubsection{Legge di conservazione delle masse [Lavoisier]}
La prima legge della chimica, nulla si crea, nulla si distrugge, tutto si trasforma.\\
Questo venne scoperto da lavoisier ponendo dei contenitori con due composti separati. Osservò che una volta mescolati i due composti avvenne la ocmbustione prevista ma il peso rimase invariato.
\subsubsection{Legge di composizione definita e costante [Proust]}
Egli scopre poi che il rapporto fra gli elementi che compongono la materia in analisi è costante.\\
Questo ci permette oggi di sapere che percentuale di un determinato componente c'è nel composto finale.
\subsubsection{Legge delle proporzioni multiple[Dalton]}
Egli infine comprende che dati due elementi A e B, essi possono creare composti differenti in base al rapporto fra i due.\\
(es. $CO$ e $CO_2$)\\
Inoltre il rapporto è sempre esprimibile grazie ad un numero intero.
\subsubsection{Modello nucleare di Dalton}
Egli assegna il peso di 1 ad H e 8 ad O. Ad oggi H pesa 1 ma O pesa 16, non 8. Questo deriva dal fatto che al tempo si credeva l'acqua fosse HO, non $H_2O$.\\
Questo errore portò ad avere una tavola periodica iniziale con molti errori.\\
Dalton non sapeva che i numeri dovevano essere tutti interi e nemmeno che l'atomo è particella più piccola esistente.
\subsubsection{Modello nucleare di Rutherford}
Egli osserva che ci sono cariche positive (+) e li chiama protoni.\\
Si viene a creare quindi un nuovo modello nucleare, si capisce che c'era un piccolo nucleo positivo con cariche negative circostanti.\\
Viene inventato quindi il numero atomico e si capisce che è differente per ogni elemento.\\
Numero atomico: il numero di protoni all'interno di un atomo (Z).
\subsubsection{Spettrometro di massa}
E' uno strumento che viene utilizzato per determinare il numero dei protoni.\\
All'inizio si credeva di misurare i protoni, poi si notano in realtà 3 misurazioni. Questo portò alla scoperta dei neutroni.\\
Quello che is trova era infatti il numero di massa.\\
Numero di massa: somma del numero di nucleoni e di protoni.\\
Neutroni = Numero di massa - numero atomico.\\
Si capisce che l'atomo ha componenti di carica positiva, negativa e alcuni neutri. Il nucleo è formato dai neutroni e protoni e costituiscono il 99\% del peso dell'atomo.
\subsubsection{Isotopi}
Isotopi: atomi dello stesso elemento(quindi con lo stesso numero atomico) ma con un diverso numero di massa.\\
(es. Neon-20, Neon-21, Neon-22)\\
Sono tre isotopi del neon con differente numero di massa (20, 21, 22) ma uguale numero atomico (10). Si capisce quindi che il numero di neutroni varia ed è pari a (10, 11, 12).
\subsubsection{Regole post-Dalton}
- L'atomo rimane la più piccola particella individuale, in quanto è lui a mantenere inalterata la sua identità nelle reazioni chimiche. In realtà sappiamo che l'atomo è a sua volta composto da particelle sub-atomiche(elettroni, neutroni, protoni)\\
- Gli atomi di un elemento non possono essere trasformati in atomi di un altro elemento. Questo è in realtà possibile ma solo con  le ultime tecnologie nucleari.\\
- Tutti gli atomi di un elemento hanno lo stesso numero di protoni ed elettroni e ciò determina il comportamento chimico di un elemento: esistono in realtà isotopi diversi per ciascun elemento, ma viene considerato per convenzione la massa media.\\
- I composti sono fermati dalla combinazione chimica di uno specifico rapporto di atomi di diversi elementi: in realtà esistono alcuni composti che possono presentare lievi variazioni dei rapporti dei loro atomi.
\subsubsection{Forza nucleare}
C'è una forza fisica che tiene gli atomi coposti anche se al loro interno sono presenti cariche dello stesso segno, le quali dovrebbero respingersi. \\
Diametro tipico di un nucleo. $10^{-15}$
\subsection{Tavola periodica [Mendeleev](1871)}
Viene creata per la prima volta una tabella che ci da le principali informazioni riguardo a tutti gli elementi chimici.\\
Ogni casella ha un elemento e sono disposti per numero atomico(z) crescente in orizzontale (non per numero di massa).\\
Le righe orizzontali si chiamano periodi e sono 7.\\
Le colonne verticali si chiamano gruppi e sono 18.\\
I gruppi 1, 2, 13-18 sono i gruppi principali.\\
I gruppi 3-12 sono chiamati gruppi di transizione.\\
Ci sono poi due righe aggiuntive poste sotto la tabella, queste due righe andrebbero compresse nel gruppo 3, periodi 6 e 7.\\
Dividiamo gli elementi in 3 gruppi:\\
- metalli: gruppi 1-13(circa, più alcuni elementi dei gruppi seguenti). Sono caratterizzati da malleabilità, duttilità, lucentezza, ottimi conduttori termici ed elettrici.\\
- semimetalli: elementi posti fra metalli e non metalli.\\
- non metalli: elementi posti sulla destra della tavola. Sono caratterizzati da pessime capacità conduttive.\\\\
In particolare:\\
- Gruppo 1: metalli alcalini.\\
- Gruppo 2: metalli alcalino terrosi.\\
- Gruppo 17: elementi alogeni. \\
- Gruppo 18: gas nobili (gli unici che si trovano solo singolarmente in natura, sono: He, Ne, Ar, Kr, Xe, Rn).
\subsubsection{Raggio Atomico}
- aumenta lungo un gruppo.\\
- diminuisce lungo un periodo
\subsubsection{Energia di ionizzazione}
- diminuisce lungo un gruppo.\\
- aumenta lungo un periodo.
\subsubsection{Affinità elettronica}
- diminusce lungo un gruppo.\\
- aumenta lungo un periodo.
\subsubsection{Comportamento metallico}
Il comportamento metallico, fra i metalli:
- aumenta lungo un gruppo.\\
- diminuisce lungo un periodo.\\
Ci sono però delle eccezioni:\\
- C è un non metallo, ma sotto forma di grafite è un buon conduttore elettrico.\\
- I è un non metallo, ma è un solido lucente.\\
- Ga e Cs sono metalli ma fondono a basse temperature.\\
- Hg è liquido a temperatura ambiente.
\subsubsection{Raggio ionico}
E' una stima delle dimensioni di uno ione cristallino.\\
Si può prevedere grazie alla relazione tra il raggio atomico e $Z_{eff}$:\\
- i cationi sono sempre più piccoli degli atomi originali poicè avendo sottratto elettroni, il nucleo attrae più vicino a se gli elettroni.\\
- gli anioni sono più grandi per l'effetto contrario rispetto ai cationi.\\\\
In tavola periodica:\\
- aumenta lungo un gruppo.\\
- diminuisce lungo un periodo