\section{Teorie del legame covalente}
La teoria VSEPR manca di un elemento il fatto che non si riesce a capire come si è arrivati a quella formula.
\subsection{Teoria del legame di valenza}
Questa teoria che un legame covalente nasce con la sovrapposizione di due orbitali:\\
\tab- Spin opposti: lo spazio massimo dell'orbitale è di due elettroni con spin opposto.\\
\tab- Massima sovrapposizione degli orbitali di legame: più gli orbitali si sovrappongono, più sono forti.\\
\tab- Ibridazione degli orbitali atomici: occorre dimostrare perchè due orbitali differenti, con forme differenti, possono sovrapporsi. Viene quindi ipotizzato da Linus Pauling un processo chiamato di Mescolamento in cui degli orbitali si uniscono tramite Ibridazione e creano dei nuovi orbitali ibridi.\\
\tab\tab- Il numero di orbitali ibridi ottenuti è uguale al numero di orbitali atomici mescolati.\\
\tab\tab- Il tipo di orbitali ibridi ottenuti varia in un'azione dei tipi di orbitali atomici mescolati.\\
\tab\tab- La presenza di un certo tipo di orbitale ibrido viene ipotizzata solo dopo aver osservato la sua geometria molecolare.\\
\tab\tab- Tipi di Ibridazione:
\tab\tab\tab- Ibridazione Sp: quando si ha una geometria lineare, mescolando due orbitali non equivalenti s e p, come da nome. E si ottengono due orbitali ibridi sp. Sarà costituito da due lobi, uno grande proveniente da p e uno piccolo proveniente da s.\\
\tab\tab\tab- Ibridazione Sp2: mescolando due orbitali p con 1 orbitale s.\\
\tab\tab\tab- Ibridazione Sp3: mescolando tre orbitali p con 1 orbitale s.\\
\tab\tab\tab- Ibridazione Sp3d: mescolando tre orbitali p con 1 orbitale s e 1 orbitale d.\\
\tab\tab\tab- Ibridazione Sp3d2: mescolando tre orbitali p con 1 orbitale s  e 2 orbitale d.\\\\
Quando un elettrone passa da un orbitale s ad un orbitale ibrido significa che è in atto una promozione elettronica.
\subsubsection{Sovrapposizione di testa}
Avviene quando fra due atomi c'è una sovrapposizione dell'estremità di un orbitale all'estremità dell'altro lungo l'asse che unisce i due atomi. Si orgigina così un legame $\sigma$ andando a formare un elissoide, ellisse 3d. Esempio $C_2H_6$
\subsubsection{Sovrapposizione di gianco (o laterale)}
Avviene quando due atomi si sovrappongono di fianco. Il legame viene chiamato legame $\pi$.\\
Esempio: $C_2H_4$\\
In questo elemento prima si forma un legame di testa fra i due C, poi in avanzo avremo un orbitale ibrido 2sp che rimane con un elettrone sopra e uno sotto e quindi tenderà ad unirsi con lo stesso orbitale lasciato libero nell’altro carbonio, formando un legame pi grego. Ecco da dove nascono i doppi legami\\\\
Oppure\\
Potrebbe capitare con ad esempio in $C_2H_2$ che i due carboni in geometria lineare abbiano due orbitali ibridi 2p con all’interno 2 elettroni in totale, uno disposto lungo asse X, e uno lungo asse Y. Questi due andranno quindi a formare, grazie a sovrapposizione di fianco con gli stessi orbitali dell’altro carbonio, due legami pi greco, uno per orbitale di atomo. Ecco da dove nasce il legame triplo\\
Il legame pi greco elimina un asse di rotazione all’atomo all’interno della molecola.
\subsection{Teoria degli orbitali molecolari}
A differenza della teoria degli orbitali di valenza, questa teoria considera tutti gli orbitali, anche quelli più interni.\\
Per considerarli tutti andiamo ad addizionare fra loro le funzioni d’onda(quelle che indicano la propabilità di localizzazione di un elettrone nell’orbitale): Quello che otteniamo sarà un orbitale molecolare di legame, questo orbitale ha un'energia più bassa rispetto alle energie degli orbitali atomici dei due atomi separati. Questo permette la delocalizzazione delle cariche su un volume maggiore \\
Se invece andiamo a sottrarre le funzioni d’onda otteniamo un orbitale molecolare di antilegame, cioè un nodo tra i nuclei, oppure una regione in cui la densità elettronica è nulla\\
Questo orbitale ha energia più alta sia dell’orbitale molecolare di legame, sia dei singoli orbitali degli atomi non ancora uniti insieme.\\
Gli orbitali molecolari possono crearsi solo se i due atomi hanno energie e orientamento simili. 
\subsection{Ordine di legame}
O.L. = $\frac{1}{2}$ (Eleganti - Eantileganti)\\
Se l’ordine di legame è <= 0 la molecola non può esistere.\\
Ecco perchè non esistono le molecole di gas nobili\\
Questa teoria può essere usata per descrivere Molecole biatomiche omonucleari, cioè costituite da due atomi identici. questa teoria dice che si possono creare, anche se per poco tempo, delle molecole biatomiche omonucleari con alcuni metalli che finora abbiamo sempre visto come singoli atomi. Basta che l’ordine di legame risulti > di 0.\\
Es. Litio, se proviamo a fare $Li_2$ otteniamo un ordine di legame = 1. quindi si potrebbe teoricamente creare. \\
Mentre ad esempio $Be_2$ è impossibile da realizzare poiche l’ordine di legame è 0.
\subsection{Ordini dei livelli energetici}
\tab- Gli orbitali molecolari dervianti da orbitali 2s hanno energia inferiiore a quella degli orbitali molecolari derivanti da orbitali 2p\\
\tab- Gli orbitali molecolari legnati hanno energia inferiore a quella degli orbitali molecolari antileganti\\
\tab- Gli orbitali atomici p hanno capacità di interagire più estesamente di testa che di fianco. Pertanto l’orbitale molecolare $\Sigma 2p$ ha energia inferiore rispetto a quella dell’orbitale $\pi 2p$. Analogamente, l’orbitale $\sigma^*2p$ avrà un effetto destabilizzante maggiore rispetto a quello dell’orbitale $\pi^* 2p$.\\
\tab- avremo quindi ottenuto il seguente ordine energetico: $\sigma 2p$ < $\pi 2p$ < $\pi^* 2p$ < $\sigma^* 2p$\\
\tab- Gli elementi $B_2$, $C_2$, $N_2$ hanno un particolare fenomeno di mescolamento fra gli orbitali $\sigma 2p$ e $\pi 2p$, invertendo i primi due termini dell’ordine scritto al punto precedente: $\pi 2p$ < $\sigma 2p$ < $\pi^* 2p$ < $\sigma^*2p$\\
Questa nuova teoria riesce a spiegare alcune proprietà magnetiche non ancora spiegate fin ora.
\subsection{Molecole biatomiche eteronucleari}
quelle costituite da due atomi diversi, presentano un diagramma asimmetrico, infatti i due orbitali hanno energie diverse. \\
Esempio: \\
HF\\
nela molecola HF l’orbitale 1s di H avrà energia superiore all’orbitale 2p del F, quindi molto maggiore rispetto all’orbitale 1s di F.\\
Con una sovrapposizione di testa andiamo a generare i due legami $\sigma$ e $\sigma^*$\\
Dato che così si genera un solo orbitale sigma, ma la teoria dice che se ne leghiamo 3 di un orbitale p, ne dobbiamo avere tre, allora i due orbitali già pieni di 2p dell’atomo di F andranno in mezzo nel grafico con la stessa energia di 2p e li chiameremo orbitali molecolari non leganti 
