\section{La forma delle molecole}
\subsection{Formule di Lewis}
\tab1) Calcolo del numero di elettroni: si sommano gli elettroni di valenza di tutti gli atomi (Sei si tratta di ioni) il numero di cariche coinvolte.\\
\tab2) Collocazione degli atomi: data una formula AB, si colloca l’atomo A in centro e gli atomi B attorno ad esso.\\
\tab3) Collocazione degli elettroni: si dispongono gli elettroni di valenza attorno a ciascun atomo.\\
\tab4) Formazione di legami singoli: si traccia un segmento che connette i due atomi legati, sostituendo i due elettroni precedentemente disegnati.\\
\tab5) Formazione di legami multipli: qualora la formazione di legmai singoli non permetta il completamento dell’ottetto di alcuni atomi, si prcede alla formazione di legami dopopi o tripli.\\
\tab6) Verifica: il numero di elettroni disegnati(di legame e di non legame) deve coincidere con la sommatoria calcolata al punto 1 di questo procedimento.\\\\
Iniziamo a disegnare formule con 3 elementi. Esempio HEO (H = idrogeno, E = elemento, O = ossigeno)\\
Iniziamo mettendo al centro l’elemento con più connessioni. \\\\
es. CH4O\\
\tab 1) calcoliamo il numero di elettroni coinvolti:\\
\tab\tab a)C:  4 elettroni\\
\tab\tab b) H:  1 elettrone\\
\tab\tab c) O:  6 elettroni\\
\tab\tab d) non ci sono cariche ioniche\\
\tab\tab e) somma: 4+1*4+6 = 14 elettroni di valenza\\
\tab 2) mettiamo quindi il carbonio al centro perchè può formare più legami:\\
\tab\tab a) C forma 4 legami\\
\tab\tab c) O forma 2 legami
\subsection{Molecole elettron deficienti}
I composti o ioni formati con Be, B, Al che non raggiungeranno mai come somma 8 ma solo 6 elettroni
\subsection{Molecole con numero dispari di elettroni}
Sono detti radicali liberi e contengono un elettrone solitario, libero, disaccoppiato. Questo rende la nostra specie paramagnetica. Ad esempio in $NO_2$, N rimane con un elettrone solitario.
\subsection{Gusci di valenza espansi}
Molte molecole e molti ioni hanno più di 8 elettroni attorno all'atomo centrale. Succede infatti che un atomo espanda il suo guscio di valenza per formare più legami.\\
Es: $SF_6$\\
Questa cosa la possono fare solo con non metalli centrali (come ad esempio S) del periodo 3 o superiore. \\
Ma non sempre questi elementi hanno gusci espani. In $H_2SO_4$ ad esempio S ha normalmente 8 elettroni. \\
Si può disegnare anche la struttura con al posto dei due legami doppi, due legami dativi, ma questo disegno è completamente errato disegnarlo, NON RAPPRESENTA UN IBRIDO DI RISONANZA. \\
Invece sono strutture di risonaza le 6 che escono per il composto $\left[SO_4\right]^{2-}$
\subsection{Geometria molecolare}
Si basa sulla teoria  VSEPR (Valence-Shell Electron-Pair repulsion)\\
Si definisce gruppo di elettroni qualsiasi insieme di elettroni che occupa una regione localizzata attorno all’atomo, ovvero costituito da: legame singolo, doppio, triplo o da una coppia solitaria (come unico gruppo) \\\\
Nel caso in cui si hanno 2 elementi oltre al nucleo si chiama: geometria lineare\\
3: geometria planare triangolare (Angolo: 120)\\
4: geometria tetraedrica (Angolo: 109.5)\\
5: geometria bipiramidale trigonale (Angolo 90 e 120)\\
6: ottaedrica (Angolo 90)\\\\
L’angolo tra parentesi viene chiamato angolo di legame, e anche detto angolo ideale perchè nella realtà va a variare un po\\
Il numero di gruppi di elettroni dell’atomo centrale si indica spesso come Numero Sterico (N.S.)\\
Gli elettroni solitari collocati sugli atomi esterni non influenzano la geometria molecolare.
\subsection{Geometria lineare}
2 gruppi disposti con angolo a 180 gradi.\\
La formula indicata come $AX_2$.\\
Esempio: $BeCl_2$\
$CO_2$ (Importante esame) spessi si sbaglia perchè il carbonio ha 2 legami doppi.
\subsection{Geometria triangolare planare}
3 gruppi disposti con angolo a 120 gradi.\\
La formula indicata come $AX_3$\\
Esempio: $BF_3$, $\left[NO_3\right]^-$\\\\
oppure\\
2 gruppi di elettroni appaiati e uno di elettroni non di non legame.\\
Con formula $AX_2E$\\
Con A al centro, 2X e 1 E che stanno attorno, in questo caso gli angoli non saranno tutti uguali all'ideale 120 ma fra le due X avremo un angolo minore di quello ideale (di un paio di gradi) mentre fra A e E abbiamo un angolo appena migliore.\
\subsection{Geometria tetraedrica}
4 gruppi disposti con angolo di 109.5 gradi\\
con formula $AX_4$\\
Esempio: $CH_4$ per disegnarlo, essendo 3 dimensionali si ricorre al disegno prospettico. Esso indica con linee continue i legami sul piano del foglio, linee tratteggiate per i legami che si trovano al di sotto del foglio e cunei per quelli che si trovano al di sopra del foglio.\\\\
oppure \\
abbiamo il caso in cui abbiamo 3 gruppi con un elemento e 1 gruppo di elettroni di non legame\\
Geometria trigonale piramidale\\
con formula $AX_3E$\\
con A al centro, 3 X con angolo fra loro di minore di 109.5 e un angolo maggiore fra X e E\\
esempio: \\
$NH_3$. che, se aggiunto un altro idrogeno diventerà tetraedrica.\\\\
oppure\\
abbiamo il caso in cui abbiamo 2 gruppi di elettroni legati e 2 gruppi di elettroni di non legame\\
Geometria Angolare\\
fra i due gruppi legati avremo un angolo inferiore a 109.5
\subsection{Geometria bipiramidale triangolare}
5 gruppi disposti con angolo di 120 e 90 gradi \\
con formula $AX_5$\\
i gruppi di elettroni si dividono in due categorie\\
\tab- gruppi equatoriali: i 3 gruppi che giacciono sul piano triangolare centrale, con angolo da 120\\
\tab- gruppi assiali: i 2 gruppi che formano i vertici dei due triangoli, uno sopra il piano e uno simmetrico sotto, con angolo da 90\\\\
esempio:\\
$PCl_4$\\\\
oppure \\
caso in cui abbiamo 4 gruppi di legame e 1 di non legame\\
geometrie ad altalena\\\\
oppure \\
caso in cui abbiamo 3 gruppi di legame e 2 di non legame\\
geometrie a T\\\\
oppure \\
caso in cui abbiamo 2 gruppi di legame e 3 di non legame\\
geometrie ad lineare\\
\subsection{Geometria ottaedrica}
6 gruppi disposti con angolo di 90 gradi \\
con formula $AX_6$\\
esempio:\\
$SF_6$\\\\
oppure\\
caso in cui abbiamo 5 gruppi di legame e 1 di non legame\\
Geometrie Piramidale Quadrata\\
$AX_5E$\\
Esempio:\\
$IF_5$\\\\
oppure\\
caso in cui abbiamo 4 gruppi di legame e 2 di non legame\\
Geometrie Planare Quadrata\\
$AX_4E_2$\\
esempio:\\
$XeF_4$\\\\
Svolgere gli esercizi:\\
\tab- Contare il numero m di atomi X a cui è legato l’atomo centrale A: Questo è il valore di AXm\\
\tab- Contare il numero n di doppietti non condivisi rimasti sull’atomo centrale : questo valore è indicato con En\\
\tab- Disegnare e indicare nome ed angolo di legame della geometria $AX_mE_n$\\
Nel caso in cui si hanno invece più di un atomo centrale le loro geometrie saranno due simmetriche,\\
Esempio $C_2H_6$\\
Oppure ad esempio $CH_3CH_2OH$\\
obbiamo tre atomi centrali che saranno i 2 carbonio e l’ossigeno. Il primo carobonio unisce i tre idrogeni con l’altro carbonio. il secondo carbonio si unisce a sua volta a due idrogeni e l’ossigeno che a sua volta si unisce all’ultimo idrogeno.
\subsection{Polarità di una molecola}
Completamente diverso dal concetto di polarità di legame\\
Se però una molecola è composta SOLO da un legame cov polare allora anche la molecola finale sarà polare. Infatti sperimentalmente si osserva che in un campo elettrico, le molecole polari si orientano rivolgendo le loro cariche parziali verso gli elettrodi con carica di segno opposto. Si definisce Momento di Dipolo il prodotto di queste cariche parziali per la loro distanza reciproca. Unità di misura è il debye(D).\\
La polarità di una molecola è un dato fondamentale definisce i parametri fisico chimici di un elemento. Ci permette di sapere temperatura di fusione ed ebollizione, solubilità, reattività, funzione biologica. \\
Non sempre presenza di legami covalenti polari da origine a una molecola polare\\
Esempio da sapere:\\
$CO_2$\\
Nonostante abbia due legami covalenti polari, molto polari questa molecola è Apolare, Perché i due vettori si annullano, essendo posti ad un angolo di 90 gradi.\\\\
Altro esempio è:\\
$H_2O$ \\
2 legami covalenti polari e due doppietti e complessivamente ha un vettore risultante molto grande che punta verso lalto quindi la molecola è polare\\
E’ un errore grave disegnare la molecola d’acqua come geometria lineare. \\
La temperatura di ebollizione dipende fortemeente dalla polartià della molecola poichè una molecola per bollire deve vincere le forze intermolecolari presenti e separare i vari atomi\\
Ci sono casi come abbiamo visto prima in cui una formula porta a due o più formule di struttura differenti. Queste formule potrebbero avere anche una risultante di polarità differente.\\ 
caso $C_2H_2Cl_2$:\\
in questo caso abbiamo due strutture, una in cui la risultante è un vettore che punta verso il basso, la seconda in cui i vettori si annullano fra di loro. Avremo quindi due molecole completamente differenti, la prima polare, la seconda apolare e con differenti temperature di ebollizione, differenza misurata pari a 13 gradi. 