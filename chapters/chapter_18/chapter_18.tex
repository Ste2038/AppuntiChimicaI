\section{Elettrochimica}
Studio delle reazioni chimiche che producono effetti elttrici e dei fenomeni chimici che sono causati dall'azione di forza elettromotrici.\\
Viene studiata attraverso l'uso di celle elettrochimiche, cioè sistemi in cui una reazione redox viene utilizzata per produrre o utilizzare energia elettrica.\\
Distinguiamo:\\
\tab- Celle che compiono lavoro rilasciando energia libera da reazione spontanea per produrre elettricità. (Es. pile)\\
\tab- Celle che compiono lavoro assorbendo energia libera da una sorgente di elettricità per far procedere una reazione non spontanea. Ad esempio, i sistemi per la separazione dei minerali che contengono.\\\\
Indipendentemente dal tipo di cella, si ha un movimento di elettroni da una specie ad un' altra nella reazione redox.
\subsection{Bilanciamneto per semireazioni}
\tab- La reazione viene divisa in una semireazione di ossidazione e una di riduzione.\\
\tab- Si effettua il bilanciamento di massa e carica di ogni semireazione.\\
\tab- Si moltiplicano tutti i coefficienti stechiometrici di una reazione per un numero affinchè i numeri di elettroni ceduti e acquistati siano uguali.\\
\tab- Si sommano le due semireazioni per ottenere la reazione redox bilanciata.\\
\subsection{Celle elettrochimiche}
Si classificano in funzione della natura termodinamica della reazione:\\
\tab- Cella voltaica (o Galvanica, pila): Usa una reazione spontanea per generare energia elettrica.\\
\tab- Cella elettrolitica (o Elettrolizzatore): Usa energia elettrica per far avvenire una reazione non spontanea. L'energia fornita da una fonte esterna viene utilizzata per trasformare i reagenti in prodotti. L'ambiente compie lavoro sul sistema.\\\\
Hanno in comune:\\
\tab- Due elettrodi: oggetti che conducono la corrente tra la cella e l' ambiente.\\
\tab- Elettrolita: una miscela di ioni, solitamente in soluzione acquosa, che sono coinvolti nelle reazioni o nel trasporto di cariche.\\\\
Un elettrodo può essere definito:\\
\tab- Anodo: elettrodo dove ha luogo la semireazione di ossidazione, si cedono elettroni.\\
\tab- Catodo: elettrodo dove ha luogo la semireazione di riduzione, si acquistano elettroni. \\\\
Per poter essere considerata una cella deve inoltre essere sempre possibile trasferire l'elettricità al di fuori della reazione.
E' possibile distinguere:\\
\tab- Elettrodi attivi: sono elettrodi costituiti da elementi che partecipano alle semireazioni.\\
\tab- Elettrodi inerti: conducono elettroni dentro e fuori la cella ma non ne prendono parte.
\subsection{Pila Daniell}
Formata da:\\
\tab- Anodo: una barra di Zn immersa in una soluzione contenente $Zn^{2+}$.\\
\tab- Catodo: una barra di Cu immersa in una soluzione contenente $Cu^{2+}$.\\
\tab- Ponte salino: un tubo a U rovesciata contenente una soluzione di ioni non reattivi dispersi in gel.\\\\
Globalmente l'anodo ha un eccesso di elettroni e assume polarità negativa, mentre il catodo ha polarità positiva.\\
In assenza del ponte salino, tutto questo smetterebbe di funzionare.
\subsection{Notazione celle voltaiche}
Esiste una notazione semplificata per descrivere i componenti di una cella.\\
Per la pila Daniell: $Zn_{(s)}\ |\ Zn_{(aq)}^{2+}\ ||\ Cu_{(aq)}^{2+}\ |\ Cu_{(s)}$.\\
Regole:\\
\tab- O componenti del compartimento anodico sono scritti a sinistra.\\
\tab- Una linea verticale rappresenta una separazione di fase, mentre una virgola rappresenta i componenti di cella che sono presenti nella stessa fase.\\
\tab- Gli elettrodi sono posti come primo e ultimo termine della rappresentazione.\\
\tab- Una doppia linea verticale separa le semicelle e rappresenta il ponte salino, o più precisamente la separazione di fase alle estremità del ponte.\\
\subsection{Potenziale di cella $\Delta E$}
L'energia elettrica prodotta da una cella voltaica è proporzionale alla differenza di potenziale elettrico tra i due elettrodi.\\
Detta anche differenza di potenziale (d.d.p.) o forza elettromotrice (fem).\\
Gli elettroni sono carichi negativamente quindi fluiscono spontaneamente dal polo negativo a quello positivo, quindi con $\Delta E > 0$.\\
L'unità di misura è il Volt $[V] = \frac{[J]}{[C]}$.\\
\subsubsection{Potenziale standard}
Per poter confrontare diverse celle fra loro dobbiamo utilizzare il potenziale standard di cella ($\Delta E^0$), ovvero quello a 25°C e con tutti i componenti allo stato standard.\\
Se si vogliono confrontare semicelle si possono usare i potenziali elettrodici standard ($E^0$) che si riferisce alla singola semireazione.\\
$\Delta E^0 = E_{catodo}^0 - E_{anodo}^0$
\subsubsection{Elettrodo standard ad idrogeno}
Ovviamente per stabilire i $\Delta E^0$ non ci sono problemi, mentre per i valori di $E^0$ si, essi non sono assoluti e non possono essere misurati. Si utilizza quindi come riferimento la semicella chiamata elettrodo standard ad idrogeno (SHE).
Esso ha valore: $E_{rif}^0 = 0V.$.\\
\subsubsection{Tabella dei potenziali elettrodici standard}
Riporta coppie redox in ordine di $E^0$ decrescenti.\\
Si tenga presente:\\
\tab- Tutti i valori sono relativi all'elettrodo standard ad idrogeno.\\
\tab- Le semireazioni sono scritte come riduzioni.\\
\tab- Più positivo $E^0$, più facilemnte avviene la semireazione\\
\tab- Le semireazioni sono scritte con la freccia di equilibrio, in quanto possono avvenire in entrambe le direzioni a seconda dell'altra semireazione con cui si accoppia.\\
\tab- La forza degli agenti ossidanti (reagenti) aumenta dal basso verso l'alto e la forza degli agenti riducenti (prodotti) aumenta dall'alto verso il basso.\\\\
Ogni reazione redox è la somma di due semireazioni, perciò c'è un agente ossidante e un agente riducente in entrambi i membri dell'equazione.\\
Per scriverla nel verso in cui avviene spontaneamente ricordiamo che la semireazione con $E^0$ maggiore avverrà nel senso della riduzione, mentre quella con a $E^0$ minore avverrà nel senso dell'ossidazione.\\
Ricordare che i valori di $E^0$ non vanno moltiplicati per un numero quando si bilancia, è una grandezza intensiva.\\
%Fine slide 17
\subsection{La serie di attività dei metalli}
Una serie ordinata sulla base:\\
\tab- Capacità di spostare lo ione di un altro metallo nella soluzione\\
\tab- Capacità di spostare $H_2$ dall'acqua o da un altro acido\\\\
Inoltre:\\
\tab- Metallo sposta $H_2$ da acido se $E^0 < 0$\\
\tab- Metallo non sposta $H_2$ da acido se $E^0 > 0$\\
\tab- Metallo sposta $H_2$ dall'acqua se $E^0 < -0.42\ V$\\
\tab- Metallo sposta altri metalli dalla soluzione: il metallo con $E^0$ sposta quello minore.\\
\subsection{Energia libera}
Una reazione spontanea ha $\Delta G < 0$, mentre una reazione elttrochimica spontanea ha $\Delta E > 0$.\\
$\Delta G = -n * F * \Delta E$ con la costante di Faraday $F = 9.65 * 10^4\ J\ V^{-1}\ mol_e^{-1}$.\\
$\Delta E^0 = \frac{RT}{nF}lnK$\\
Ciò significa che una semplice misura di $\Delta E^0$ consente l'ottenimento di K e $\Delta G^0$ per una reazione redox.\\
\subsubsection{Equazione di Nernst}
$\Delta E = \Delta E^0 - \frac{RT}{nF}lnQ$.\\
Il termine n indica il numero di elettroni trasferiti. Coincide con il coefficiente stechiometrico di $e^-$.\\
Sostituendo i valori delle costanti R e F, operando a 25° C e convertendo il log in base 10.\\
Otteniamo: $\Delta = \Delta^0 - \frac{0.0592}{n}LogQ$.
%fine slide 23
\subsection{pHmetro}
Costituito da un elettrodo a vetro e uno di riferimento.\\
Viene misurata la differenza di potenziale fra l'elettrodo di riferimento (a potenziale costante e noto) e l'elettrodo in vetro.\\
In alcuni phMetri si trova un solo elettrodo con i due combinati all'interno.\\
La misurazione viene trasmessa tramite cavo coassiale all'unità centrale.\\
L'elettrodo di riferimento solitamente è Ag oppure AgCl, ma sono stati studiati altri elementi per utilizzi particolari.\\
\subsection{Batterie}
Un insieme di celle voltaiche poste in serie, in modo che le differenze di potenziali si sommino.\\
\tab- Batterie primarie: Non ricaricabili, gettate dopo il suo unico utilizzo.\\
\tab\tab- Batteria a secco (o di Leclanchè): formata da un anodo di Zn e un catodo inerte in grafite. Il potenziale è di 1.5V\\
\tab\tab- Batteria alcalina: segue la batteria a secco. Non ha cadute di tensione e lunghi tempi di vita.\\
\tab\tab- Batteria a bottone: può essere a Hg (1.3V) o Ag (1.6V). Entrambe con un catodo in acciaio.\\
\tab\tab- Batteria primaria al litio: produce 1F con 7g di Li. Il potenziale è di 3V.\\
\tab- Baterie secondarie (o accumulatori): sono ricaricabili:
\tab\tab- Batteria piombo-acida: tipica batteria dell'automobile, 12V, costituita da 6 celle voltaiche collegate in serie.\\
\tab\tab- Batteria Ni-metallo idruro: Utilizzata in apparecchi cordless e flash fotografici, leggera, elevata potenza, si scarica anche se non utilizzata.\\
\tab\tab- Batteria litio-ione: atomi di Li intercalati tra piani di grafite, costo elevato e infiammabilità dei solventi organici presenti.\\
\tab- Celle a combustione (o fuel cell): La più comune utilizza $H_2$ ed una sua combustione mantiene una temperatura di 80°C.\\
\tab- Celle elettrolitiche: Si utilizza energia esterne per far avvenire una reazione non spontanea. La reazione produce elementi fortemente riducenti.\\
\tab\tab- Elettrolisi di sali fusi: separa il metallo dal non metallo in un sale.\\
\tab\tab- Elettrolisi di miscele di sali fusi: se sono presenti più sali fusi, la specie che si ossida più facilmente reagisce all'anodo e la specie si riduce più facilmente reagisce al catodo.\\
\tab\tab- Elettrolisi dell'acqua: all'anodo si ha ossidazione di $H_2O$ con variazione del N.O. di O.\\
\tab\tab- Elettrolisi di soluzioni acquose di elettroliti: occorre studiare se il fenomeno di elettrolisi riguarda l'acqua o un sale. Ciò rientra nello studio di precedenza di scarica. Quando due semireazioni sono possibili ad un elettrodo, avviene la riduzione della specie col potenziale elettrodico più positivo\\
\subsection{Prima legge di Faraday}
La quantità di sostanza prodotta da un processo di elettrolisi è direttamente proporzionale alla quantità di corrente che ha attraversato la cella. La quantità di carica che attraversa la cella è $q(C) = I\ \Delta t$, dove I è l'intensità di corrente e $\Delta t$ è la durata del processo.
\subsection{Seconda legge di Faraday}
A parità di elettroni ch efluiscono attraverso la cella, si ottengono quantità di sostanza diverse a seconda della variazione di NO delle specie. Occorre quindi bilanciare la semireazione, attribuendo i coefficienti stechiometrici sia alle specie del processo di riduzione, sia all'elettrone. Il numero di moli di elettroni si indifca con q(F). Pertanto: $q(C) =q(F)*(9.65 * 10^4)$