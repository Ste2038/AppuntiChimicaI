\section{Modelli del legame chimico}
\subsection{Simboli di Lewis}
I simboli di Lewis sono puntini che vengono disposti intorno ad un elemento, sopra, sotto, destra e sinistra. In ogni posizione possono essercene 1 o 2 (indicabile anche con un segmentino). L'ordine dei punti non è importante, ma va rispettata la regola di Hund:\
\tab- Nei metalli i simboli di Lewis indicano il numeroo massimo di elettroni che il metallo cede formando un catione.\\
\tab- Nei non metalli, il numero dei puntini spaiati, quindi il numero di puntini mancanti per arrivare a 8, è il numero che deve acquistare o avere in condivisione nella reazione.
\subsection{Regola dell'ottetto}
Espressa da Gilbert Newton Lewis.\\
Si dice che il comportamento degli elettroni tende sempre a raggiungere l'ottetto. Quando gli atomi si legano, essi cedono o acquistano o condividono elettroni con l'obbiettivo di raggiungere 8 elettroni nel loro livello esterno.
\subsection{Energia reticolare}
E' la variazione di entalpia che accompagna l'unione di un catione e di un anione nella composizione di un sale. Essa indica l'intensità delle interazioni ioniche e influenza la temperatura di fusione, la durezza e la soubilità dei solidi ionici.\ Si calcola sottraendo la sommatoria del $\Delta H$ nei processi, meno il $\Delta H$ finale.
\subsection{Ciclo di Born-Haber}
E' la procedura che si usa per trovare la variazione di entalpia partendo dagli elementi in stato naturale, fino al raggiungimento del composto voluto conoscendo tutte le entalpie delle reazioni del ciclo. Grazie alla legge di Hess questo ci permette di calcolare la variazione di entalpia della reazione stessa.\\
L’energia reticolare ($\Delta H^0$ reticolare) è direttamente proporzionale all’energia elettrostatica e, mediante la legge di coulomb e la considerazione che la distanza minima tra catione e anione è uguale alla distanza tra i loro centri (che equivale alla somma del loro raggi ionici). \\
Cio permette di studiare:\\
\tab- effetto del raggio ionico: scendendo lungo un gruppo, il raggio ionico aumenta e l’energia di attrazione tra cationi e anioni dovrebbe diminuire a causa dell’aumento della distanza interparticellare. In effetti si osserva che l’energia reticolare diminuisce lungo un gruppo. \\
\tab- effetto della carica ionica: aumenta notevolmente l’energia reticolare. \\\\
Il modello del legame ionico spiega la natura dei solidi ionici. Per esempio NaCl, sale da cucina è duro(non si lascia penetrare facilmente), rigido (non si piega), fragile(si rompe prima di deformarsi)\\
La maggior parte dei composti ionici non conduce elettricità allo stato solido poichè gli ioni sono ordinati e fissi. Se li portiamo allo stato liquido ad esempio con la solvatazione in acqua. gli ioni solo liberi di muoversi e quindi conducono energia. 
\subsection{Elettronegatività}
L'elettronegatività(x) è la capacità relativa di un atomo legato di attrarre gli elettroni condivisi.\\
E' importante che non si tratta di una proprietà di un atomo isolato, ma è sempre riferita ad un atomo legato ad un altro.\\
Gli studi si attribuiscono a Linus Pauling.\\
es.\\
$H_2$ (432 $\frac{Kj}{mol}$)\\
$F_2$ (159 $\frac{Kj}{mol}$)\\
$HF$ (565 $\frac{Kj}{mol}$) notiamo che HF non è uguale alla media fra H e F.\\\\
Questa differenza era dovuta al contributo elettrostatico.\\
Secondo Pauling il fluoro attrae la coppia di elettroni di legame più di H (Cioè F è più elettronegattivo di H /oppure/ H è più elettropositvo di F) quindi la coppia di elettroni condivisi sta fisicamente più vicino al fluoro rispetto che all'idrogeno..\\
Dopo molti studi Pauling creo una scala dei valori relativi di elettronegatività\\\\
Il numero di ossidazione può anche essere calcolato come:\\
N.O. = eValenza - (eValenzaCondivisi + eValenzaNonCondivisi)
\subsection{Legame Ionico}
Il legame ionico è il trasferimento di elettroni da un metallo a un non metallo per formare ioni che si uniscono in un composot ionico solido. Il numero totale di elettroni ceduti dagli atomi metallici è uguale al numero di elettroni acquistati dagli atomi non metallici. Questo scambio di elettroni produce energia, questo contributo è detto energia reticolare.\
Questa energia viene calcolata attraverso la legge di Hess applicata al ciclo di Born-Haber.
\subsection{Legame covalente}
Si forma quando i due atomi hanno poca differenza di elettronegatività, di solito fra 2 non metalli. Tendono sempre ad ottenere la stabilità e ad avere la configurazione elettronica uguale al gas nobile vicino.\\
Studiamo la formazione di $H_2$\\
I due atomi si avvicinano diminuendo lo spazio internucleare. \\
Nel momento in cui i due atomi si avvicinano l’energia potenziale sarebbe minima e i due elettroni sono legati fra loro.\\
Se si tenta ad avvicinare ancora di più i due atomi l’energia potenziale cresce quindi i due non devono stare troppo vicino.\\
74 picometri è la distanza internucleare nel momento in cui due atomi di idrogeno si legamo covalentemente. (Potenziale di Lennard-Jones)\\
Questo legame si indica con H:H oppure H-H\\
Gli elettroni condivisi vengono chiamati Coppia di legame. Gli elettroni in coppia non condivise vengono chiamate coppie solitarie.\\
L’ordine di legame è il numero di coppie di elettroni che vengono condivisi. In questo caso vale 1. \\\\
Es. in $C_2H_4$ troviamo un legame doppio fra gli atomi di carbonio.\\
Es2. in $N_2$ troviamo un legame triplo fra i due atomi di azoto.\\\\
L’energia di legame (o entalpia di attrazione) è l’energia necessaria per vincere l’attrazione fra i due atomi ($\frac{Kj}{mol}$)\\
La rottura di un legame è un processo endotermico, quindi l’energia di legame è sempre positiva\\
La lunghezza di legame è la distanza tra i due nuclei legati. la lunghezza di legame è strettamente legata all’ordine di legame. (misurata in picometri)\\
Il legame covalente crea:\\
\tab- forti legami intramolecolare(cioè all’interno della molecola)\\
\tab- deboli legami intermolecolari (cioè fra molecole): questo porta a deboli proprietà meccaniche.
\subsubsection{Legame covalente reticolare}
In questo caso non si hanno molecole separate, ma si tratta di reticoli tridimensionali formati da legami covalenti tra tutti gli atomi del campione. Questo comporta qualità meccaniche eccezionali. Es. Il diamante, il quarzo.\\
I composti covalenti sono solitamente pessimi conduttori elettrici.
\subsubsection{Legame covalente dativo}
In questo legame troveremo un atomo che dona una coppia di legame e sarà chiamato donatore, mentre l'altro atomo, che avrà un oritale vuoto pronto ad accettare la coppia donata, si chiamerà accettore.\\
Per distinguerlo dal normale legame covalente invece che il trattino disegneremo una freccia dal donatore verso l’accettore, seguendo il flusso degli elettroni.\\\\
A volte la risoluzione del problema può avvenire con più strutture, ad esempio per O3, avremo la prima struttura con un dativo e poi un doppio, e la seconda struttura con prima il doppio e poi il dativo.\\
Ma in chimica non esistono due strutture diverse dello stesso composto.\\
Inoltre, nessuna di queste due strutture alla fine è quella corretta. \\\\
Le due strutture correttamente rappresentate ma errate vengono dette  strutture di risonanza, ma in realtà quella giusta è la medi afra le due e viene detta ibrido di risonanza. \\
(come se invece di avere un legame 1 e un legame 2, avessimo 2 legami da 1.5)\\
Questo fenomeno di elettroni non correttamente posizionati nelle strutture di risonanza viene detto delocalizzazione di coppie di elettroni\\
Se dovesse succedere durante l’esame disegneremo solo la più importante, che è determinata dalla carica formale.\\
C.F = eValenza - (eValenzaNonCondivisi + ½ eValenzaCondivisi)\\
La somma delle cariche formali di tutti gli elementi rispetterà la carica ionica del composto, se ione, altrimenti sarà 0\\
Criteri per trovare la migliore:\\
\tab- Le C.F. più piccole sono preferibili a quelle grandi\\
\tab- Non sono desiderabili C.F simili su due atomi adiacenti\\
\tab- Una C.F. più negativa dovrebbe risiedere su un atomo più elettronegativo
\subsubsection{Legame metallico}
Si forma fra metalli grazie alla condivisione di elettroni, non fra 2 ma fra molti atomi. \\
Spesso definito come “un mare di elettroni” condivisi. \\
Secondo li modello di “mare di elettroni” per il legame metallico, tutti gli atomi vanno a condividere con gli altri gli elettroni di valenza. \\
L’insieme di elettroni viene detto delocalizzato perchè non stanno in uno spazio preciso ma si trovano su tutto il volume. Questo è molto differente dagli altri legami che abbiamo visto fin ora.\\\\
Questo permette ad esempio di piegare un metallo, poichè gli elettroni prenderanno la forma desiderata senza spezzare il pezzo, diffrente da quello che succede in un sale ad esmepio.\\
I metalli formano tipicamente leghe, cioè miscele solide con composizione variabile. \\\\
Caratteristiche:\\
\tab- temperatura di fusione molto alta\\
\tab- temperatura di ebollizione moooolto molto alta.\\ 
\tab- duttili\\
\tab- malleabili\\
\tab- buoni conduttori
\subsection{Polarità di legame}
Il legame covalente si divide in polare e puro (o apolare).\\
Polare è quando si uniscono due atomi con elettronegatività diverse, viceversa per il puro. \\
Si chiama attribuzione della polarità di legame l’attribuzione di una polarità al legame. ATTENZIONE DIVERSA DA POLARITA’ DI MOLECOLA\\
Avendo fatto questa distinzione però ora abbiamo due tipi di legami che si basano sulla differenza di elettronegatività, ovvero quello covalente polare e quello ionico, come li distinguiamo?\\
Grazie al numero chiamato Carattere ionico di un legame calcolato così:\\
Come soglia al PoliTo utilizziamo 50\% di ionicità, al disotto abbiamo il covalente polare e sopra quello ionico. \\
E’ più veloce però controllare semplicemente l’elettronegatività, se la differenza è maggiore di 1,7 abbiamo un legame ionico, altrimenti un covalente polare. \\
Oppure consideriamo che lo ionico si ha con M + NM, per il resto si controlla la tavola periodica. 
