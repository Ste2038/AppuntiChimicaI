\section{Configurazione elettronica}
\subsection{Numero quantico di Spin (Ms)}
Il quarto numero quantico descrive una porprietà dell’elettrone (mentre i primi 3 descrivono una proprietà dell’orbitale in cui si trovano). Indica l’orientamento (a dx o sx) dell’elettrone intorno al proprio asse. Può assumere valore $+\frac{1}{2}$ o $-\frac{1}{2}$.\\\\ 
Es. H\\
n = 1; //ha un elettrone, quindi si posiziona nel primo orbitale\\
l = 0; //può  andare da [0; n - 1], in questo caso n-1 = 0\\
Ml = 0 //può andare da [-n+2; n -1], in questo caso 0\\
Ms = $+\frac{1}{2}$ //caratteristica dell’elettrone.\\\\
Quando analizziamo un solo elettrone si può non mettre il segno.\\
Quando analizziamo due o più elettroni si mette $+\frac{1}{2}$ al primo, $-\frac{1}{2}$ al secondo, $+\frac{1}{2}$ al terzo, e così via.\\\\
Il successivo elemento nella tavola periodica è L’elio (He).\\
il primo elettrone ha le stesse caratteristiche quantiche di quelli di H, il secondo no, secondo la teoria di Wolfgang Pauli (Principio di esclusione)\\
Questo principio indica che due elettroni dello stesso atomo non possono avere i 4 numeri quantici uguali. Pertanto il secondo elettrone He sarà descritto dai seguenti numeri quantici:\\
n=1\\
l=0\\
Ml = 0\\
Ms = $-\frac{1}{2}$\\\\
Dato che Ms può contenere solo due valori, capiamo che un orbitale atomico può contenere al massimo due elettroni e questi devono avere spin opposti (antiparalleli)\\\\
Si dice quindi che l’orbitale 1s è pieno (o Completo)\\
Terzo elemento della tavola periodico è il Litio (Li)\\
Il litio ha tre elettroni. in questo caso i primi due si sistemano come per l’elio(He), manca da capire il terzo elettrone, il primo orbitale è però ormai pieno.\\
Si passa quindi al secondo orbitale (orbitale successivo con energia più bassa): orbitale 2 (s o p, non sappiamo quale perchè non sappiamo quale ha energia più bassa)\\
n = 2 //l’elettrone si piazza nel secondo elettroni
\subsection{Splitting}
Capita se noi andiamo a guardare lo spettro dell’He e troveremo che ha esattamente il doppio dei fasci rispetto a quelli di H perche gli elettroni si sono divisi(splitting) in sottolivelli.\\
Questa divisione era prevedibile grazie al numero quantico n e in parte anche grazie a l.\\
QUESTO FENOMENO NON ACCADE PER I SISTEMI MONOELETTRONICI\\\\
Ricordiamo la legge di coulomb.\\
\tab- Maggiore è la distanza tra due cariche opposte, più debole è la loro attrazione reciproca. Quando nucleo ed elettrone sono a grande distanza, l’energia è più alta (il sistema è meno stabile)\\
\tab- Maggiori sono le due quantità di carica di segno opposte, più forte è l’attrazione. Quado un nucleo di carica maggiore attrae un elettrone, l’energia è più bassa(il sistema è più stabile) rispetto a quando un nucleo di carica minore attrae un elettrone\\\\
Cerchiamo di capire dove va il terzo elettrone…\\
Un sistema stabile è un sistema a bassa energia\\
Un sistema instabile è un sistema ad alta energia\\
E’ più facile separare un elettrone da un sistema instabile perchè dobbiamo spendere meno energia. \\\\
Distinguiamo 4 effetti che influenzano l’energia di un orbitale:\\
\tab- Effetto della carica nucleare: Ad esempio se guardiamo idrogeno ed elio notiamo che l’elettrone presente in elio  ha carica $-5250\ {Kj}{mol}$, mentre l’idrogeno ha carica $-1311\ \frac{Kj}{mol}$. Quindi l’elio è più stabile e dovremo spendere tanta energia per separare un elettrone dall’elio\\
\tab- Schermatura da parte di un elettrone addizionale nello stesso orbitale: se c’è più di un elettrone nell’orbitale in cui si trova l’elettrone che sta cercando di fuggire, l’amico lo aiuta a fuggire, quindi sarà più facile separarlo. Si misura quindi la carica nucleare effettiva($Z_{eff}$) cioè la carica nucleare a cui un elettrone è soggetto effettivamente.\\
\tab- Schermatura da parte di elettroni addizionali in orbitali più interni: Nel caso come il litio, l’elettrone che sarà nell'orbitale 2 sarà influenzato anche dalle cariche degli elettroni del primo orbitale. Sarà molto più facile da rimuovere nel caso in cui ci sono due elettroni nel primo orbitale piuttosto che nel caso in cui è da solo\\
\tab- Penetrazione orbitalica: i livelli 2s e 2p hanno diversi valori di energia. questo si vede nel grafico di probabilità radiale, vediamo che l’orbitale s ha un massimo vicino al nucleo più alto rispetto a p, questo significa che l’elettrone su s sarà più attratto dal nucleo e quindi più difficile da rimuovere. Capiamo quindi che a parità di numero di livello, s < p < d < f, in cui s sarà il più vicino al nucleo rispetto a tutti mentre f il più lontano. (COME AL SOLITO CI SONO ECCEZIONI). In generale la penetrazione è il conseguente effetto dello splitting.\\\\
Per capire meglio (pensiero mio) è come se il nucleo avesse solo un tot di potenza, e più elettroni ci sono, meno forza di attrazione hanno ciascuno e sono quindi più facili da strappare\\
Abbiamo due modi per disegnare uno schema di distribuzione degli elettroni negli orbitali.  Entrambi seguono il principio di Aufbau. \\
Metodo 1: Configurazione Elettronica\\
Comprende il livello energetico principale (n), lda designazione letterale del sottolivello(l) e il numero scritto ad apice di elettroni nel sottolivello(\#). La forma generale é \\
Metodo 2: Diagramma degli orbitali\\
E’ costituito da una casella per ciascuno orbitale, in un dato livello energetico, con i livelli raggruppati per sottolivelli e con una freccia che indica la presenza di un elettrone e la direzione orientata del suo spin. (su per $+\frac{1}{2}$, giù per $-\frac{1}{2}$)\\\\
Es:\\
\begin{tabular}{l l l l lll}
  H	&(Z = 1) &=	&$1s^1$&[$\uparrow$]\\
  He&(Z = 2) &= &$1s^2$&[$\uparrow$$\downarrow$]&\\
  Li&(Z = 3) &= &$1s^2\ 2s^1$&[$\uparrow$$\downarrow$]&[$\uparrow$]&[ , , ]\\
  Be&(Z = 4) &= &$1s^2\ 2s^2$&[$\uparrow$$\downarrow$]&[$\uparrow$$\downarrow$]&[ , , ]\\
  B	&(Z = 5) &= &$1s^2\ 2s^2\ 2p^1$&[$\uparrow$$\downarrow$]&[$\uparrow$$\downarrow$]&[$\uparrow$, , ]\\
  C	&(Z = 6) &= &$1s^2\ 2s^2\ 2p^2$&[$\uparrow$$\downarrow$]&[$\uparrow$$\downarrow$]&[$\uparrow$,$\uparrow$, ]\\
  N	&(Z = 7) &= &$1s^2\ 2s^2\ 2p^3$&[$\uparrow$$\downarrow$]&[$\uparrow$$\downarrow$]&[$\uparrow$,$\uparrow$,$\uparrow$]\\
  O	&(Z = 8) &= &$1s^2\ 2s^2\ 2p^4$&[$\uparrow$$\downarrow$]&[$\uparrow$$\downarrow$]&[$\uparrow$$\downarrow$,$\uparrow$,$\uparrow$]\\
  F	&(Z = 9) &= &$1s^2\ 2s^2\ 2p^5$&[$\uparrow$$\downarrow$]&[$\uparrow$$\downarrow$]&[$\uparrow$$\downarrow$,$\uparrow$$\downarrow$,$\uparrow$]\\
  Ne&(Z = 10) &= &$1s^2\ 2s^2\ 2p^6$&[$\uparrow$$\downarrow$]&[$\uparrow$$\downarrow$]&[$\uparrow$$\downarrow$,$\uparrow$$\downarrow$,$\uparrow$$\downarrow$]\\
\end{tabular}\\
Per fare il C dobbiamo utilizzare la regola di Hund.\\
Quando sono disponibili più orbitali di energia della stessa energia, la configurazione elettronica di energia più bassa ha il numero massimo di elettroni spaiati con spin paralleli (Italiano: primo di mettere la freccia in giù dentro un graffa devono essere pieni i posti di frecce in su)\\
Come vediamo dopo N, per mettere O abbiamo messo il primo giu.\\\\
Esiste un modo di scrivere la configurazione elettronica chiamata condensata, ovvero si taglia la parte iniziale del composto sostituendolo con il gas nobile che ha quella configurazione\\\\
Es. \\
Si	(Z = 14) = $1s^2\ 2s^2\ 2p^6\ 3s^2\ 3p^2$ = $[Ne]\ 3s^2\ 3p^2$\\\\
SE ALL’ESAME CHIEDE LA CONFIGURAZIONE ELETTRONICA SIGNIFICA QUELLA COMPLETA\\
Notiamo che nella tavola periodica, tutti gli elementi dello stesso gruppo finiscono allo stesso modo. Inoltre, visto che finiscono tutti allo stesso modo, questo significa che formeranno tutti lo stesso tipo di composto.\\
\tab- Gruppo 1: terminano tutti con , significa che tutti formeranno composti ionici con non metalli, anche detti sali\\
\tab- Gruppo 17: terminano tutti con . Questo significa che esistono come molecole biatomiche(X2). Tutti formano composti ionici con i metalli (MXn), composti covalenti con idrogeno (HX, che danno soluzioni acide in acqua) oppure composti covalenti con il carbonio (CX4)\\
L’orbitale 4s si riempie prima dell’orbitale 3d, è dovuto dagli effetti  di schermatura e di penetrazione visti prima. Questo si può vedere dal grafico probabilistico che analizzavamo prima. Quindi si riempirà prima 4s con 2 elettroni poi si inizia a riempire 3d.\\\\
\begin{tabular}{l l l l lll}
  Cr&(Z = 24)&=&$[Ar]\ 4s^1\ 3d^5$\\
  Cu&(Z = 29)&=&$[Ar]\ 3s^2\ 3d^{10}$\\
\end{tabular}\\
Distinguiamo tre diverse categorie di elettroni:\\
\tab- Interni: quelli del gas nobile che precede l’elemento analizzato (quelli che togliamo quando si fa la v. condensata)\\
\tab- Esterni: quelli del livello energetico più alto\\
\tab- Di valenza: quelli che intervengono nella formazione dei composti, per i gruppi principali questi elettroni sono quelli esterni, mentre negli elementi di transizione la formazione del legame coinvolge spesso anche alcuni elettroni d interni. \\\\
Negli elementi dei gruppi principali, il numero del gruppo è uguale al numero degli elettroni esterni\\
Il numero del periodo è il valore di n del livello energetico più alto occupato.\\
Il valore $n^2$ da il numero totale di orbitali in quel livello energetico, mentre il numero massimo di elettroni in quel livello sarà dato da $2n^2$. \\
es. n = 3\\
Avrà 9 orbitali (un 3s, tre 3p, cinque 3d) e 18 elettroni al massimo\\\\
Nel sesto e settimo periodo della tavola periodica abbiamo i lantanidi e gli attinidi che corrispondono alle 2 righe inserite sotto la tavola periodica, Questi elementi saranno quelli che riempiranno gradualmente i gruppi 4f e 5f.\\
FANNO ECCEZIONE Cr, Cu, Ag, Au, Mo.\\\\
Esistono tre proprietà degli atomi che sono influenzate dalla configurazione elettronica: raggio atomico, energia di ionizzazione e affinità elettronica.
\subsection{Nella tavola periodica}
\subsubsection{Raggio atomico}
Si misura come la distanza tra i nuclei atomici diviso due:\\
\tab- raggio metallico: metà della distanza tra due di atomi adiacenti.\\
\tab- raggio covalente: metà della distanza tra i nuclei di atomi identici legati covalentemente\\\\
Il raggio atomico nella tavola periodica:\\
\tab- aumenta lungo il gruppo\\
\tab- diminuisce lungo un periodo
\subsubsection{Energia di Ionizzazione}
E’ l’energia necessaria per rimuovere 1 mol di elettroni da 1 mol di atomi o ioni gassosi.\\
L’energia di ionizzazione è sempre positiva.\\\\
Li + Energia = $Li^+$\\
Si parla di Energia di 1 Ionizzazione quando ad un atomo neutro sottraiamo un elettrone, come nell’esempio precedente\\
Si parla di Energia di 2 Ionizzazione quando ad uno ione già positivo es(Li+) andiamo a togliere un elettrone. (es. $Li^{++}$)\\
Esistono energia di 3, 4 ionizzazione, etc, come la seconda.\\\\
Nella tavola periodica:\\
\tab- decresce lungo un gruppo: perché il raggio atomico aumenta, aumenta la distanza tra nucleo ed elettrone, perciò è più facile rimuovere gli elettroni\\
\tab\tab- Fa eccezione il gruppo 3, Sende fra B e Al ma poi non scende più\\
\tab- Aumenta lungo un periodo: gli elementi hanno via via un Zeff maggiore, che porta ad un raggio atomico minore e ad un rafforzamento dell’interazione tra nucleo ed elettroni esterni i quali diventano più difficili da rimuovere. 
\subsubsection{Energia di Ionizzazione successive}
E’ l’energia che rimane all’atomo dopo avergli strappato un elettrone. Questa energia non è lineare poichè come abbiamo visto per strappare il secondo elettrone serve molta più energia rispetto allo strappo del primo atomo.\\\\
Es. Be\\
$Be+$ = $0,90\ \frac{Mj}{mol}$\\
$Be^{++}$ = $1,76\ \frac{Mj}{mol}$\\
$Be^{+++}$ = $15,85\ \frac{Mj}{mol}$ //notiamo una grande differenza fra questo e quello precedente perchè questo sta nell’orbitale più vicino al nucleo quindi costa molto di più
\subsubsection{L'affinità elettronica}
E’ il procedimento inverso alla ionizzazione. \\
E’ eseguibile sono in stato gassoso.\\
E’ quel procedimento che permette di aggiungere un elettrone ad un atomo andando a creare uno ione- (anione)\\\\
l’affinità elettronica è un valore generalmente negativo\\
L’andamento in tavola periodica:\\
\tab- decresce lungo un gruppo\\
\tab- aumenta lungo un periodo\\\\
in generale segue lo stesso trend dell’energia di ionizzazione: ma è una grande generalizzazione perché presenta infinite eccezioni.\\
Distinguiamo tre categorie per generalizzare un po meno:\\
\tab- Non metalli reattivi: gli elementi dei gruppi 16 e 17 hanno affinità elettronica negative (esotermiche) in quanto attraggono fortemente elettroni e nei loro composti ionici formano ioni negativi.\\
\tab- Metalli reattivi: gli elementi del gruppo 1 hanno affinità elettroniche lievemente negative. mentre quelli del gruppo 2 hanno affinità eleettroniche positive. In generale gli elementi di questi due gruppi cedono elettroni facilmente, mentre non attraggono molto. formano composti ioni positivi (cationi).\\
\tab- Gas nobili: hanno affinità elettroniche lievemente positive, pertanto non tendono ad acquistare elettroni.
\subsubsection{Comportamento metallico}
esattamente come quelli per la tavola periodico:\\
\tab- Metallo: tipicamente solidi, lucenti, temperature di fusione moderate alte, sono buoni conduttori termici ed elettrici e tendono a cedere elettroni nelle reazioni con non metalli\\
\tab- Non metalli: sono tipicamente non lucenti, basse temperature di fusione, non conduttori e nelle reazioni redox tendono ad acquistare elettroni nelle reazioni coi metalli\\
\tab- Metalloidi hanno proprietà intermedie tra quelle delle due classi superiori\\\\
Il comportamento metallico, fra i metalli, diminuisce lungo un periodo e aumenta lungo un gruppo. \\
Ci sono eccezioni: \\
C che è un non metallo, ma sotto forma di grafite è un buon conduttore elettrico. \\
I è un non metallo, ma è un solido lucente.\\
Ga e Cs sono metalli ma fondono a basse temperature\\
Hg è già liquido a temperatura ambiente\\
\paragraph*{Nelle reazioni redox}:\\
I metalli sono tipicamente riducenti e tendono a cedere elettroni nelle reazioni.\\
I non metalli sono tipicamente ossidanti e tendono ad acquistare elettroni nelle reazioni.\\
\paragraph*{Nelle reazioni Acido-Base}:\\
I metalli si distinguono dai non metalli anche per il comportamento acido-base e dei loro ossidi in acqua:\\
\tab- La maggior parte dei metalli trasferisce elettroni all'ossigeno formando ossidi ionici. Agiscono come basi.\\
\tab- I non metalli condividono elettroni con l'ossigeno quindi ne derivano ossidi covalenti. In acqua questi agiscono come acidi.\\
\tab- Alcuni metalli e molti metalloidi formano ossidi anfoteri, capci di agire sia come acidi che come basi.\\
\paragraph*{Ioni}:\\
Ora analizziamo gli ioni:\\
\tab- Gruppo 18: cercano di mantenere il loro stato poichè hanno già raggiunto la stabilità (anche detta ottetto)\\
\tab- Gruppo 1, 2, 16, 17: I loro ioni sono detti isoelettrici col gas nobile più vicino.\\
\tab- I metalli dei gruppi 13, 14, 15 si comportano differentemente nella formazione di cationi in quanto essi sarebbe energicamente impossibile raggiungere la configurazione esterna di gas nobile, dovrebbero perdere troppi elettroni. Questi elementi raggiungono quindi la configurazione elettronica pseudonobile. Cioè vanno a raggiungere la configurazione elettronica con il sottolivello d pieno. Se lo vogliamo rendere ancora più stabile riempiamo sia il sottolivello d sia s.\\
\paragraph*{Crossover}:\\
Fenomeno che si verifica ad esempio con gli orbitali 3d e 4s. nel momento in cui le due funzioni nel grafico si intersecano c’è uno scambio e l’orbitale 3d diventa più stabile di quello 4s.\\
Questo ci fa capire che in un eventuale processo di ionizzazione noi, nonostante a riempirsi prima sia stato il 4s e successivamente il 3d, dovremo prendere l’elettrone sempre dal 4s, poiché dopo il crossover è l’orbitale più lontano dal nucleo e quindi quello che richiede meno energia.\\
Infatti la formazione di ioni deve seguire le seguenti regole:\\
\tab- Metalli del blocco s: rimozione completa degli elettroni col più alto valore di n. (es. il sodio se ionizzato va a perdere gli elettroni 3s).\\
\tab- Metalli del blocco p: rimozione degli elettroni np prima di quelli ns.\\
\tab- Metalli del blocco d: rimozione degli elettroni ns prima degli elettroni (n - 1)d.\\
\tab- Non metalli: aggiunta di elettroni agli orbitali p con n maggiore.\\
\paragraph*{Elementi parametrici e diamagnetici}:\\
Gli elementi che rimangono con uno o più elettroni spaiati. Come abbiamo visto è sottoposto al fenomeno di splitting, si definisce quindi Paramagnetico. un elemento paramagnetico è affetto dalle cariche elettromagnetiche circostanti\\
Al contrario un elemento con tutti gli elettroni appaiati, allora viene definito elemento diamagnetico. questo elemento non è effetto dal fenomeno di splitting, ne dalle cariche elettromagnetiche circostanti.\\