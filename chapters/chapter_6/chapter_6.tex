\section{La termochimica}
E? un ramo della termodinamica che si occupa dello studio dell'energia nelle reazioni chimiche.
\subsection{Principio della conservazioned dell'energia}
Questo è il primo principio della termodinamica.\\
La somma dell'energia del sistema e dell'ambiente è costante.
\subsection{Unità di misura}
\tab- Joule (j): equazione\\
\tab- Caloria (cal): definita come quantità di calore che si deve formare ad 1g d'acqua per passare da 14,5 a 15,5 gradi Celsius.\\
\tab- 1Cal = 4,18 J\\
\tab- In campo alimentare Cal è usato in modo improprio, si intende Kcal.
\subsection{Funzioni di stato}
E' una funzione di stato una proprietà che dipende solo dallo stato attuale del sistema, non da come si è arrivati a quello stato.\\
Due reazioni di combustione dello stesso elemento, non importa come viene condotto, raggiungeremo sempre lo stesso risultato energetico.
\subsubsection{Entalpia}
I più importanti lavori svolti in chimica sono quello elettrico e quello di espressione di un gas.
\subsubsection{Espansione di un gas}
Essa viene misurata $W = -P * \Delta V$.\\
$\Delta V = $variazione di compressione.
\subsubsection{Energia}
Se una reazione avviene a pressione costante, ovvero quasi tutte, è possibile definire un'Entalpia (H).\\
$\Delta U = q+w = q+(-p*\Delta V) = q-p\Delta V$\\
Sostituiamo q con la variazione di entalpia $\Delta H$ e con una formula inversa troviamo che $\Delta H =\Delta U + p * \Delta V$.\\
Possiamo avere tre tipi di reazione:
\tab- No Gas: i liquidi e solidi hanno piccolissime variazioni di V, quindi $\Delta V = 0$\\
\tab\tab- $\Delta H =\Delta U$ circa\\
\tab- Moli costanti: $\Delta V = 0$\\
\tab\tab- $\Delta H = \Delta U$\\
\tab- Quantità di gas variabile\\
\tab\tab- $\Delta H = \Delta U + P*\Delta V$\\\\
Come prima, se il sistema avrà una variazione di entalpia $\Delta H < 0$ significa che il processo è esotermico perchè rilascia calore.\\
Se il sistema avrà una variazione > 0 allora il processo è endotermico perchè acquista calore.\\\\
Tipi:\\
\tab- Entalpia di combustione: nel caso in cui un elemento va in combustione con O2. (La indichiamo con $\Delta H_{comb}$)\\
\tab- Entalpia di formazione: quando 1 mol di composto viene prodotto dai suoi elementi. (La indichiamo con $\Delta H_f$)\\
\tab- Entalpia di fusione: quando una mol fonde. (La indichiamo con $\Delta H_{fus}$)\\
\tab- Entalpia di vaporizzazione: quando 1 mol di sostanze evapora. (La indichiamo con $\Delta H_{vap}$)\\\\
Nello studio di variazione di entalpia nei gas possiamo unire due formule conosciute in $\Delta H = \Delta U + \Delta N * R * T$ per (T, P const.)
\subsection{Calorimetria}
L'entalpia non è direttamente misurabile in quanto può essere misurata la sua variazione in un periodo di tempo. Per eseguire questa misurazione si deve utilizzare un sistema completamente controllato che non disperda calore, altrimenti la lettura non è veritiera.
\subsubsection{Capacità termica}
indica la capacità che ha un corpo di assorbire calore, ovvero di riscaldarsi con una quantità minore di calore. Viene definita come $C = \frac{q}{\Delta T}$.
\subsubsection{Calore specifico}
Indica la capacità termica per unità di massa. Indica la quantità di calore necessaria per variare di 1K la temperatura della singola unità di massa di sostanza.\\
$c = \frac{q}{\Delta T* m}$
\subsubsection{Calore specifico molare}
Indica la quantità di calore necessaria per variare di 1K la temperatura di 1 mol di sostanza.\\
$c_m = \frac{q}{\frac{\Delta T}{n}}$
\subsubsection{Strumentazione}
\tab- Calorimetro a pressione costante.\\
\tab- Calorimetro a volume costante.\\\\
Un metodo per registrare la variazione di entalpia per una particolare reazione è l'equazione termochimica, cioè un'equazione bilanciata che indica anche i calori di reazione($\Delta H_r$).
\subsection{Legge di Hess}
Utilizzata per calcolare la variazione di entalpia senza dover eseguire l'operazione in laboratorio.\\
Procedimento:\\
\tab- Scrivere l'equazione chimica il cui $\Delta H$ è incognito.\\
\tab- Scrivere le equazioni chimiche il cui $\Delta H$ è noto.\\
\tab- Manipolare quest'ultime equazioni in modo che i loro reagenti e prodotti si trovino dalla parte corretta dell'equazione a $\Delta H$ incognito.\\
\tab\tab- il segno $\Delta H$ va cambiato se la relazione viene invertita.\\
\tab\tab- se la relazione viene moltiplicata per un numero, anche $\Delta H$ va moltiplicato per il numero.\\
\tab- Sommare le equazioni a $\Delta H$ noto, verificando che l'equazione somma coincida con l'equazione di cui occorre determinare $\Delta H$.
\subsubsection{Tabulazioni}
Molti dati termodinamici sono tabulati facendo riferimento a particolari stati standard, cioè un insieme di condizioni e concentrazioni specifiche.\\
\tab- Gas: lo stato standard è 1atm e si assume un comportamento ideale.\\
\tab- Sostanza in soluzione acquosa: lo stato standard prevede una concentrazione pari a 1mol/L\\
\tab- Sostanza pura: lo stato standard è la forma più stabile della sostanza. A 1atm e 25C.\\\\
Si contrassegna una grandezza espressa allo stato standard ponendo uno 0 ad apice.\\
Calore standard di formazione ($\Delta H^0_f$) è la variazione di entalpia di una reazione di formazione quando tutte le sostanze sono nei loro stati standard. Anche essi sono tabulati.\\
Se $\Delta H^0_f < 0$ significa che il composto è più stabile dei suoi elementi costituenti.\\\\
Da dove viene questa energia?\\
\tab- Contributo dell'energia cinetica: la molecola si muove attraverso lo spazio, la molecola ruota, gli atomi legati vibrano, gli elettroni si muovo entro ciascun atomo. I primi tre contributi sono direttamente proporzionali alla temperatura assoluta (costante), e quindi non variano. Anche il modo degli elettroni non cambia perchè non è influenzato dalla reazione.\\
\tab- Contributi all'energia potenziale: forze tra gli atomi legati vibranti, forze tra nucleo ed elettroni in ciascun atomo, forze tra protoni e neutroni in ciascun nucleo, forze tra i nuclei e la coppia di elettroni condivisa in ogni legame.\\
