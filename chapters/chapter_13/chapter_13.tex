\section{Proprietà delle miscele, soluzioni e colloidi}
\subsection{Soluzioni}
Miscele omogenee, di una sola fase, senza interfacce fra i suoi componenti.\\
Composta da un Solvente(componente maggioritario) e uno o più soluti.\\
Si crea per dissoluzione dei suoi componenti\\
Esistono tre tipi di soluzioni, quelle solide(come le leghe), liquide e gassose\\
Si formerà una soluzione di uno di questi tre tipi in base allo stato di aggregazione del solvente, in quanto è componente maggioritario. In generale diciamo che “simile scioglie simile”.\\
I due componenti sono quindi miscibili perché possono mescolarsi mantenendo una soluzione omogenea con qualunque proporzioni
\subsection{Solubilità (S)}
Quantità massimo di soluto che si sciogli e in una quantità fissa di un particolare solvente ad una precisa temperatura\\
Diluito e concentrato soon due termini quantitativi che si usano per capire che, nel primo caso il soluto è poco presente, mentre nel secondo caso il soluto è molto presente.
\subsection{Colloidi}
Miscela eterogenea, con più fasi. Caratterizzati da un ocmponente disperso sotto forma di particelle di piccolo diametro in un altro colloide. Es. Latte, Fumo.\
Si crea per dispersione
\subsection{Forze intermolecolari nei liquidi}
\subsubsection{Forze one-dipolo}
Le forze ione-dipolo sono molto importanti nella solubilità dei composti ionici.\\
Quando NaCl (sale da cucina) si va a sciogliere in acqua, ogni ione che forma la superficie del cristallo attra l’estremità caricata di segno opposto di un dipolo d’acqua. Queste forze attrattive vincono le forze interioniche e demoliscono la struttura del cristallo. Via via che ciascuno ione si separa, aumenta il numero delle molecole d’acqua che si aggregano attorno ad esso in un guscio di idratazione (o sfera di idratazione). Le molecole d’acqua nel guscio di idratazione sono legate con ponti ad idrogeno ad altre molecole d’acqua un pò più lontane dallo ione. Il numero di molecole d’acqua nel guscio di idratazione dipende dal raggio dello ione, può essere 4 per Na e 6 per Li. 
\subsubsection{Forze dipolo-dipolo e legami ad idrogeno}
sono fattori primari nella capacità dell'acqua di sciogliere numerosi composti organici e biologici ossigenati e azotati (alcoli, zuccheri, ammine, amminoacidi, ...).
\subsubsection{Forze tra carica e dipolo-indotto}
Si basa sulla polarizzabilità dei componenti.\\
Es. $Fe_2^+$ dell'emoglobina con $O_2$
\subsubsection{Forze di dispersione di London}
Le principali forze attrattive nelle soluzioni delle sostanze apolari.\\
Es. petrolio
\subsection{Tipi di soluzioni}
\subsubsection{Soluzioni liquido-liquido e solido-liquido}
Questa reazione si viene a formare se tra solvente e soluto c’è interazione simile o maggiore rispetto a quelle che devono essere distrutte all’interno del solvente e del soluto.\\
Osserviamo una serie di Alcoli (molecole organiche contenenti una catena di C e un gruppo finale -OH polare) e li poniamo in due solventi\\
\tab- In Acqua: qui osserviamo che fra alcoli e acqua si creano:\\
\tab\tab- legami ad idrogeno, grazie al gruppo -OH\\
\tab\tab- forze dipolo-dipolo indotto fra le molecole di C. \\
\tab- In Esano(solvente apolare costituito da una serie di C): qui osserviamo che abbiamo:\\
\tab\tab- forze dipolo-dipolo indotto dove -OH fa da dipolo\\
\tab\tab- forze di dispersione di london fra le catene di atomi di C  \\\\
Se andiamo ad aumentare il numero di atomi di C nei nostri alcoli la molecola diventerà sempre più apolare, vedremo che in acqua diventerà sempre meno solubile perchè aumentano le forze dipolo-dipolo indotto, e vanno a superare quelle dei legami ad idrogeno. Mentre nell’esano osserviamo il contrario, che le forze di dispersione continuano ad aumentare favorendo la solubilità
\subsubsection{Soluzioni gas-liquido}
I gas apolari non molto solubili in acqua perchè le forze sono troppo deboli.\\
In alcuni casi vediamo però $CO_2$ che ha alta solubilità in acqua, ma in realtà quello che sta avvenendo è una reazione chimica.
\subsubsection{Soluzioni gas-gas}
I gas fra loro sono infinitamente solubili l'uno nell'altro, vedo l'aria.
\subsubsection{Soluzioni gas-solido}
un gas è in grado di occupare gli spazi fra le particelle di un'impaccatura solida. Questo è spesso indesiderato come per esempio nell'industria del rame.
\subsubsection{Soluzioni solido-solido}
I solidi si diffondono pochissimo, quindi le loro miscele sono solitamente eterogenee, (es. ghiaia e sabbia).\\
Si aggiungono poi le leghe, soluzioni solido-solido omogenee nella maggior parte dei casi.
\subsection{Variazioni di energia}
Affinchè una sostanza si sciolga in un’altra  passiamo per diverse tappe:\\
\tab- Le particelle di soluto si separano l’una dall’altra: questo processo è endotermico quindi con $\Delta H_{soluto} > 0$\\
\tab- Le particelle di solvente si separano fra loro e questo processo è endotermico quindi $\Delta H_{solvente} > 0$\\
\tab- Infine le particelle di soluto si legano con quelle di solvente. Questo procedimento è esotermico quindi $\Delta H_{mesc} < 0$ \\\\
La sommatoria di questi tre step è detta Entalpia di soluzione ($\Delta H_{soluzione}$). il risultato può essere sia positivo che negativo.\\
In questo corso andremo ad analizzare solo variazioni di energia in cui il solvente è l’acqua
\subsubsection{$\Delta H$ idratazione}
Il $\Delta H_s{solvente}$ e $\Delta H_{mescolamento}$ possono essere difficili da misurare, ma dal momento che abbiamo già questi valori per ‘acqua chiamare la somma di questi due deltaHidratazione, da nome del processo Idratazione.\\
Il $\Delta H_{soluzione}$ sara ora uguale a $\Delta H_{idr} + \Delta H_{soluto}$\\
Il $\Delta H_{idr}$ sarà sempre negativo poichè la rottura di molti legami ad idrogeno nell’acqua richiede meno energia rispetto alle parecchie forze ione-dipolo.\\
Più l’elemento è grosso, meno sarà il $\Delta H_{idr}$
\subsubsection{$\Delta H$ soluto}
a