\section{Proprietà delle miscele, soluzioni e colloidi}
\subsection{Soluzioni}
Miscele omogenee, di una sola fase, senza interfacce fra i suoi componenti.\\
Composta da un Solvente(componente maggioritario) e uno o più soluti.\\
Si crea per dissoluzione dei suoi componenti\\
Esistono tre tipi di soluzioni, quelle solide(come le leghe), liquide e gassose\\
Si formerà una soluzione di uno di questi tre tipi in base allo stato di aggregazione del solvente, in quanto è componente maggioritario. In generale diciamo che “simile scioglie simile”.\\
I due componenti sono quindi miscibili perché possono mescolarsi mantenendo una soluzione omogenea con qualunque proporzioni
\subsection{Solubilità (S)}
Quantità massimo di soluto che si sciogli e in una quantità fissa di un particolare solvente ad una precisa temperatura\\
Diluito e concentrato soon due termini quantitativi che si usano per capire che, nel primo caso il soluto è poco presente, mentre nel secondo caso il soluto è molto presente.
\subsection{Colloidi}
Miscela eterogenea, con più fasi. Caratterizzati da un ocmponente disperso sotto forma di particelle di piccolo diametro in un altro colloide. Es. Latte, Fumo.\
Si crea per dispersione
\subsection{Forze intermolecolari nei liquidi}
\subsubsection{Forze one-dipolo}
Le forze ione-dipolo sono molto importanti nella solubilità dei composti ionici.\\
Quando NaCl (sale da cucina) si va a sciogliere in acqua, ogni ione che forma la superficie del cristallo attra l’estremità caricata di segno opposto di un dipolo d’acqua. Queste forze attrattive vincono le forze interioniche e demoliscono la struttura del cristallo. Via via che ciascuno ione si separa, aumenta il numero delle molecole d’acqua che si aggregano attorno ad esso in un guscio di idratazione (o sfera di idratazione). Le molecole d’acqua nel guscio di idratazione sono legate con ponti ad idrogeno ad altre molecole d’acqua un pò più lontane dallo ione. Il numero di molecole d’acqua nel guscio di idratazione dipende dal raggio dello ione, può essere 4 per Na e 6 per Li. 
\subsubsection{Forze dipolo-dipolo e legami ad idrogeno}
sono fattori primari nella capacità dell'acqua di sciogliere numerosi composti organici e biologici ossigenati e azotati (alcoli, zuccheri, ammine, amminoacidi, ...).
\subsubsection{Forze tra carica e dipolo-indotto}
Si basa sulla polarizzabilità dei componenti.\\
Es. $Fe_2^+$ dell'emoglobina con $O_2$
\subsubsection{Forze di dispersione di London}
Le principali forze attrattive nelle soluzioni delle sostanze apolari.\\
Es. petrolio
\subsection{Tipi di soluzioni}
\subsubsection{Soluzioni liquido-liquido e solido-liquido}
Questa reazione si viene a formare se tra solvente e soluto c’è interazione simile o maggiore rispetto a quelle che devono essere distrutte all’interno del solvente e del soluto.\\
Osserviamo una serie di Alcoli (molecole organiche contenenti una catena di C e un gruppo finale -OH polare) e li poniamo in due solventi\\
\tab- In Acqua: qui osserviamo che fra alcoli e acqua si creano:\\
\tab\tab- legami ad idrogeno, grazie al gruppo -OH\\
\tab\tab- forze dipolo-dipolo indotto fra le molecole di C. \\
\tab- In Esano(solvente apolare costituito da una serie di C): qui osserviamo che abbiamo:\\
\tab\tab- forze dipolo-dipolo indotto dove -OH fa da dipolo\\
\tab\tab- forze di dispersione di london fra le catene di atomi di C  \\\\
Se andiamo ad aumentare il numero di atomi di C nei nostri alcoli la molecola diventerà sempre più apolare, vedremo che in acqua diventerà sempre meno solubile perchè aumentano le forze dipolo-dipolo indotto, e vanno a superare quelle dei legami ad idrogeno. Mentre nell’esano osserviamo il contrario, che le forze di dispersione continuano ad aumentare favorendo la solubilità
\subsubsection{Soluzioni gas-liquido}
I gas apolari non molto solubili in acqua perchè le forze sono troppo deboli.\\
In alcuni casi vediamo però $CO_2$ che ha alta solubilità in acqua, ma in realtà quello che sta avvenendo è una reazione chimica.
\subsubsection{Soluzioni gas-gas}
I gas fra loro sono infinitamente solubili l'uno nell'altro, vedo l'aria.
\subsubsection{Soluzioni gas-solido}
un gas è in grado di occupare gli spazi fra le particelle di un'impaccatura solida. Questo è spesso indesiderato come per esempio nell'industria del rame.
\subsubsection{Soluzioni solido-solido}
I solidi si diffondono pochissimo, quindi le loro miscele sono solitamente eterogenee, (es. ghiaia e sabbia).\\
Si aggiungono poi le leghe, soluzioni solido-solido omogenee nella maggior parte dei casi.
\subsection{Variazioni di energia}
Affinchè una sostanza si sciolga in un’altra  passiamo per diverse tappe:\\
\tab- Le particelle di soluto si separano l’una dall’altra: questo processo è endotermico quindi con $\Delta H_{soluto} > 0$\\
\tab- Le particelle di solvente si separano fra loro e questo processo è endotermico quindi $\Delta H_{solvente} > 0$\\
\tab- Infine le particelle di soluto si legano con quelle di solvente. Questo procedimento è esotermico quindi $\Delta H_{mesc} < 0$ \\\\
La sommatoria di questi tre step è detta Entalpia di soluzione ($\Delta H_{soluzione}$). il risultato può essere sia positivo che negativo.\\
In questo corso andremo ad analizzare solo variazioni di energia in cui il solvente è l’acqua
\subsubsection{$\Delta H$ idratazione}
Il $\Delta H_s{solvente}$ e $\Delta H_{mescolamento}$ possono essere difficili da misurare, ma dal momento che abbiamo già questi valori per ‘acqua chiamare la somma di questi due deltaHidratazione, da nome del processo Idratazione.\\
Il $\Delta H_{soluzione}$ sara ora uguale a $\Delta H_{idr} + \Delta H_{soluto}$\\
Il $\Delta H_{idr}$ sarà sempre negativo poichè la rottura di molti legami ad idrogeno nell’acqua richiede meno energia rispetto alle parecchie forze ione-dipolo.\\
Più l’elemento è grosso, meno sarà il $\Delta H_{idr}$
\subsubsection{$\Delta H$ soluto}
Il $\Delta H_{soluto}$ è uguale a - $\Delta H_{reticolare}$. \\
Ne consegue che $\Delta H_{soluto}$ > 0\\\\
Quindi:\\
$\Delta H_{soluzione} = -1 * \Delta H_{reticolare} + \Delta H_{idr}$
\subsection{Entropia (S)}
La tendenza del sistema a distribuire la propria energia nel maggior numero di modi possibili. \\
La terza funzione di stato che andiamo a studiare. \\
E' direttamente proporzionale al numero di modo in cui le particelle possono trovarsi in quello stato di aggregazione \\
Possiamo già sapere che $S_{gas} > S_{liquido} > S_{solido}$. \\
Allo stesso modo $\Delta S_{vaporizzazione} > 0 e \Delta S_{fusione} > 0$\\
Sappiamo inoltre che $\Delta S_{soluzione} > 0$\\
La tendenza naturale della materia è quella di avere minore entalpia e maggiore entropia. \\\\
Esempi:\\
NaCl in Esano:\\
Non si scioglie perchè la somma delle energie delle disgregazioni di solvente è soluto è mooolto grande, mentre il delta di mescolamento è piccolo poichè le interazioni che si formano sono deboni, con una risultante molto alta, troppo alta.\\\\
Ottano in esano:\\
In questo caso andiamo a  disgregare Ottano ed Esano rompendo le loro forze di london, per poi riaggregarli insieme creando altre forze di london. Essendo tutte forze di london il DeltaHsoluzione è circa 0.\\\\
$NH_4NO_3$ in acqua:\\
Si scioglie perchè il forte aumento di entropia prevale sul parallelo aumento di entalpia 
\subsection{Solubilità}
Nel caso in cui noi poniamo una polvere allo stato solido in un liquido  succederanno due cose contemporaneamente. nel solido si scioglie in acqua mentre una parte del liquido ricristallizza tornando solito. Nel caso in cui noi mettiamo troppo composto solido noteremo che sul fondo si formerà un micchietto del nostro materiale. in realtà non è sempre lo stesso ma continua a cambiare per i motivi sopra descritti. Nel momento in cui il numero di particelle che cristallizzano e che si sciolgono è uguale, la soluzione si dice satura.\\
Soluzione satura: contiene la quantità massima di soluto disciolto a una data temperatura in presenza di un soluto indisciolto\\
In alcuni casi si riesce a far sciogliere più soluto di quello previsto dalla soluzione. In questo caso viene detta soluzione soprassatura, un sistema in equilibrio metastabile
\subsubsection{Lasolubilità in confronto alla temperatura}
LA solubilità generalmente aumenta con l'aumentare della temperatura.\\
C'è un eccezione che è il calore. $Ce_2(SO_4)_3$
\subsubsection{Solubilità dei gas}
I gas hanno deltaHsoluto circa 0, quindi, avendo fissato prima deltaHidr < 0, avremo un deltaHsoluzione < di 0. Questo significa che la reazione fra liquido e gas sarà esotermica e produrrà calore. \\
Inoltre, si capisce che se scaldiamo il sistema acqua andiamo contro le necessità del gas quindi la solubilità diminuisce. (contrariamente ai solidi)\\
I gas osno influenzati dalla pressione che viene posta su di loro durante il processo di dissoluzione. \\\\
Si può definire con Sgas = Kh * Pgas\\
Con Sgas = solubilità del gas\\
Kh = costante di Henry differente per ogni gas [mol / (L * Atm)]\\
Pgas = Pressione del gas. \\\\
Il volume è influenzato dalla temperatura.
\subsection{Concentrazione volumica}
Effetto che si ha unendo due liquidi di volume molare diverso, otterremo che il volume finale sarà minore della somma dei volumi iniziali, poiché il liquido con particelle più piccole va ad occupare un pò dello spazio lasciato dalle particelle grandi. \\\\
Quindi nello svolgimento dei problemi, possiamo sommare due volumi solo se:\\
\tab1) si satnno miscelando soluzione aventi lo stesso solvente, es. NaCl in $H_2O$ e LiCl in $H_2O$\\
\tab2) si può fare se il testo riporta “si considerino i volumi additivi”
\subsection{Diluizione}
Diluire una soluzione significa in gergo “allungarla” ovvero aggiungere altro solvente in modo da abbassare la concentrazione di soluto in soluzione\\
Il numero di moli di soluto non cambiano.\\
formula: $M_i * V_i = M_f * V_f$ Equazione di diluizione
\subsection{Proprietà colligative}
Sono quelle proprietà influenzate dal numero di particelle di soluto e non identità
\subsection{Divisione elettrolita dei soluti}
\tab- Elettroliti Forti: Il soluto si dissocia completamente in cationi e anioni, e la soluzione risultante conduce corrente elettrica di grande intensità. Ne sono esmpi i sali solubili in acqua, gli acidi forti e le basi forti.\\
\tab- Elettroliti Deboli: il soluto si dissocia solo parzialmente in cationi e anioni, la soluzione risultante conduce corrente elettrica di piccola intensità. Ne sono esempi acidi deboli e basi deboli. \\
\tab- Non elettroliti: i soluti non danno dissociazione ionica, e le soluzioni risultanti non conducono corrente elettrica. Ne sono esempi molti composti organici.  \\\\
Consideriamo ora le proprietà colligative di soluti non elettroliti e non volatili (es. Saccarosio)\\
Proprietà colligative:\\
\tab- Abbassamento della tensione di vapore: La tensione di vapore di una soluzione è sempre più bassa della tensione di vapore del solvente puro. Ciò si poteva dedurre dal fatto che la natura tende al disordine creando i solventi puri, quindi con tensione più alta. Attenzione, il soluto deve essere NON volatile\\
\tab- Innalzamento ebullioscopico: La  temperatura di ebollizione di una soluzione è più alta di quella del solvente puro. La differenza può essere misurato con un termine deltaTeb, innalzamento ebullioscopico. Esso sarà sempre > 0. Ricordare che Keb nella formula dipende solo dal solvente. \\
\tab\tab- Da ricordare che Keb, costante ebullioscopica del solvente, ha come unità di misura [C * Kg / Mol]\\
\tab\tab- Per l’acqua, che bolle a 100C abbiamo un Keb di 0.512\\
\tab- Abbassamento Crioscopico: La temperatura di solidificazione di una soluzione è più bassa di quella del solvente puro. Questo è dovuto all’abbassamento della tensione di vapore. Tramite formula otteniamo quindi il deltaTcr, sempre > 0, dove indica l’abbassamento crioscopico\\
\tab\tab- Da ricordare che Kcr è la costante crioscopica del solvente con unità di misura di [C * Kg / Mol]\\
\tab\tab- Per l’acqua che solidifica a 0 gradi vale 0,18.\\
\tab- Pressione osmotica: la pressione osmotica è quella pressione che dobbiamo esercitare per impedire l’osmosi. L’osmosi è quel fenomeno che avviene ponendo due liquidi in un tubo a forma di U, il quale presenta una membrana al centro che permette solamente all’acqua di passare. Se noi non applichiamo nessuna pressione dopo un po otterremo l’equilibrio osservando che in questo stato uno dei due liquidi sarà più in alto. La pressione osmotica è quindi la pressione che serve esercitare sul liquido più alto per raggiungere livello pari all’altro.
\subsection{Soluzioni reali}
Sono:\\
\tab- Soluzioni diluite, con concentrazione $10^0$, massimo $10^1$ molare.\\
\tab- La tensione di vapore del soluto è trascurabile rispetto a quella del solvente\\
\tab- Il soluto non presenta fenomeni associativi o dissociativi \\
\tab- Il soluto non deve reagire con il solvente\\
\tab- Le interazioni intermolecolari fra soluto e solvente, solvente solvente, e soluto soluto devono essere uguali. 
\subsection{Proprietà colligative con soluzioni elettroliti e non volatili}
Si introduce un fattore moltiplicativo all'interno dell'equazione chiamata coefficiente di Van't Hoff.\\
Nella formula compare alfa, esso è una percentuale (\%) che indica il grado di dissociazione, nel caso di elettroliti forti sarà 100\%\\
\begin{tabular}{|l|l|l|}\hline
  Tipo & Sostanze & Esempio \\\hline
  Aerosol & Liquido-gas & Nebbia\\\hline
  Aerosol & Solido-gas & Fumo\\\hline
  Schiuma & Gas-liquido & Panna montata\\\hline
  Schiuma solida & Gas-solido & Caramella gommosa\\\hline
  Emulsione & Liquido-liquido & Latte\\\hline
  Emulsione solida & Liquido-liquido & Burro\\\hline
  Sol & Solido-liquido & Vernice\\\hline
  Sol solido & Solido-solido & Opale\\\hline
\end{tabular}
\subsection{Effetto Tyndall}
Effetto per cui quando un fascio di luce attraversa un colloide noi osserviamo quel fascio in diffusione molto chiaramente. (Luce fra le finestre e si vede le particelle volare).\\
E' dovuto dal fatto che un colloide disperde casualmente le onde di luce poichè la lunghezza d'onda ha dimensione simile a quella della particella di colloide.
\subsection{Modo Browniano}
Il moto che assumono le particelle di un colloide, è un moto continuo con particelle in sospensione. Le particelle riescono a rimanere in sospensione grazie alle spinte causate dalle particelle del mezzo disperdente.\\
Un colloide può essere distrutto grazie al riscaldamento del corpo, oppure grazie ad un elettrolita.
\subsection{Micelle}
I più comuni colloidi sono le Micelle\\
Si tratta di aggregati colloidali di molecole formate da un tensioattivo in soluzione. Se questi tensioattivi vengono posti in quantità maggiore rispetto ad un minimo chiamato concentrazione micellare critica (CMC), essi vanno a raggruppare tutte le code verso l’interno e le teste verso l’esterno.\\
Questo è il principio della maggior parte dei lavaggi, quando laviamo dei tensioattivi vanno a circondare lo sporco, catturandolo e portandolo con se. 