\section{L'equilibrio, l'entità delle reazioni chimiche}
\subsection{Equilibio chimico}
Si riferisce all'entità di una reazione, cioè la concentrazione di un reagente/prodotto dopo un tempo illimitato o quando non avvengono ulteriori variazioni quantitative.
\subsection{Cinetica chimica}
Si applica alla velocità di una reazione, ovvero alla variazione della concentrazione di una specie nell'unità di tempo.
\subsection{Reazioni Reversibili}
Si dicono reversibili quelle reazioni che possono essere eseguite in entrambe le direzioni, si indicano con $\Leftrightarrow $\\
Attenzione perchè molte volte quando il sistema ha raggiunto l'equilibrio, questo non significa che non stanno più avvenendo reazioni (equilibrio statico), potrebbe essere, ma nella maggior parte dei casi si parla di equilibrio dinamico, ovvero che la reazione sta avvenendo in entrambe le direzioni contemporaneamente, e, nel caso avvenisse in pari quantità, ad occhio umano sembrerebbe che non avvenga nulla.
\subsection{Legge di azione di massa (o dell'equilibrio chimico)}
Una volta raggiunto il nostro equilibrio dinamico noi avremo (ad una data temperatura) che il rapporto delle concentrazioni sia costante.\\
Definiamo quindi $K_c$ ocme costante di equilibrio in termini di molarità.\\
$K_c = \frac{\left[prodotto\right]^p}{\left[reagente\right]^r}$\\
Qual'ora ci siano più prodottii e/o reagenti, si porranno a numeratore o denominatore i prodotti delle molarità dei singoli prodotti e/o reagenti.\\
Se $K_c < 1$ allora la reazione tende ai reagenti, significa che nel nostro equilibrio abbiamo più reagenti che prodotti.\\
Se $K_c = 1$ significa che la quantità di reagenti è uguale a quella di prodotti.\\
Se $k_c > 1$ significa che la quantità di prodotto è maggiore a quella dei reagenti.\\
Questa costante di equilibrio può essere aclcolata in ogni momento ma assumerà un nome diverso, quoziente di reazione($Q_c$).\\
Nel momento in cui $K_c$ e $Q_c$ saranno uguali significa che abbiamo raggiunto l'equilibrio.\\
Le loro unità di misura dipendono dalla formula, quinid si omettono.\\
Per ottenere $Q_c$ diretta = $\frac{1}{Q_c}$ inversa.
\subsection{Risoluzione dei problemi di equilibrio}
\tab- E’ fondamentale determinare se vogliamo usare la molarità oppure la pressione parziale oppure in moli\\
\tab\tab- se utilizziamo le moli, al termine dell’esercizio dobbiamo trasformarli in pressioni parziali o molarità.\\
\tab\tab- se il volume non è specificato, il volume non è necessario, significa che abbiamo lo stesso volume sopra e sotto e quindi si semplifica. \\
\tab- La percentuale in volume, per un gas perfetto, coincide numericamente con una percentuale in moli.\\
\tab- Se nel testo del problema mi viene dato sia i reagenti che i prodotti, occorre considerare che la reazione potrebbe procedere da destra verso sinistra. \\
\tab\tab- si confrontano K e Q, se la reazione procede verso sinistra si porranno prima i termini +x e -x nello schema IVE per reagenti e prodotti, rispettivamente\\
\tab\tab- oppure se non abbiamo K e Q, si procede per una risoluzione standard con  lo schema IVE, ma l’esercizio potrà essere risolto solo prendendo il valore negativo di x uscito dalla risoluzione dell’equazione di secondo grado. \\
\tab- Molti problemi non danno lo stato di aggregazione, significa che devo sapere in base alla temp in che stato si trova. va quindi capito e indicato fin da subito. \\
\tab\tab- Lo schema IVE si fa solo per lo stato gassoso.
\subsection{Principio di Le Chatelier}
Afferma: Quando un sistema chimico in equilibrio viene perturbato, esso ritorna all'equilibrio subendo una reazione netta che riduce l'effetto della perturbazione
\subsubsection{Perturbazione}
Si indica che è stata condotta una perturbazione nel sistema avente come effetto un passaggio da $Q = K$ a $Q \ne K$.\
Attenzione! K non varia, Q varia a temperatura costante.\\
Perturbazioni comuni:\\
\tab- Variazione della concentrazione.\\
\tab- Variazione di pressione o volume.\\ 
\tab- Variazione di temperatura.\\\\
\paragraph*{Variazione di concentrazione}:\\
Se all'equilibrio la concentraione di un reagente/prodotto aumenta, il sistema reagisce in modo da consumare una certa quantità e viceversa.
\paragraph*{Variazione di pressione}:\\
Se il volume del recipiente viene ritto, la pressione delle specie gassose aumenta immediatamente. Il sistema reagisce quindi riducendo il numero di molecole gassose spostando la reazione verso il membro con meno moli di gas.
\paragraph*{Aggiunta di un gas inerte}:\\
L'aggiunta di un gas non presente nella reazione iniziale non ha effetto sulla posizione dell'equilibrio, se il volume è costante.\
\paragraph*{Variazione di temperatura}:\\
Questa è l'unica perturbazione che può modificare K\\
Un aumento di temperatura favorirà una reazione endotermica.\\
Una diminuzione di temperatura favorirà una reazione esotermica.\\
La modifica della temperatura ha un effetto anche su equilibri non gassosi.
\subsection{Reazione netta}
Risposta del sistema alla perturbazione, costituisce la direzione che il sistema assume al fine di ristabilire l'equilibrio perso.\
Il sistema che ora è posto in una situazione $K \ne Q$, cerca di ritrovare l'equilibrio $Q = K$.